\include{template_a5}

\cfoot{} %% if no page number is needed

\begin{document}

\begin{header}
CQFR -- Chapitre 2

Solutions aqueuses
\end{header}

À la fin du chapitre, je suis capable de :
\begin{itemize}
\item[•] définir les termes suivants : solvant, soluté ;
\item[•] identifier le solvant et le soluté connaissant la composition ou le mode de préparation d'une solution;
\item[•] différencier masse volumique et concentration massique ;
\item[•] dire que la concentration massique d'un soluté ne peut dépasser une valeur maximale ;
\item[•] déterminer la concentration massique d'un soluté connaissant le mode de préparation d'une solution ;
\item[•] déterminer la solubilité d'un soluté en utilisant des résultats expérimentaux ;
\item[•] préparer une solution par dissolution ou dilution en utilisant le matériel adapté ;
\item[•] justifier mes choix de matériel, verrerie en comparant leur précision ;
\item[•] déterminer la concentration d'une solution en utilisant une échelle de teinte, une courbe d'étalonnage ;
\item isoler chaque terme d'une formule du type $a=\dfrac{b}{c}$.
\end{itemize}

\end{document}