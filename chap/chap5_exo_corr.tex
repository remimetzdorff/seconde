\documentclass[12pt,a4paper,fleqn]{article}
\usepackage{rmpackages}																% usual packages
\usepackage{rmtemplate}																% graphic charter
\usepackage{rmexocptce}																% for DS with cptce eval

%\cfoot{} 													% if no page number is needed
%\renewcommand\arraystretch{1.5}		% stretch table line height

\begin{document}

\begin{header}
Chapitre 5 -- Exercices
\end{header}

\section*{Corrections}

\subsection*{Exercice 4 page 216}

\begin{enumerate}
\item La vitesse du son dans l'air est $v_\mathrm{son} \approx \unit{340}{m/s}$.

\item
\begin{enumerate}
\item Sur le dessin, le son se propage dans l'air et les rails (acier).
\item La vitesse du son dans l'acier est plus grande que dans l'air ($v_\mathrm{son\ acier} \approx \unit{5\,000}{m/s}$).
\end{enumerate}
\end{enumerate}

\subsection*{Exercice 5 page 216}

\begin{enumerate}
\item
\[
v = \frac{d}{\Delta t}
\]
Dans l'air :
\[
v = \frac{1000}{3{,}0} \approx \unit{333}{m/s}.
\]
Dans l'eau liquide :
\[
v = \frac{15}{1{,}0\times 10^{-2}} = \unit{1500}{m/s} = \unit{1{,}5\times 10^{3}}{m/s}.
\]

\item Le son se propage plus vite dans l'eau que dans l'air.
De manière générale, le son se propage plus vite dans un liquide que dans un gaz.
\end{enumerate}

\subsection*{Exercice 6 page 216}

Sur le graphique, on mesure \unit{1{,}3}{cm} pour une période or \unit{5{,}6}{cm} correspond à \unit{10}{ms} sur l'axe des abscisses :
\begin{center}
\begin{tabular}{c|c}
\unit{5{,}6}{cm} & \unit{10}{ms} \\
\hline
\unit{1{,}3}{cm} & $T$ 
\end{tabular}
\end{center}
\[
T = \frac{1{,}3 \times 10}{5{,}6} \approx \unit{2{,}3}{ms}.
\]

\paragraph{Remarque :} Le résultat de cette mesure est plus précis si on mesure plusieurs périodes. Sur le graphique 4 périodes correspondent à \unit{5{,}1}{cm} donc $4T=\unit{9{,}1}{ms}$ et $T=\unit{2{,}27}{ms}$.

\paragraph{Remarque :} La fréquence $f$ de ce signal est $f=\frac{1}{T} \approx \unit{440}{Hz}$ ce qui correspond bien à la fréquence d'un diapason classique.

\subsection*{Exercice 7 page 217}

\begin{enumerate}
\item On sait que 
\[f = \frac{1}{T},\]
donc 
\[T = \frac{1}{f}.\]
L'application numérique donne :
\[T = \frac{1}{380} \approx \unit{0{,}0026}{s} = \unit{2{,}6}{ms} . \].
\end{enumerate}

\subsection*{Exercice 11 page 217}

\paragraph{Sans calcul :}
La période du son B est plus courte que celle du son A donc la fréquence du son B est plus élevée que celle du son A.
Le son B est donc plus aigu que le son A.

\paragraph{Avec calcul :}
La démarche est la même que pour l'exercice 6 page 216.

L'échelle des abscisses est la même sur les deux graphiques : \unit{1{,}2}{cm} correspond à \unit{5}{ms}.
\begin{itemize}
\item[•] Son A : sur le graphique, une période mesure \unit{1{,}1}{cm} :
\begin{center}
\begin{tabular}{c|c}
\unit{1{,}2}{cm} & \unit{5}{ms} \\
\hline
\unit{1{,}1}{cm} & $T$ 
\end{tabular}
\end{center}
\[
T_A = \frac{1{,}1 \times 5}{1{,}2} \approx \unit{4{,}6}{ms} = \unit{4{,}6\times10^{-3}}{s},
\]
donc
\[
f_A = \frac{1}{T_A} = \frac{1}{4{,}6\times10^{-3}} \approx \unit{217}{Hz}.
\]
\item[•] Son B : en suivant la même méthode, on trouve $T_B \approx \unit{2{,}3}{ms}$ et $f_B \approx \unit{435}{Hz}$.
\end{itemize}
Le son B est plus aigu que le son A car la fréquence du son B est plus élevée que celle du son A.

\subsection*{Exercice 17 page 218}

\begin{enumerate}
\item Sur la courbe, on voit que le seuil d'audibilité atteint sa valeur minimale proche d'une fréquence de \unit{4000}{Hz}.
La sensibilité de l'oreille humaine est la plus grande pour des sons de fréquence proche de \unit{4000}{Hz}.
\item On voit sur la courbe pour un niveau d'intensité sonore de \unit{40}{dB}, le domaine des fréquences audibles se situe entre \unit{80}{Hz} et \unit{18000}{Hz} environ.
\item On lit sur la courbe qu'un son de fréquence \unit{40}{Hz} est audible à partir d'une niveau d'intensité sonore d'environ \unit{60}{dB}.
\item Le seuil d'audibilité dépend de la fréquence.
Par exemple, comme dit à la question 1, l'oreille humaine est très sensible à des sons de fréquence \unit{4000}{Hz}.
En revanche, pour qu'un son de fréquence \unit{30}{Hz} soit perçu, il faut que son niveau d'intensité sonore soit très élevé.
\item La hauteur d'un son n'est pas un facteur de risque.
En effet sur le graphique, on voit que le seuil de douleur est presque le même à toutes les fréquences.
\item Les ultrasons ont une fréquence supérieure à \unit{20}{kHz}.
Ils sont donc situés à droite du graphique.
\end{enumerate}

\subsection*{Exercice 18 page 218}

\begin{enumerate}
\item
\begin{multicols}{2}
\begin{itemize}
\item[•] Son A (diapason) : sur le graphique, \unit{20}{mm} correspond à \unit{5}{ms} et une période mesure \unit{9}{mm} :
\begin{center}
\begin{tabular}{c|c}
\unit{20}{mm} & \unit{5}{ms} \\
\hline
\unit{9}{mm} & $T$ 
\end{tabular}
\end{center}
\[
T_A = \frac{9\times 5}{20} = \unit{2{,}25}{ms}.
\]
\item[•] Son B (guitare) : sur le graphique, \unit{17}{mm} correspond à \unit{5}{ms} et une période mesure \unit{8}{mm} :
\begin{center}
\begin{tabular}{c|c}
\unit{17}{mm} & \unit{5}{ms} \\
\hline
\unit{8}{mm} & $T$ 
\end{tabular}
\end{center}
\[
T_B = \frac{8\times 5}{17} \approx \unit{2{,}25}{ms}.
\]
\end{itemize}
\end{multicols}
\item
\begin{multicols}{2}
\begin{itemize}
\item[•] Son A :
\[ T_A =  \unit{2{,}25}{ms} =  \unit{2{,}25\times10^{-3}}{s}. \]
\[  f_A = \frac{1}{T_A} = \frac{1}{2{,}25\times10^{-3}} \approx \unit{444}{Hz}. \]
La fréquence du son émis par le diapason est \unit{444}{Hz}.
\item[•] Son B :
\[ T_B \approx  \unit{2{,}25}{ms} =  \unit{2{,}25\times10^{-3}}{s}. \]
\[  f_B = \frac{1}{T_B} = \frac{1}{2{,}25\times10^{-3}} \approx \unit{444}{Hz}. \]
La fréquence du son émis par la guitare est \unit{444}{Hz}.
\end{itemize}
\end{multicols}
\item Les deux fréquences sont égales, la guitare est donc accordée.
\item L'amplitude du signal associé au son B est plus grande que celle du son A, le son B a donc un niveau d'intensité sonore plus grand que celui du son A.
\end{enumerate}

\subsection*{Exercice 20 page 219}

\begin{enumerate}
\item La méthode est la même que pour les exercices 6, 11 et 18.
On trouve $T_A = \unit{3{,}2}{ms}$ et $T_B \approx \unit{500}{\micro s}$.
\item 
\begin{multicols}{2}
\begin{itemize}
\item[•] Son A :
\[T_A = \unit{3{,}2\times10^{-3}}{s}\]
\[f_A = \frac{1}{T_A} = \frac{1}{3{,}2\times10^{-3}} \approx \unit{313}{Hz}.\]
\item[•] Son B :
\[T_B = \unit{500\times10^{-6}}{s}\]
\[f_B = \frac{1}{T_B} = \frac{1}{500\times10^{-6}} \approx \unit{2000}{Hz}.\]
\end{itemize}
\end{multicols}
$f_B > \unit{1000}{Hz}$ donc le son B n'est pas audible par le patient.
\end{enumerate}

\subsection*{Exercice 21 page 219}

\begin{enumerate}
\item
\begin{enumerate}
\item La période est divisée par deux quand la fréquence est doublée.
\item La période reste la même.
\end{enumerate}
\item cf. manuel page 315.
\end{enumerate}

\subsection*{Exercice 27 page 221}

\begin{enumerate}
\item
\begin{enumerate}
\item Il faut d'abord convertir la distance exprimée en toises en mètres :
\begin{center}
\begin{tabular}{c|c}
1 toise & \unit{1{,}95}{m} \\
\hline
14\,636 toises & $d$
\end{tabular}
\end{center}
\[ d = \frac{1{,}95\times14\,636}{1} \approx \unit{28\,540}{m}. \]
On utilise ensuite la formule de la vitesse (la vitesse de la lumière est traditionnellement notée $c$) :
\[ c = \frac{d}{\Delta t_\mathrm{lumi\grave{e}re}}, \]
d'où :
\[ \Delta t_\mathrm{lumi\grave{e}re} = \frac{d}{c} = \frac{28\,540}{3{,}00\times 10^8} \approx \unit{9{,}51\times10^{-5}}{s} \approx \unit{95{,}1}{\micro s}. \]
La durée mise par la lumière pour parcourir 14\,636 toises est $\Delta t_\mathrm{lumi\grave{e}re} \approx \unit{95{,}1}{\micro s}.$
\item La durée mise par le son pour parcourir 14\,636 toises (\unit{84{,}6}{s}) est beaucoup plus grande (environ $10^6=1\,000\,000$ de fois plus grande) que la durée mise par la lumière pour parcourir la même distance.
On peut donc négliger le temps de parcours de la lumière lors de la mesure ce qui justifie le \og Il suffit \fg{} du protocole.
\end{enumerate}
\item Les sources d'erreur portent sur la mesure du temps que mets le son à parvenir à l'observateur (temps de réaction, erreur instrumentale, etc.), sur la mesure de la distance entre l'observateur et le canon.
La présence d'obstacles peut aussi perturber la mesure (échos, etc.).

\item
\[ v_\mathrm{son} = \frac{d}{\Delta t_\mathrm{son}} = \frac{28\,540}{84{,}6} \approx \unit{337}{m/s}. \]
La vitesse de propagation du son dans l'air mesurée à l'époque est \unit{337}{m/s}, ce qui est cohérent avec les mesures plus récentes effectuée avec des moyens modernes.
\end{enumerate}


\end{document}