\include{template}
%\cfoot{} %% if no page number is needed

%\renewcommand\arraystretch{1.2}

\begin{document}

\begin{header}
TP

Un simple verre de sirop...
\end{header}

Un athlète de haut niveau surveille scrupuleusement son alimentation pour être toujours au top de sa forme.
Pour conserver sa bonne humeur, il s'accorde tout de même quelques petits plaisirs de temps à autre.
Son péché mignon : un verre de sirop de menthe.
Pour lui, le compromis entre plaisir gustatif et apport nutritionnel est atteint lorsque la concentration massique en sucre de sa boisson vaut précisément 67\,g/L.
Vous vous unissez pour préparer un grand verre à notre athlète !

\section*{Objectif}

\begin{objectif}
\textbf{Préparer 50\,mL de boisson au sirop de menthe pour l'athlète}
\end{objectif}

\section*{Contraintes}

Chaque élève rédige un compte-rendu qui fera partie du cours.
À la fin du TP, le professeur tire au sort un compte-rendu par groupe qui sera noté.

Le nettoyage de la verrerie et le rangement de la paillasse seront pris en compte dans la notation.

À la fin de la séance, vous amènerez votre boisson au professeur qui la validera.

\section*{Aides}

Vous pouvez demander de l'aide au professeur qui pourra vous donner des indications...

On redonne la relation :
\begin{equation}
C_\mathrm{m,mère} \times V_\mathrm{mère} = C_\mathrm{m,fille} \times V_\mathrm{fille}.
\nonumber
\end{equation}

Le sirop de menthe est présent au bureau, vous pouvez consulter l'étiquette et vous servir avec modération.

Une petite quantité de boisson correctement réalisée est présente au bureau.

Pour remplir l'objectif et rédiger un bon compte-rendu, vous devrez :
\begin{enumerate}
\item \textbf{Définir le problème}.
\begin{itemize}
\item[•] Identifier les informations utiles.
\item[•] Reformuler l'objectif en utilisant le vocabulaire scientifique.
\end{itemize}
\item \textbf{Rédiger le protocole.}
\begin{itemize}
\item[•] Écrire en quelques lignes ce que vous prévoyez de faire ;
\item[•] Effectuer le calcul nécessaire ;
\item[•] Établir une liste du matériel ;
\item[•] Faire un schéma de la manipulation ;
\item[•] Lister les étapes de manipulation.
\end{itemize}
\item \textbf{Réaliser la manipulation.}
\item \textbf{Valider le protocole.}
\begin{itemize}
\item[•] Proposer une méthode pour contrôler la boisson que vous avez réalisée.
\end{itemize}
\end{enumerate}

\section*{Notation}

\begin{center}
\begin{tabular}{l|l|c|c}
\textbf{Compétences} & Aptitudes à vérifier \hfill \textbf{Suis-je capable de ... ?} & \cmark & \xmark\\
\hline
\hline
S'approprier & Respecter les consignes données dans l'énoncé & & \\
\app         & Identifier les informations utiles & & \\
             & S'approprier une problématique & & \\
\hline
Analyser --    & Proposer un protocole & & \\
Raisonner        & Élaborer un protocole & & \\
\anarai              & Planifier une tache & & \\
\hline
Réaliser   & Effectuer un calcul & & \\
\rea    & Réaliser un schéma de la manipulation & & \\
      & Mettre en œuvre le protocole choisi & & \\
\hline
Valider      & Avoir un regard critique sur mes résultats & & \\
\val         & Dire si mes résultats sont en accord avec ceux attendus & & \\
\hline
Communiquer  & Rendre compte de façon écrite ou orale & & \\
\com         & Utiliser un vocabulaire adapté & & \\
\hline
Autonomie--  & Travailler en autonomie & & \\
Initiative   & Gérer mon temps & & \\
\auto        & & &
\end{tabular}
\end{center}

\end{document}