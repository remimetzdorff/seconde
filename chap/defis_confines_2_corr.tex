\documentclass[12pt,a4paper]{article}
\usepackage{rmpackages}																% usual packages
\usepackage{rmtemplate}																% graphic charter
%\usepackage{rmexocptce}																% for DS with cptce eval
\usepackage{pythontex}

%\cfoot{}	 													% if no page number is needed
\renewcommand\arraystretch{1.5}		% stretch table line height

\begin{document}

\begin{header}
Les défis confinés -- Épisode 2
\end{header}

\section*{Combien d'atomes de fer dans un clou ?}

\begin{enumerate}
\item Pour trouver efficacement le nombre de grains de semoule dans un paquet de \unit{500}{g}, il faut diviser la masse du paquet par la masse d'un seul grain de semoule, en, faisant attention aux unités.
\[
\unit{500}{g} = \unit{500\,000}{mg}
\]
donc le nombre de grains de semoule est : 
\[
\frac{500\,000}{1{,}25} = 400\,000.
\]
Il y a donc 400\,000 grains de semoule dans un paquet de \unit{500}{g}.

\item Ce résultat est proche de celui de la vidéo, on peut dire qu'il est en accord avec le comptage de la vidéo.
En effet l'écart entre les deux valeur est d'environ \unit{1}{\%}.

\item Le problème est le même que précédemment :
\[
\unit{4{,}2}{g} = \unit{0{,}0042}{kg}
\]
donc le nombre d'atomes de fer dans un clou est :
\[
\frac{0{,}0042}{9{,}3\times 10^{-26}} \approx 4{,}52\times 10^{22}.
\]

Il y a environ $4{,}52\times 10^{22}$ atomes de fer dans un clou.

\item Même si l'on pouvait les agrandir suffisamment, on ne pourrait pas les compter car il y en a vraiment beaucoup trop.
En les comptant un par un, au rythme d'un atome par seconde, il faudrait environ 1{,}4 million de milliard d'années.
C'est long !
\end{enumerate}

\section*{Construire un atome}


\begin{enumerate}
\item Le symbole de l'hélium est He.

\item Un tel atome de carbone est composé de 6 protons, 2 neutrons (ce qui fait 8 nucléons) et 6 électrons.

Remarque : il y a une erreur dans l'énoncé qui n'empêche cependant pas de répondre à la question.
L'atome de carbone décrit ci-dessus n'existe pas dans la nature.
L'atome de carbone le plus abondant sur Terre est en réalité le carbone 12 composé de 6 protons, 6 neutrons (ce qui fait 12 nucléons) et 6 électrons.

\item Le seul atome de béryllium stable est composé de 4 protons, 5 neutrons (donc 9 nucléons) et possède 4 électrons.

\item Le nombre de masse correspond au nombre de protons et de neutrons, c'est à dire au nombre de nucléons.

\item Il suffit d'ajouter un électron à un atome de fluor.
\end{enumerate}

\section*{Tracer une courbe avec python\texttrademark{}}



\begin{enumerate}
\item

\item On reconnait les données de concentration massique et de masse volumique.

\item La ligne 6 permet de tracer la courbe (et de choisir sa couleur, etc.).

\item

\item 

\item 

\item
\end{enumerate}


\end{document}