\documentclass[12pt,a4paper]{article}
\usepackage{rmpackages}																% usual packages
\usepackage{rmtemplate}																% graphic charter
\usepackage{rmexocptce}																% for DS with cptce eval

%\cfoot{} 													% if no page number is needed
%\renewcommand\arraystretch{1.5}		% stretch table line height

\begin{document}

\thispagestyle{empty}
\newgeometry{top=2cm,bottom=2cm, left=2cm, right=2cm}

\foreach \n in {1,...,3} {
\hrule

\section*{Période de révolution}

Pour un objet du système solaire (une planète, une comète, etc.), la \textbf{période de révolution}, correspond à la durée nécessaire à cet objet pour faire un tour complet \textbf{autour du Soleil.}

\begin{enumerate}
\setcounter{enumi}{11}
\item \rco{}

Quelle est la période de révolution de la Terre ?

\item \app{} \anarai{}

À l'aide des fichiers à votre disposition, déterminer la période de révolution de la comète Tchouri, puis celle de Jupiter.
Justifier en expliquant succinctement votre démarche.
\end{enumerate}

\setcounter{counterappelenv}{2}
\begin{appel}
\val{}
\end{appel}
}
\hrule

\end{document}