\include{template}
%\cfoot{} %% if no page number is needed

\begin{document}

\begin{header}
Chapitre 1 -- Corps purs et mélanges
\end{header}


\subsection*{Tests caractéristiques chimiques}

\begin{itemize}
\item[•] Test d'identification de l'eau : \href{https://tinyurl.com/thbmmcd}{https://tinyurl.com/thbmmcd} : en présence d'eau, le sulfate de cuivre devient bleu.
\item[•] Test d'identification du dioxygène : \href{https://tinyurl.com/y4y7elaf}{https://tinyurl.com/y4y7elaf} : en présence de dioxygène, un buchette incandescente se rallume.
\item[•] Test d'identification du dihydrogène : \href{https://tinyurl.com/y6a4kqw6}{https://tinyurl.com/y6a4kqw6} : quand on approche une flamme du dihydrogène, une détonation se produit.
\item[•] Test d'identification du dioxyde de carbone : \href{https://tinyurl.com/lef4pb5}{https://tinyurl.com/lef4pb5} : en présence de dioxyde de carbone, l'eau de chaux se trouble.
\end{itemize}

\begin{conseil}
Comme dans beaucoup de cas en chimie, une expérience se résume bien sous la forme d'un schéma.
En particulier, il est souvent efficace d'utiliser le schéma narratif en trois étapes :
\begin{itemize}
\item situation initiale ;
\item transition (flèche + phrase) ;
\item situation finale.
\end{itemize}
\end{conseil}

\begin{conseil}
Utiliser les vidéos et/ou le livre page 19 pour légender les schémas de l'activité 2.
Pour le test caractéristique de l'eau, faire le schéma en respectant le schéma narratif.
\end{conseil}

\end{document}