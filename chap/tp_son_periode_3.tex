\documentclass[12pt,a4paper]{article}
\usepackage{rmpackages}																% usual packages
\usepackage{rmtemplate}																% graphic charter
\usepackage{rmexocptce}																% for DS with cptce eval

%\cfoot{} 													% if no page number is needed
%\renewcommand\arraystretch{1.5}		% stretch table line height

\begin{document}

\begin{header}
TP

Son et musique
\end{header}

L'objectif de ce TP est de caractériser le son produit par plusieurs instruments : la voix, le diapason, la guitare, etc.

\begin{enumerate}
\item \app{}
\label{quest:diapason}

D'après les documents, quelle est la fréquence de la note émise par un diapason ?

\item \app{}

En vous aidant des documents, proposer une formule permettant de calculer la fréquence $f$ d'un signal périodique d'après la valeur de sa période $T$.

\emph{Aide : le chapitre 12 du livre (page 209--211) vous permettra de vérifier votre réponse.}

\item \rea{} \anarai{}

Avec un smartphone et en vous aidant du document~\ref{doc:phyphox}, réalisez l'acquisition de votre voix : 
\begin{itemize}
\item[•] en répétant rapidement \og La physique, c'est fantastique !\fg{} ;
\item[•] alors que vous bloquez sur le \og i \fg{} de physique : \og La physiiiiiiiiiiiiiiiiiiiiiiiiiiiiiiiiiiiiiiiiiii... \fg{}.
\end{itemize}
Lequel de ces sons est associé à un signal périodique ?
Justifier.

\item \rea{}

Reproduire l'allure d'une période sur votre compte-rendu.

\item \rea{}

Mesurer la période $T$ du signal périodique.
Comment réaliser la mesure la plus précise possible ?
\end{enumerate}

\vfill

\begin{center}
\includegraphics[scale=0.35]{images/jingle_bells.png}
\end{center}

\newpage

\begin{objectif}
Déterminer si un instrument est accordé.
\end{objectif}

La guitare est un instrument qui se désaccorde facilement si la température de la pièce varie trop fortement.
Les fenêtres de la salle de TP étant régulièrement ouvertes, il est probable que celle présente dans la salle soit complètement désaccordée.

Le diapason, lui est moins sensible à ces variations.

\begin{enumerate}
\item Choisir l'instrument que vous allez étudier pendant cette séance (diapason ou guitare).
Pour la guitare, choisissez une corde en particulier.
Indiquez votre choix sur le compte-rendu.

\item \app{} \anarai{} \rea{} \val{} \com{}

En respectant les étapes de l'aide à la rédaction du compte-rendu, vous déterminerez si l'instrument choisi est accordé ou non.

Appeler le professeur après avoir reformulé le problème, ou en cas de difficulté.
\appelprof{\app}

Appeler le professeur pour lui présenter votre protocole, ou en cas de difficulté.
\appelprof{\anarai}

Appeler le professeur pour lui faire part de vos conclusions.
\appelprof{}

\end{enumerate}

\section*{Aide à la rédaction du compte-rendu}

\begin{enumerate}
\item \textbf{Reformuler le problème} en utilisant le vocabulaire scientifique. \hfill \app{}

\item \textbf{Hypothèse}. Donnez votre hypothèse et justifiez-la : \og Je pense que ... car ... \fg{}. \hfill \anarai{}

\item \textbf{Protocole}. \hfill \app{} \anarai{} \rea{}

Mettre en place un protocole pour vérifier votre hypothèse. Il peut contenir :
\vspace{-0.5\baselineskip}
\begin{multicols}{2}
\begin{itemize}
\item[•] une expérience :
\begin{enumerate}
\item liste du matériel ;
\item schémas ;
\item observations et mesures ;
\end{enumerate}
\item[•] un calcul :
\begin{enumerate}
\item formule littérale ;
\item conversion ;
\item application numérique ;
\end{enumerate}
\end{itemize}
\end{multicols}
\vspace{-1.\baselineskip}
\begin{itemize}
\item[•] un raisonnement, une étude de documents, etc.
\end{itemize}
\item \textbf{Conclusion}. Pour terminer le compte-rendu : \hfill \val{}
\begin{itemize}
\item[•] donner les conclusions en reprenant ce qui a été trouvé dans le protocole ;
\item[•] dire si les conclusions sont en accord avec votre hypothèse ;
\item[•] répondre à la question posée !
\end{itemize}
\end{enumerate}

\newpage

%%%%%%%%%%%%%%%%%%%%%%%%%%%%%
% DOCUMENTS
%%%%%%%%%%%%%%%%%%%%%%%%%%%%%

\begin{doc}
\label{doc:phyphox}
\textbf{Acquisition d'un signal sonore avec l'application Phyphox}

Lancer l'application Phyphox (disponible sur Androïd \href{https://play.google.com/store/apps/details?id=de.rwth_aachen.phyphox&hl=fr&gl=US}{https://tinyurl.com/y7fpzd55} et iOS \href{https://apps.apple.com/fr/app/phyphox/id1127319693#?platform=iphone}{https://tinyurl.com/yd56x48k}), orienter le smartphone en mode paysage (à l'horizontale) pour plus de confort d'utilisation puis choisir l'expérience \emph{Mesure du son}.
L'écran du smartphone doit alors être similaire à l'image de gauche ci-dessous.

\begin{center}
\includegraphics[scale=0.2]{images/phyphox1.jpeg}
\includegraphics[scale=0.2]{images/phyphox2.jpeg}
\end{center}

\vspace{-2\baselineskip}
\begin{multicols}{2}
\textbf{Réaliser une acquisition}
\begin{enumerate}
\item Modifier la durée d'acquisition : appuyer sur le cadre situé sous le graphe, à côté de \emph{Durée} et rentrer la valeur voulue.
Choisir \unit{200}{ms} pour démarrer.
\item Appuyer sur le bouton \includegraphics[height=0.75\baselineskip]{images/phyphox_play.jpeg} pour démarrer l'acquisition.
\item Appuyer sur le bouton \includegraphics[height=0.75\baselineskip]{images/phyphox_pause.jpeg} pour arrêter l'acquisition.
\end{enumerate}
\newpage

\textbf{Faire une mesure}
\begin{enumerate}
\item Appuyer sur le graphe pour accéder à tous les menus.
L'écran du smartphone doit alors être similaire à l'image de droite ci-dessus.
\item Appuyer sur \emph{Déplacement et zoom} pour zoomer sur une partie de l'acquisition.
\item En appuyant sur \emph{Détail d'une mesure}, on peut mesurer des intervalles en faisant un \og toucher glisser \fg{}. 
\end{enumerate}
\end{multicols}
\end{doc}

\begin{doc}
\label{doc:periodic_signal}
\textbf{Signal sonore périodique}

Un signal périodique est un signal qui se reproduit à l'identique à intervalles de temps égaux :
\begin{itemize}
\item[•] la \textbf{période} $T$ correspond à la plus petite durée au bout de laquelle le signal se reproduit.
Elle s'exprime en seconde (s).
\item[•] la \textbf{fréquence} $f$ correspond au nombre de périodes du signal par seconde.
Elle s'exprime en hertz (Hz).
\end{itemize}

Le graphe ci-dessous représente un signal électrique périodique de période $T=\unit{2}{ms}$ et de fréquence $f=\unit{500}{Hz}$.

\begin{center}
\includegraphics[scale=1]{images/periodic_signal.png}
\end{center}
\end{doc}

\begin{doc}
\textbf{La gamme tempérée}

La musique occidentale est composée avec les notes de la gamme dite tempérée : do, ré, mi, fa, sol, la, si, do à nouveau et ainsi de suite.
Le tableau ci-dessous donne la \textbf{fréquence} de certaines de ces notes :
\begin{center}
\begin{tabular}{l|c|c|c|c|c|c|c|c|c}
\textbf{Note}						& \textbf{Do\textsubscript{1}} & \textbf{Mi\textsubscript{1}} & \textbf{La\textsubscript{1}} & \textbf{Ré\textsubscript{2}} & \textbf{Fa\textsubscript{2}} & \textbf{Sol\textsubscript{2}} & \textbf{Si\textsubscript{2}} & \textbf{Mi\textsubscript{3}} & \textbf{La\textsubscript{3}} \\
\hline
\textbf{Fréquence (Hz)} 	& 65{,}4 & 82{,}4 & 110 & 147 & 175 & 196 & 247 & 330 & 440 \\
\end{tabular}
\end{center}
L'indice situé après le nom de chaque note correspond à l'octave : il permet de différencier le mi grave (Mi\textsubscript{1}) du mi aigu d'une guitare (Mi\textsubscript{3}) par exemple.
\end{doc}

\begin{multicols}{2}
\begin{doc}
\label{doc:diapason}
\textbf{Le diapason}

Le diapason est instrument utilisé pour accorder d'autres instruments.
Une fois frappé, il émet une note unique : le La\textsubscript{3}.
\begin{center}
%\includegraphics[scale=1.07]{images/diapason.jpg}
\includegraphics[scale=1.07]{images/diapason.png}
\end{center}
\end{doc}

\begin{doc}
\textbf{La guitare}

Une guitare acoustique possède en général six cordes.
De la corde la plus grave à la plus aigüe, l'accordage standard est :

\begin{center}
\includegraphics[scale=1]{images/cordes_guitare.jpg}

Mi\textsubscript{1}, La\textsubscript{1}, Ré\textsubscript{2}, Sol\textsubscript{2}, Si\textsubscript{2}, Mi\textsubscript{3}.
\end{center}
\end{doc}
\end{multicols}

\end{document}

\begin{tabular}{l|c|c|c|c|c|c|c}
\textbf{Note}						& \textbf{Do\textsubscript{1}} & \textbf{Ré\textsubscript{1}} & \textbf{Mi\textsubscript{1}} & \textbf{Fa\textsubscript{1}} & \textbf{Sol\textsubscript{1}} & \textbf{La\textsubscript{1}} & \textbf{Si\textsubscript{1}} \\
\hline
\textbf{Fréquence (Hz)} 	& 65{,}4 & 73{,}4 & 82{,}4 & 87{,}3 & 98{,}0 & 110 & 123 \\
\hline \hline
\textbf{Note}						& \textbf{Do\textsubscript{2}} & \textbf{Ré\textsubscript{2}} & \textbf{Mi\textsubscript{2}} & \textbf{Fa\textsubscript{2}} & \textbf{Sol\textsubscript{2}} & \textbf{La\textsubscript{2}} & \textbf{Si\textsubscript{2}} \\
\hline
\textbf{Fréquence (Hz)} 	& 131 & 147 & 165 & 175 & 196 & 220 & 247 \\
\hline \hline
\textbf{Note}						& \textbf{Do\textsubscript{3}} & \textbf{Ré\textsubscript{3}} & \textbf{Mi\textsubscript{3}} & \textbf{Fa\textsubscript{3}} & \textbf{Sol\textsubscript{3}} & \textbf{La\textsubscript{3}} & \textbf{Si\textsubscript{3}} \\
\hline
\textbf{Fréquence (Hz)} 	& 262 & 294 & 330 & 350 & 392 & 440 & 494
\end{tabular}