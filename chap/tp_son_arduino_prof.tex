\documentclass[12pt,a4paper]{article}
\usepackage{rmpackages}																% usual packages
\usepackage{rmtemplate}																% graphic charter
\usepackage{rmexocptce}																% for DS with cptce eval

%\cfoot{} 													% if no page number is needed
%\renewcommand\arraystretch{1.5}		% stretch table line height

\begin{document}

\begin{header}
TP -- Fiche prof

Un diapason électronique
\end{header}

\section*{Contextualisation}

Lors du dernier TP, caractérisation de son avec Phyphox.

Rappel : le diapason est un instrument qui permet émet une seule note dont la fréquence est \unit{440}{Hz}. (écrit au tableau)

Aujourd'hui on va produire un son en utilisant une carte équipée d'un micro-contrôleur : Arduino.

Jouer Star Wars avec Arduino.

\section*{Consignes}

Questions 4 à 9 à rédiger.

C'est noté, un CR chacun, j'en ramasse un au sort par groupe en fin de séance.

\section*{Matériel}

En salle 604 :
\begin{itemize}
\item[•] cartes Arduino ;
\item[•] câbles USB ;
\item[•] jumpers ;
\item[•] résistances \unit{220}{\ohm} ;
\item[•] HP ou piezo.
\end{itemize}

Présenter rapidement les breadboards en montrant au tableau les pins connectées. 

\section*{Évaluation}

\begin{center}
\begin{tabular}{l|l|c}
\textbf{Compétence} & \textbf{Aptitude} / Observable & \textbf{Niveau} \\
\hline \hline
\rea   		& \textbf{Réaliser le montage correspondant à l'expérience proposée}	& \\
				& Le montage fonctionne															& A \\
				& 	Peu d'erreurs																				& B \\
				& Beaucoup d'erreurs																	& C \\
				& Élève démuni, il faut faire le montage pour lui						& D \\
\hline
\anarai	& \textbf{Proposer une méthode pour vérifier mon hypothèse} & \\
				& Protocole OK																				& A \\
				& Idée OK mais protocole confus 												& B \\
				& Comment pourrait-on observer l'influence de chaque argument ? & C \\
				& Modifier un paramètre puis l'autre, etc.									& D \\
\hline
\val		 	& \textbf{Avoir un regard critique sur mes résultats} 				& \\
				& Tout marche																				& A \\
				& 	Fréquence moitié																		& B \\
				& Fonction loop pas utilisée														& C \\
				& Ne fonctionne pas																	& D \\
\end{tabular}
\end{center}

\section*{Aides}

\begin{enumerate}
\setcounter{enumi}{4}

\item
\begin{itemize}
\item[•] Comment savoir sur quelle broche envoyer le signal périodique ?
\item[•] Quelle grandeur permet de caractériser un signal périodique ?
\item[•] Souvent avec Arduino, les durées sont exprimées en ms.
\end{itemize}

\item
\begin{itemize}
\item[•] Comment caractériser un son associé à un signal périodique ?
\item[•] Phyphox ?
\end{itemize}

\item
\begin{itemize}
\item[•] Expliquez ce que vous avez modifier et ce qui a changé.
\end{itemize}

\item
\begin{itemize}
\item[•] loop = boucle
\end{itemize}

\item
\begin{itemize}
\item[•] lit. delay = retard ou retardement
\item[•] À votre avis, c'est en quelle unité ?
Convertir en s.
\end{itemize}

\item
\begin{itemize}
\item[•] Tracer le repère tension pin 13 = f(temps)
\item[•] Tracer le signal pendant 5s.
\item[•] Mimer.
\end{itemize}

\item
\begin{itemize}
\item[•]
\end{itemize}

\item
\begin{itemize}
\item[•] Reformuler l'objectif avec un vocabulaire scientifique (signal, fréquence, son, etc...)
\item[•] Le haut-parleur est-il connecté à la bonne sortie ?
\item[•] Phyphox ?
\item[•] Il y a pas un facteur deux par hasard ?
\end{itemize}
\end{enumerate}

\paragraph{Questions subsidiaires:}
\begin{enumerate}[resume]
\item En utilisant la fonction \texttt{tone} du \texttt{programme1}, produire la suite de son suivant :

\item Comment améliorer la qualité du son produit par le montage ?
\begin{itemize}
\item[•] Dessiner l'allure du signal sortant de la broche 13.
\item[•] Et celle associée au diapason ?
\end{itemize}
\end{enumerate}

\end{document}