\documentclass[12pt,a5paper]{article}
\usepackage{rmpackages}																% usual packages
\usepackage{rmtemplate}																% graphic charter
\usepackage{rmexocptce}																% for DS with cptce eval

\cfoot{} 													% if no page number is needed
%\renewcommand\arraystretch{1.5}		% stretch table line height

\begin{document}

\begin{header}
CQFR -- Chapitre 9

Spectres d'émission
\end{header}

À la fin du chapitre, je suis capable de :
\begin{itemize}
\item[•] citer la \textbf{valeur de la vitesse de la lumière} ;

\item[•] obtenir le \textbf{spectre} d'un rayonnement (TP) ;

\item[•] exploiter un \textbf{spectre de raies} (TP, 17 page 236) ;

\item[•] exploiter un \textbf{spectre continu} (Activité, 9 page 234) ;

\item[•] caractériser le spectre du rayonnement émis par un \textbf{corps chaud} (Activité, 10 page 235) ;

\item[•] caractériser un un rayonnement monochromatique par sa \textbf{longueur d'onde} (20 page 236).
\end{itemize}

\end{document}