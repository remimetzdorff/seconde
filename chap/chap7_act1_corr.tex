\documentclass[12pt,a4paper,fleqn]{article}
\usepackage{rmpackages}																% usual packages
\usepackage{rmtemplate}																% graphic charter
\usepackage{rmexocptce}																% for DS with cptce eval

%\cfoot{} 													% if no page number is needed
%\renewcommand\arraystretch{1.5}		% stretch table line height

\begin{document}

\begin{header}
Correction de l'activité 1 page 66
\end{header}

\begin{enumerate}
\item
\begin{enumerate}
\item Cf. cours.
\item
Les atomes d'éléments appartenant à une même ligne ont la même dernière couche occupée.

Les atomes d'éléments appartenant à une même colonne ont le même nombre d'électrons de valence.
\end{enumerate}

\item La dernière couche occupée de cette configuration électronique est la troisième : cet élément appartient donc à la troisième ligne.
Il a quatre électrons de valence, il appartient donc à la quatorzième colonne. 
\textit{(Il y a dix colonnes entre la deuxième et la treizième pour les atomes de numéro atomique supérieur à 18.)}

\item
\begin{enumerate}
\item O ($Z=8$) : 1s\textsuperscript{2} 2s\textsuperscript{2} 2p\textsuperscript{4}

\item La dernière couche occupée d'un atome d'oxygène est la deuxième : il appartient donc à la deuxième ligne.
Il possède six électrons de valence donc il est dans la seizième colonne.
\end{enumerate}

\item \textbf{Pour savoir à quelle ligne appartient un élément, on regarde la dernière couche occupée.}

\textbf{Pour savoir à quelle colonne il appartient, on compte ses électrons de valence.}
\end{enumerate}

\end{document}