\documentclass[12pt,a4paper]{article}
\usepackage{rmpackages}																% usual packages
\usepackage{rmtemplate}																% graphic charter
\usepackage{rmexocptce}																% for DS with cptce eval

%\cfoot{} 													% if no page number is needed
%\renewcommand\arraystretch{1.5}		% stretch table line height

\newcommand{\jitem}{\refstepcounter{enumi}\item[\justify{} \theenumi .]}

\begin{document}

\begin{header}
Devoir à la maison 3
\end{header}

\begin{center}
\newcommand{\y}{\linewidth/8.01}
\newcommand{\z}{\y/2}
\begin{tikzpicture}
\foreach \i in {0,1, ..., 7} {
  \draw (\i*\y, 0) rectangle ++ (\y, \y);
}
\foreach \i in {0,1, ..., 7} {
  \draw (\i*\y, \y) rectangle ++ (\y, \y);
}
\draw (0, 2*\y) rectangle ++ (\y, \y);
\draw (7*\y, 2*\y) rectangle ++ (\y, \y);

\draw [line width=3pt] (2\y, 0) -- ++ (0, 2\y);

\draw (\z, 3*\y) node [below] {Hydrogène};
\draw (\z, 2.5*\y) node [] {\Large $\mathrm{H}$};
\draw (\z, 2*\y) node [below] {Lithium};
\draw (\z, 1.5*\y) node [] {\Large $\mathrm{Li}$};
\draw (\z, 1*\y) node [below] {Sodium};
\draw (\z, .5*\y) node [] {\Large $\mathrm{Na}$};

\draw (1.5\y, 2*\y) node [below] {Béryllium};
\draw (1.5\y, 1.5*\y) node [] {\Large $\mathrm{Be}$};
\draw (1.5\y, 1*\y) node [below] {Magnésium};
\draw (1.5\y, .5*\y) node [] {\Large $\mathrm{Mg}$};

\draw (2.5\y, 2*\y) node [below] {Bore};
\draw (2.5\y, 1.5*\y) node [] {\Large $\mathrm{B}$};
\draw (2.5\y, 1*\y) node [below] {Aluminium};
\draw (2.5\y, .5*\y) node [] {\Large $\mathrm{Al}$};

\draw (3.5\y, 2*\y) node [below] {Carbone};
\draw (3.5\y, 1.5*\y) node [] {\Large $\mathrm{C}$};
\draw (3.5\y, 1*\y) node [below] {Silicium};
\draw (3.5\y, .5*\y) node [] {\Large $\mathrm{Si}$};

\draw (4.5\y, 2*\y) node [below] {Azote};
\draw (4.5\y, 1.5*\y) node [] {\Large $\mathrm{N}$};
\draw (4.5\y, 1*\y) node [below] {Phosphore};
\draw (4.5\y, .5*\y) node [] {\Large $\mathrm{P}$};

\draw (5.5\y, 2*\y) node [below] {Oxygène};
\draw (5.5\y, 1.5*\y) node [] {\Large $\mathrm{O}$};
\draw (5.5\y, 1*\y) node [below] {Soufre};
\draw (5.5\y, .5*\y) node [] {\Large $\mathrm{S}$};

\draw (6.5\y, 2*\y) node [below] {Fluor};
\draw (6.5\y, 1.5*\y) node [] {\Large $\mathrm{F}$};
\draw (6.5\y, 1*\y) node [below] {Chlore};
\draw (6.5\y, .5*\y) node [] {\Large $\mathrm{Cl}$};

\draw (7.5\y, 3*\y) node [below, color=bleu_f] {Hélium};
\draw (7.5\y, 2.5*\y) node [color=bleu_f] {\Large $\mathrm{He}$};
\draw (7.5\y, 2*\y) node [below, color=bleu_f] {Néon};
\draw (7.5\y, 1.5*\y) node [color=bleu_f] {\Large $\mathrm{Ne}$};
\draw (7.5\y, 1*\y) node [below, color=bleu_f] {Argon};
\draw (7.5\y, .5*\y) node [color=bleu_f] {\Large $\mathrm{Ar}$};

\draw (.5\y, 3\y) node [above, color=red_f] {\large\textbf{1}};
\draw (1.5\y, 2\y) node [above, color=red_f] {\large\textbf{2}};
\draw (7.5\y, 3\y) node [above, color=red_f] {\large\textbf{18}};
\foreach \i in {13,14,...,17} {
  \draw (\i*\y-10.5\y, 2\y) node [above, color=red_f] {\large\textbf{\i}};
}
\end{tikzpicture}
\end{center}

\section*{Qui suis-je ?}

Pour chaque devinette, en vous aidant de la classification périodique ci-dessus, donner le nom et le symbole de l'élément dont il est question.
La réponse aux devinettes précédées du symbole \justify{} doit être justifiée.

\begin{enumerate}
\jitem Mon numéro atomique $Z$ est le 3.
Qui suis-je ?

\jitem J'appartiens à la deuxième ligne du tableau et je possède 5 électrons de valence.
Qui suis-je ?

\item J'appartiens à la 3\textsuperscript{ème} période et à la 13\textsuperscript{ème} famille.
Qui suis-je ?

\jitem Je suis le premier des gaz nobles.
Qui suis-je ?

\jitem Ma configuration électronique fondamentale est 1s\textsuperscript{2} 2s\textsuperscript{2} 2p\textsuperscript{6} 3s\textsuperscript{1}.
Qui suis-je ?

\jitem Je suis un atome.
En perdant deux électrons, j'obtiens la même configuration électronique que le néon.
Qui suis-je ?

\item Je forme l'ion $\mathrm{X^-}$ pour avoir la même configuration électronique que le néon.
Qui suis-je ?
\end{enumerate}

\section*{Le carbone}

Le carbone est le quatrième élément le plus abondant dans l'univers.
Il est l'un des éléments indispensables au vivant : c'est le composant essentiel des molécules organiques. 

\begin{enumerate}[resume]
\item À l'aide de la classification périodique ci-dessus, donner le numéro atomique $Z$ du carbone.

\item Indiquer, en le justifiant, le nombre d'électrons d'un atome de carbone.

\item Donner la configuration électronique fondamentale de cet atome.

\item Justifier sa place dans le tableau périodique.

\item Le charbon (composé en majorité de carbone) brûle dans l'oxygène pour former du dioxyde de carbone $\dioxydedecarbone$ dont le schéma de Lewis est représenté ci-dessous :
\begin{center}
\chemfig[atom sep=20pt]{\lewis{35,O}=C=\lewis{17,O}}
\end{center}
Justifier la stabilité de chaque atome de cette molécule.

\item Le propanoate d'éthyle représenté ci-dessous est un arôme utilisé pour son odeur de fruit rouge.

\begin{center}
\chemfig[angle increment=90, atom sep=20pt]{\textcolor{bleu_f}{C}([-1]-H)([2]-H)([1]-H)-C([-1]-H)([1]-H)-C([1]=O)-\textcolor{orange_f}{O}-C([-1]-H)([1]-H)-C([-1]-H)([1]-H)-\textcolor{green_f}{H}}
\end{center}

Recopier le schéma de Lewis incomplet de cette molécule et le compléter.

\item En utilisant le schéma de Lewis \textbf{complet}, justifier la stabilité des atomes colorés.

\end{enumerate}

\section*{L'ion lithium}

\begin{enumerate}[resume]
\item Donner, en la justifiant, la formule chimique de l'ion lithium, seul ion stable formé à partir d'un atome de lithium.
\end{enumerate}

\end{document}