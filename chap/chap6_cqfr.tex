\documentclass[12pt,a5paper]{article}
\usepackage{rmpackages}																% usual packages
\usepackage{rmtemplate}																% graphic charter
\usepackage{rmexocptce}																% for DS with cptce eval

\cfoot{} 													% if no page number is needed
%\renewcommand\arraystretch{1.5}		% stretch table line height

\begin{document}

\begin{header}
CQFR -- Chapitre 6

Décrire un mouvement
\end{header}

À la fin du chapitre, je suis capable de :
\begin{itemize}
\item[•] \textbf{choisir un référentiel adapté} (4 page 160) ;

\item[•] dire que \textbf{le mouvement d'un système dépend du référentiel} (9 page 161) ;

\item[•] calculer une \textbf{vitesse moyenne} (22 page 163) ;

\item[•] donner les \textbf{caractéristiques du vecteur vitesse} (13 page 161) ;

\item[•] \textbf{représenter le vecteur vitesse} en un point de la trajectoire d'un système (saut de grenouille) ;

\item[•] caractériser un mouvement d'après une chronophotographie ou la représentation de vecteurs vitesse (14 page 161, 31 page 165).
\end{itemize}

\end{document}