\documentclass[12pt,a4paper]{article}
\usepackage{rmpackages}																% usual packages
\usepackage{rmtemplate}																% graphic charter
\usepackage{rmexocptce}																% for DS with cptce eval

\cfoot{}	 													% if no page number is needed
%\renewcommand\arraystretch{1.5}		% stretch table line height

\begin{document}

\begin{header}
Les défis confinés -- Épisode 1
\end{header}

\section*{Devinettes}

\begin{enumerate}
\item Trouve l'intrus dans la liste ci-dessous !
\begin{multicols}{4}
\begin{itemize}
\item[•] \unit{80}{kg}
\item[•] \unit{15}{mL}
\item[•] \unit{29{,}7}{cm}
\item[•] \unit{1}{g/L}
\item[•] \unit{0{,}5}{mm}
%\item[•] \unit{1}{mL}
%\item[•] \unit{20}{g}
%\item[•] \unit{1{,}75}{m}
\item[•] $3{,}1415$
\item[•] \unit{1}{kg/L}
\item[•] \unit{20}{m\cubed}
%\item[•] \unit{19{,}3}{g/cm^3}
\end{itemize}
\end{multicols}

\item Pour chacune des valeurs ci-dessus, indique s'il s'agit d'une longueur, d'un volume, d'une masse ou d'une masse volumique.

\item Pour chacune des valeurs ci-dessus, propose un objet qui possède cette propriété.

\textit{Par exemple, une cuillère à soupe a un volume d'environ \unit{15}{mL}.}
\end{enumerate}
\textbf{Et la physique-chimie dans tout ça ?}
N'oublie pas les unités : la valeur d'une mesure ou un résultat n'a aucun sens s'il n'est pas suivie de son unité !

\section*{Fabrique un cube}

\begin{enumerate}
\item Avec les matériaux à ta disposition (papier, carton, bois, métal, tissus, mousse ou quoi que ce soit), réalise deux cubes :
\begin{itemize}
\item[•] le premier de \unit{1}{dm} de côté ;
\item[•] le deuxième, plus petit, de seulement \unit{1}{cm} de côté.
\end{itemize}
Tu peux aussi trouver deux objets en forme de cube qui ont les bonnes dimensions.

Prends en photo tes réalisations ou tes trouvailles à côté d'une règle graduée (pour que l'on puisse se rendre compte de la taille), tu pourras les montrer en classe.

\item Calcule le volume de chaque cube et exprime le résultat 
\begin{itemize}
\item[•] en $\mathrm{dm^3}$ et en $\mathrm{L}$ pour le premier ;
\item[•] en $\mathrm{cm^3}$ et en $\mathrm{mL}$ pour le deuxième.
\end{itemize}

\item Relie chaque contenant au bon volume :
\begin{center}
\begin{tabular}{rc p{0.15\textwidth} c p{0.3\textwidth}}
cuillère à café & • && • & \unit{0{,}3}{mL} \\
baignoire & • && • & $\unit{3\,000}{m\cubed} = \unit{3\,000\,000}{L}$ \\
bouteille d'eau & • && • & \unit{5}{mL} \\
piscine olympique & • && • & \unit{100}{L} \\
réserve d'encre d'un stylo bille & • && • & \unit{1{,}5}{L} 
\end{tabular}
\end{center}
\end{enumerate}
\textbf{Et la physique-chimie dans tout ça ?}
Pour avoir un œil critique sur tes hypothèses ou tes réponses, tu pourras comparer le résultat obtenu avec celui que tu attends, qui te semble raisonnable compte tenu de la situation maintenant que tu sais exactement le volume que représente \unit{1}{L}, \unit{1}{mL}, etc. 



\end{document}