\documentclass[12pt,a4paper]{article}
\usepackage{rmpackages}																% usual packages
\usepackage{rmtemplate}																% graphic charter
\usepackage{rmexocptce}																% for DS with cptce eval

%\cfoot{} 													% if no page number is needed
%\renewcommand\arraystretch{1.5}		% stretch table line height

\begin{document}

\thispagestyle{empty}
\newgeometry{top=1cm,bottom=1cm}

\section*{Au clair de la lune}

Compléter le \texttt{programme4} pour jouer \og Au clair de la lune \fg{} avec l'Arduino.

\paragraph{Partition}

\begin{center}
\begin{tabular}{|ccccccccccc|}
\hline
Do 		& Do 		& Do 		& Ré 		& Mi 		& Ré 		& Do 		& Mi 		& Ré 		& Ré 		& Do \\
\cdot	& \cdot	& \cdot	& \cdot	& --			& --			& \cdot	& \cdot	& \cdot	& \cdot	& -- \\
\hline
\end{tabular}
\end{center}

\paragraph{Fréquence de quelques notes}

\begin{center}
\begin{tabular}{l|c|c|c|c|c|c|c}
\textbf{Note}						& \textbf{Do} & \textbf{Ré} & \textbf{Mi} & \textbf{Fa} & \textbf{Sol} & \textbf{La} & \textbf{Si} \\
\hline
\textbf{Fréquence (Hz)} 	& 131 & 147 & 165 & 175 & 196 & 220 & 247 \\
\end{tabular}
\end{center}

\hrule{}

\section*{Au clair de la lune}

Compléter le \texttt{programme4} pour jouer \og Au clair de la lune \fg{} avec l'Arduino.

\paragraph{Partition}

\begin{center}
\begin{tabular}{|ccccccccccc|}
\hline
Do 		& Do 		& Do 		& Ré 		& Mi 		& Ré 		& Do 		& Mi 		& Ré 		& Ré 		& Do \\
\cdot	& \cdot	& \cdot	& \cdot	& --			& --			& \cdot	& \cdot	& \cdot	& \cdot	& -- \\
\hline
\end{tabular}
\end{center}

\paragraph{Fréquence de quelques notes}

\begin{center}
\begin{tabular}{l|c|c|c|c|c|c|c}
\textbf{Note}						& \textbf{Do} & \textbf{Ré} & \textbf{Mi} & \textbf{Fa} & \textbf{Sol} & \textbf{La} & \textbf{Si} \\
\hline
\textbf{Fréquence (Hz)} 	& 131 & 147 & 165 & 175 & 196 & 220 & 247 \\
\end{tabular}
\end{center}

\hrule{}

\section*{Au clair de la lune}

Compléter le \texttt{programme4} pour jouer \og Au clair de la lune \fg{} avec l'Arduino.

\paragraph{Partition}

\begin{center}
\begin{tabular}{|ccccccccccc|}
\hline
Do 		& Do 		& Do 		& Ré 		& Mi 		& Ré 		& Do 		& Mi 		& Ré 		& Ré 		& Do \\
\cdot	& \cdot	& \cdot	& \cdot	& --			& --			& \cdot	& \cdot	& \cdot	& \cdot	& -- \\
\hline
\end{tabular}
\end{center}

\paragraph{Fréquence de quelques notes}

\begin{center}
\begin{tabular}{l|c|c|c|c|c|c|c}
\textbf{Note}						& \textbf{Do} & \textbf{Ré} & \textbf{Mi} & \textbf{Fa} & \textbf{Sol} & \textbf{La} & \textbf{Si} \\
\hline
\textbf{Fréquence (Hz)} 	& 131 & 147 & 165 & 175 & 196 & 220 & 247 \\
\end{tabular}
\end{center}

\hrule{}

\section*{Au clair de la lune}

Compléter le \texttt{programme4} pour jouer \og Au clair de la lune \fg{} avec l'Arduino.

\paragraph{Partition}

\begin{center}
\begin{tabular}{|ccccccccccc|}
\hline
Do 		& Do 		& Do 		& Ré 		& Mi 		& Ré 		& Do 		& Mi 		& Ré 		& Ré 		& Do \\
\cdot	& \cdot	& \cdot	& \cdot	& --			& --			& \cdot	& \cdot	& \cdot	& \cdot	& -- \\
\hline
\end{tabular}
\end{center}

\paragraph{Fréquence de quelques notes}

\begin{center}
\begin{tabular}{l|c|c|c|c|c|c|c}
\textbf{Note}						& \textbf{Do} & \textbf{Ré} & \textbf{Mi} & \textbf{Fa} & \textbf{Sol} & \textbf{La} & \textbf{Si} \\
\hline
\textbf{Fréquence (Hz)} 	& 131 & 147 & 165 & 175 & 196 & 220 & 247 \\
\end{tabular}
\end{center}

\section*{Au clair de la lune}

Compléter le \texttt{programme4} pour jouer \og Au clair de la lune \fg{} avec l'Arduino.

\paragraph{Partition}

\begin{center}
\begin{tabular}{|ccccccccccc|}
\hline
Do 		& Do 		& Do 		& Ré 		& Mi 		& Ré 		& Do 		& Mi 		& Ré 		& Ré 		& Do \\
\cdot	& \cdot	& \cdot	& \cdot	& --			& --			& \cdot	& \cdot	& \cdot	& \cdot	& -- \\
\hline
\end{tabular}
\end{center}

\thispagestyle{empty}

\paragraph{Fréquence de quelques notes}

\begin{center}
\begin{tabular}{l|c|c|c|c|c|c|c}
\textbf{Note}						& \textbf{Do} & \textbf{Ré} & \textbf{Mi} & \textbf{Fa} & \textbf{Sol} & \textbf{La} & \textbf{Si} \\
\hline
\textbf{Fréquence (Hz)} 	& 131 & 147 & 165 & 175 & 196 & 220 & 247 \\
\end{tabular}
\end{center}

\hrule{}

\section*{Au clair de la lune}

Compléter le \texttt{programme4} pour jouer \og Au clair de la lune \fg{} avec l'Arduino.

\paragraph{Partition}

\begin{center}
\begin{tabular}{|ccccccccccc|}
\hline
Do 		& Do 		& Do 		& Ré 		& Mi 		& Ré 		& Do 		& Mi 		& Ré 		& Ré 		& Do \\
\cdot	& \cdot	& \cdot	& \cdot	& --			& --			& \cdot	& \cdot	& \cdot	& \cdot	& -- \\
\hline
\end{tabular}
\end{center}

\paragraph{Fréquence de quelques notes}

\begin{center}
\begin{tabular}{l|c|c|c|c|c|c|c}
\textbf{Note}						& \textbf{Do} & \textbf{Ré} & \textbf{Mi} & \textbf{Fa} & \textbf{Sol} & \textbf{La} & \textbf{Si} \\
\hline
\textbf{Fréquence (Hz)} 	& 131 & 147 & 165 & 175 & 196 & 220 & 247 \\
\end{tabular}
\end{center}

\hrule{}

\section*{Au clair de la lune}

Compléter le \texttt{programme4} pour jouer \og Au clair de la lune \fg{} avec l'Arduino.

\paragraph{Partition}

\begin{center}
\begin{tabular}{|ccccccccccc|}
\hline
Do 		& Do 		& Do 		& Ré 		& Mi 		& Ré 		& Do 		& Mi 		& Ré 		& Ré 		& Do \\
\cdot	& \cdot	& \cdot	& \cdot	& --			& --			& \cdot	& \cdot	& \cdot	& \cdot	& -- \\
\hline
\end{tabular}
\end{center}

\paragraph{Fréquence de quelques notes}

\begin{center}
\begin{tabular}{l|c|c|c|c|c|c|c}
\textbf{Note}						& \textbf{Do} & \textbf{Ré} & \textbf{Mi} & \textbf{Fa} & \textbf{Sol} & \textbf{La} & \textbf{Si} \\
\hline
\textbf{Fréquence (Hz)} 	& 131 & 147 & 165 & 175 & 196 & 220 & 247 \\
\end{tabular}
\end{center}

\hrule{}

\section*{Au clair de la lune}

Compléter le \texttt{programme4} pour jouer \og Au clair de la lune \fg{} avec l'Arduino.

\paragraph{Partition}

\begin{center}
\begin{tabular}{|ccccccccccc|}
\hline
Do 		& Do 		& Do 		& Ré 		& Mi 		& Ré 		& Do 		& Mi 		& Ré 		& Ré 		& Do \\
\cdot	& \cdot	& \cdot	& \cdot	& --			& --			& \cdot	& \cdot	& \cdot	& \cdot	& -- \\
\hline
\end{tabular}
\end{center}

\paragraph{Fréquence de quelques notes}

\begin{center}
\begin{tabular}{l|c|c|c|c|c|c|c}
\textbf{Note}						& \textbf{Do} & \textbf{Ré} & \textbf{Mi} & \textbf{Fa} & \textbf{Sol} & \textbf{La} & \textbf{Si} \\
\hline
\textbf{Fréquence (Hz)} 	& 131 & 147 & 165 & 175 & 196 & 220 & 247 \\
\end{tabular}
\end{center}

\section*{Devinez}

\begin{center}
\begin{tabular}{|cccccccc|}
\hline
Do 		& Ré 		& Fa 		& La 		& Sol		& La 		& Sol		& Fa 		\\
\cdot	& \cdot	& --			& --			& \cdot	& \cdot	& \cdot	& --	\\
\hline
\end{tabular}
\begin{tabular}{|cccccccc|}
\hline
Fa 		& Sol 		& Fa 		& Sol 		& Fa		& Sol 		& Fa		& Ré 		\\
\cdot	& \cdot	& \cdot	& \cdot	& \cdot	& \cdot	& \cdot	& --	\\
\hline
\end{tabular}
\end{center}

\thispagestyle{empty}

\hrule{}

\section*{Devinez}

\begin{center}
\begin{tabular}{|cccccccc|}
\hline
Do 		& Ré 		& Fa 		& La 		& Sol		& La 		& Sol		& Fa 		\\
\cdot	& \cdot	& --			& --			& \cdot	& \cdot	& \cdot	& --	\\
\hline
\end{tabular}
\begin{tabular}{|cccccccc|}
\hline
Fa 		& Sol 		& Fa 		& Sol 		& Fa		& Sol 		& Fa		& Ré 		\\
\cdot	& \cdot	& \cdot	& \cdot	& \cdot	& \cdot	& \cdot	& --	\\
\hline
\end{tabular}
\end{center}

\hrule{}

\section*{Devinez}

\begin{center}
\begin{tabular}{|cccccccc|}
\hline
Do 		& Ré 		& Fa 		& La 		& Sol		& La 		& Sol		& Fa 		\\
\cdot	& \cdot	& --			& --			& \cdot	& \cdot	& \cdot	& --	\\
\hline
\end{tabular}
\begin{tabular}{|cccccccc|}
\hline
Fa 		& Sol 		& Fa 		& Sol 		& Fa		& Sol 		& Fa		& Ré 		\\
\cdot	& \cdot	& \cdot	& \cdot	& \cdot	& \cdot	& \cdot	& --	\\
\hline
\end{tabular}
\end{center}

\hrule{}

\section*{Devinez}

\begin{center}
\begin{tabular}{|cccccccc|}
\hline
Do 		& Ré 		& Fa 		& La 		& Sol		& La 		& Sol		& Fa 		\\
\cdot	& \cdot	& --			& --			& \cdot	& \cdot	& \cdot	& --	\\
\hline
\end{tabular}
\begin{tabular}{|cccccccc|}
\hline
Fa 		& Sol 		& Fa 		& Sol 		& Fa		& Sol 		& Fa		& Ré 		\\
\cdot	& \cdot	& \cdot	& \cdot	& \cdot	& \cdot	& \cdot	& --	\\
\hline
\end{tabular}
\end{center}

\hrule{}

\section*{Devinez}

\begin{center}
\begin{tabular}{|cccccccc|}
\hline
Do 		& Ré 		& Fa 		& La 		& Sol		& La 		& Sol		& Fa 		\\
\cdot	& \cdot	& --			& --			& \cdot	& \cdot	& \cdot	& --	\\
\hline
\end{tabular}
\begin{tabular}{|cccccccc|}
\hline
Fa 		& Sol 		& Fa 		& Sol 		& Fa		& Sol 		& Fa		& Ré 		\\
\cdot	& \cdot	& \cdot	& \cdot	& \cdot	& \cdot	& \cdot	& --	\\
\hline
\end{tabular}
\end{center}

\hrule{}

\section*{Devinez}

\begin{center}
\begin{tabular}{|cccccccc|}
\hline
Do 		& Ré 		& Fa 		& La 		& Sol		& La 		& Sol		& Fa 		\\
\cdot	& \cdot	& --			& --			& \cdot	& \cdot	& \cdot	& --	\\
\hline
\end{tabular}
\begin{tabular}{|cccccccc|}
\hline
Fa 		& Sol 		& Fa 		& Sol 		& Fa		& Sol 		& Fa		& Ré 		\\
\cdot	& \cdot	& \cdot	& \cdot	& \cdot	& \cdot	& \cdot	& --	\\
\hline
\end{tabular}
\end{center}

\hrule{}

\section*{Devinez}

\begin{center}
\begin{tabular}{|cccccccc|}
\hline
Do 		& Ré 		& Fa 		& La 		& Sol		& La 		& Sol		& Fa 		\\
\cdot	& \cdot	& --			& --			& \cdot	& \cdot	& \cdot	& --	\\
\hline
\end{tabular}
\begin{tabular}{|cccccccc|}
\hline
Fa 		& Sol 		& Fa 		& Sol 		& Fa		& Sol 		& Fa		& Ré 		\\
\cdot	& \cdot	& \cdot	& \cdot	& \cdot	& \cdot	& \cdot	& --	\\
\hline
\end{tabular}
\end{center}

\hrule{}

\section*{Devinez}

\begin{center}
\begin{tabular}{|cccccccc|}
\hline
Do 		& Ré 		& Fa 		& La 		& Sol		& La 		& Sol		& Fa 		\\
\cdot	& \cdot	& --			& --			& \cdot	& \cdot	& \cdot	& --	\\
\hline
\end{tabular}
\begin{tabular}{|cccccccc|}
\hline
Fa 		& Sol 		& Fa 		& Sol 		& Fa		& Sol 		& Fa		& Ré 		\\
\cdot	& \cdot	& \cdot	& \cdot	& \cdot	& \cdot	& \cdot	& --	\\
\hline
\end{tabular}
\end{center}

\hrule{}

\section*{Devinez}

\begin{center}
\begin{tabular}{|cccccccc|}
\hline
Do 		& Ré 		& Fa 		& La 		& Sol		& La 		& Sol		& Fa 		\\
\cdot	& \cdot	& --			& --			& \cdot	& \cdot	& \cdot	& --	\\
\hline
\end{tabular}
\begin{tabular}{|cccccccc|}
\hline
Fa 		& Sol 		& Fa 		& Sol 		& Fa		& Sol 		& Fa		& Ré 		\\
\cdot	& \cdot	& \cdot	& \cdot	& \cdot	& \cdot	& \cdot	& --	\\
\hline
\end{tabular}
\end{center}

\hrule{}


\end{document}