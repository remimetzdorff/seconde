\documentclass[12pt,a4paper,fleqn]{article}
\usepackage{rmpackages}																% usual packages
\usepackage{rmtemplate}																% graphic charter
\usepackage{rmexocptce}																% for DS with cptce eval

%\cfoot{} 													% if no page number is needed
%\renewcommand\arraystretch{1.5}		% stretch table line height

\begin{document}

\begin{header}
Correction de l'activité 4 page 69
\end{header}

\begin{enumerate}
\item
Les électrons de valence sont les électrons de la dernière couche occupée :
\begin{itemize}
\item[•] l'hélium a 2 électrons de valence ;
\item[•] le néon et l'argon ont 8 électrons de valence.
\end{itemize}

\item
\begin{multicols}{3}
\begin{itemize}
\item[•]
Le méthane :
\begin{center}
\chemfig[angle increment=90, atom sep=20pt]{H-C([-1]-H)([1]-H)-H}
\end{center}
\item[•]
L'eau :
\begin{center}
\chemfig[angle increment=90, atom sep=20pt]{H-\lewis{26,O}-H}
\end{center}
\item[•]
Le sulfure d'hydrogène :
\begin{center}
\chemfig[angle increment=90, atom sep=20pt]{H-\lewis{26,S}-H}
\end{center}
\item[•]
L'ammoniac :
\begin{center}
\chemfig[angle increment=90, atom sep=20pt]{H-\lewis{6,N}([1]-H)-H}
\end{center}
\item[•]
Le dioxyde de carbone :
\begin{center}
\chemfig[angle increment=90, atom sep=20pt]{\lewis{35, O}=C=\lewis{17,O}}
\end{center}
\item[•]
Le dioxygène :
\begin{center}
\chemfig[angle increment=90, atom sep=20pt]{\lewis{35, O}=\lewis{17,O}}
\end{center}
\end{itemize}
\end{multicols}
Dans toutes ces molécules :
\begin{itemize}
\item[•] chaque atome d'hydrogène est entouré de 2 électrons de valence ce qui leur permet d'avoir une configuration électronique similaire à celle de l'hélium ;
\item[•] les autres atomes (carbone, azote, oxygène et soufre) sont chacun entourés de 8 électrons de valence ce qui leur permet d'avoir une configuration électronique similaire à celle du néon.
\end{itemize}

\item Sur le schéma du livre il manque les doublets non-liants.
Le schéma de Lewis du glucose est :
\begin{center}
\chemfig[angle increment=90, atom sep=20pt]{H-C([1]=\lewis{13,O})-C([-1]-H)([1]-\lewis{04,O}-H)-C([-1]-H)([1]-\lewis{04,O}-H)-C([-1]-H)([1]-\lewis{04,O}-H)-C([-1]-H)([1]-\lewis{04,O}-H)-C([-1]-H)([1]-\lewis{04,O}-H)-H}
\end{center}

\item
Dans une molécule, les atomes peuvent mettre en commun des électrons pour former des \textbf{liaisons covalentes}.
Ils obtiennent ainsi la \textbf{configuration électronique du gaz noble le plus proche} ce qui les stabilise.
\end{enumerate}

\end{document}