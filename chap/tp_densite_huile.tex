\include{template}
\cfoot{} %% if no page number is needed

\begin{document}

\begin{header}
TP

Qui de l'eau ou de l'huile est la plus dense ?
\end{header}

\section*{Objectif}

Répondre à la question !

\section*{Aide}

Le sujet du TP sera souvent formulé comme une question assez ouverte.
Pour répondre à la question et rédiger un bon compte-rendu, vous devez respecter les principales étapes de la \textbf{démarche scientifique} :
\begin{enumerate}
\item \textbf{Hypothèse}.
Donnez votre hypothèse et justifiez-la : \og Je pense que ... car ... \fg{}.
\item \textbf{Protocole}.
Mettre en place un protocole pour valider (ou invalider !) votre hypothèse :
\begin{itemize}
\item[•] écrire en quelques lignes ce que vous prévoyez de faire ;
\item[•] établir une liste du matériel ;
\item[•] faire un schéma de l'expérience ;
\item[•] réaliser l'expérience : décrire les étapes de la manipulation ;
\item[•] relever les mesures utiles : définir des notations indiquer les unités ;
\item[•] indiquer les observations utiles : schéma et description (\og J'observe que ... \fg{}).
\end{itemize}
\item \textbf{Conclusion}. Pour terminer le compte-rendu :
\begin{itemize}
\item[•] donner les conclusions en reprenant ce qui a été trouvé dans le protocole ;
\item[•] dire si les conclusions sont en accord avec votre hypothèse ;
\item[•] répondre à la question posée !
\end{itemize}
\end{enumerate}

\section*{Contraintes}

Un compte-rendu à rédiger par élève.
\`A la fin, le professeur tire au sort un compte rendu par groupe qui sera noté.

\section*{Données}

\danger
Toutes les données ne sont pas forcément utiles pour ce TP !
\danger

Le sulfate de cuivre devient bleu en présence d'eau.

La masse volumique est donnée par $\rho = \dfrac{m}{V}$.

La densité d'un liquide est donnée par $d = \dfrac{\rho}{\rho_\mathrm{eau}}$.

La masse volumique de l'air est $\rho_\mathrm{air} \approx \unit{1{,}2}{\kilo\gram/\meter^3}$.

\end{document}

\section*{Grille de compétences / aptitudes à valider}

\begin{center}
\begin{tabular}{|l|l|c|c|}
\hline
\textbf{Compétences} & \textbf{Suis-je capable de ... ?} & Auto- & Eval. \\
                     &                                   & eval. & prof. \\
                     &                                   & \cmark ou \xmark & \cmark ou \xmark \\
\hline
\hline
S'approprier & Respecter les consignes données dans l'énoncé & & \\
\app         & Me servir correctement des ressources disponibles (doc, énoncé, ...) & & \\
\hline
\end{tabular}
\end{center}

\end{document}