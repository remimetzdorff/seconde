\documentclass[12pt,a4paper,fleqn]{article}
\usepackage{rmpackages}																% usual packages
\usepackage{rmtemplate}																% graphic charter
\usepackage{rmexocptce}																% for DS with cptce eval

%\cfoot{} 													% if no page number is needed
%\renewcommand\arraystretch{1.5}		% stretch table line height

\newcommand{\ritem}{\refstepcounter{enumi}\item[$\star$ \theenumi .]}

\begin{document}

\begin{header}
Activité -- Compter des entités chimiques Correction
\end{header}

\section*{Quelle est la masse d'une molécule ?}

\begin{enumerate}
\item La masse $m_\mathrm{O_2}$ d'une molécule de dioxygène est :
\[
m_\mathrm{O_2} = 2 \times m_\mathrm{O} = 2 \times 2\dcoma66 \times 10^{-26} = \unit{5\dcoma32 \time 10^{-26}}{kg}.
\]

\item La masse $m_\methane$ d'une molécule de méthane est :
\[
m_\methane = m_\mathrm{C} + 4 \times m_\mathrm{H} = 1\dcoma99 \times 10^{-26} + 4 \times 1\dcoma67 \times 10^{-27} \approx \unit{2\dcoma66 \time 10^{-26}}{kg}.
\]

\item La masse $m_\mathrm{benz}$ d'une molécule de benzaldéhyde est :
\begin{align*}
m_\mathrm{benz}	&= m_\mathrm{O} + 7 \times m_\mathrm{C} + 6 \times m_\mathrm{H}\\
								&= 2\dcoma66 \times 10^{-26} + 7 \times 1\dcoma99 \times 10^{-26} + 6 \times 1\dcoma67 \times 10^{-27} \\
								&\approx \unit{1\dcoma76 \time 10^{-25}}{kg}.
\end{align*}

\end{enumerate}

\section*{Compter des particules}

\begin{enumerate}[resume]
\item Le nombre de balles de tennis dans le seau est $\frac{3600}{57} \approx 63$.

On trouve trois balles de plus car l'énoncé indique que \unit{3600}{g} est la masse du seau rempli de balles, soit la masse du seau et celle des balles.
Pour obtenir le vrai nombre de balles, il faudrait retirer la masse du seau (qui n'est pas connue ici).

\item Le nombre $N$ d'atomes d'aluminium contenus dans la règle est :
\[
N = \frac{m_\mathrm{règle}}{m_\mathrm{Al}} = \frac{0\dcoma040}{4\dcoma48\times10^{-26}} \approx 8\dcoma9 \times 10^{23}.
\]

\item La formule qui donne le nombre $N$ d'entités de masse $m_\mathrm{entité}$ dans un échantillon de masse $m$ est :
\[
N = \frac{m}{m_\mathrm{entité}}.
\]
On pouvait trouver cette formule, soit en utilisant les calculs précédents, soit en regardant page 88 du livre.
\end{enumerate}

\section*{Quantité de matière}

\begin{enumerate}[resume]
\item Cf. vidéo \href{https://youtu.be/\_kosdfe79OU}{https://youtu.be/\_kosdfe79OU}.
\end{enumerate}

\section*{Autant de molécules d'eau dans un dé à coudre que d'étoiles dans l'Univers ?}

\begin{enumerate}[resume]
\item La formule chimique de l'eau est $\eau$.

\item La masse $m_\mathrm{\eau}$ d'une molécule d'eau est :
\[
m_{\eau} = m_\mathrm{O} + 2 \times m_\mathrm{H} = 2\dcoma66 \times 10^{-26} + 2 \times 1\dcoma67 \times 10^{-27}  \approx \unit{3\dcoma00 \time 10^{-26}}{kg}.
\]

\item Le nombre $N$ de molécules d'eau contenues dans le dé à coudre est :
\[
N = \frac{m}{m_{\eau}} = \frac{3 \times 10^{-3}}{3\dcoma00 \time 10^{-26}} \approx 1 \times 10^{23}.
\]
\warning{} Les deux masses doivent être exprimées avec la même unité.
On a choisi les kilogrammes : $m = \unit{3}{g} = \unit{3 \times 10^{-3}}{kg}$.

\item Dans la vidéo, Cyrus North indique qu'il y a environ $\frac{6\dcoma02 \times 10^{23}}{6} \approx 1 \times 10^{23}$ grains de sable sur Terre, soit le même nombre que le nombre de molécules d'eau dans le dé à coudre.
En effet :
\[
\frac{\text{nombre de molécules d'eau dans le dé à coudre}}{\text{nombre de grains de sable sur Terre}} = \frac{1 \times 10^{23}}{1 \times 10^{23}} = 1.
\]
\textbf{Il y a autant de molécules d'eau dans le dé que de grains de sable sur Terre !}

\item D'après les chiffres de l'énoncé, il y a environ $250 \times 10^9 \times 400 \times 10^9 = 1\dcoma00 \times 10^{23}$ étoiles dans l'Univers.

\textbf{Il y a autant de molécules d'eau dans le dé à coudre que d'étoiles dans l'Univers !}

Remarque : il y a aussi autant de grains de sable sur Terre que d'étoiles dans l'Univers !

\item En utilisant la formule de la page 88 :
\[
n = \frac{N}{N_A} = \frac{1 \times 10^{23}}{6\dcoma02 \times 10^{23}} \approx \unit{0\dcoma2}{mol}.
\]
Le dé à coudre contient \unit{0\dcoma2}{mol} d'eau.

\end{enumerate}

\end{document}

\section*{Conversation fictive entre Avogadro et Ampère}