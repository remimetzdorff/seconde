\documentclass[12pt,a4paper]{article}
\usepackage{rmpackages}																% usual packages
\usepackage{rmtemplate}																% graphic charter
\usepackage{rmexocptce}																% for DS with cptce eval

%\cfoot{} 													% if no page number is needed
%\renewcommand\arraystretch{1.5}		% stretch table line height

\begin{document}

\begin{header}
TP

Quel médicament choisir ?
\end{header}

Après une journée de huit heures de cours, vous en avez plein la tête et un mal de crâne s'installe.
Une tisane ?
Non.
Pour mettre fin au calvaire, vous choisissez le comprimé de Doliprane\textsuperscript{\textregistered}
1000 mg.
Mais...
Malheur !
L'armoire à pharmacie est vide et il est trop tard pour vous rendre en pharmacie...

En désespoir de cause vous vous tournez vers trois connaissances,  Alice, Bob et Charlie, commerciaux dans l'industrie pharmaceutique
Ravis d'avoir un potentiel client, ils accourent et vous proposent chacun un échantillon de leur médicament.
Une querelle commence où chacun accuse les autres de vouloir vous vendre un médicament inefficace ou pire, dangereux.

Contre les maux de tête, seul un de leurs médicaments est en réalité vraiment efficace.
Un autre est issu d'une méthode de synthèse douteuse et peut nuire gravement à votre santé et le dernier est une contrefaçon qui soulagera simplement vos aigreurs d'estomac.

\section*{Objectif}
\begin{objectif}
Identifier chacun des trois \og médicaments \fg{}
\end{objectif}

\section*{Contraintes}

Vous devrez mettre en œuvre deux méthodes d'identification basées sur des principes différents.

Chaque élève rédige un compte-rendu qui fera partie du cours.
À la fin du TP, le professeur tire au sort un compte-rendu par groupe qui sera noté.

\section*{Aides}

Vous pouvez demander de l'aide au professeur qui pourra vous donner des indications...

Les trois \og médicaments \fg{} sont présents au bureau dans les coupelles A, B et C (pour Alice, Bob et Charlie).
Vous pouvez bien sûr en prendre des échantillons afin de réaliser les tests nécessaires.

Pour remplir l'objectif et rédiger un bon compte-rendu, vous devez respecter les principales étapes de la \textbf{démarche scientifique} :
\begin{enumerate}
\item \textbf{Hypothèse}.
Donnez votre hypothèse et justifiez-la : \og Je pense que ... car ... \fg{}.
\item \textbf{Protocole}.
Mettre en place un protocole pour valider (ou invalider !) votre hypothèse :
\begin{itemize}
\item[•] écrire en quelques lignes ce que vous prévoyez de faire ;
\item[•] établir une liste du matériel ;
\item[•] si besoin, faire un schéma de l'expérience ;
\item[•] réaliser l'expérience : décrire les étapes de la manipulation ;
\item[•] relever les mesures utiles : définir des notations, indiquer les unités ;
\item[•] indiquer les observations utiles : schéma et description (\og J'observe que ... \fg{}).
\end{itemize}
\item \textbf{Conclusion}. Pour terminer le compte-rendu :
\begin{itemize}
\item[•] donner les conclusions en reprenant ce qui a été trouvé dans le protocole ;
\item[•] dire si les conclusions sont en accord avec votre hypothèse ;
\item[•] répondre à la problématique !
\end{itemize}
\end{enumerate}

\begin{doc}
\textbf{1 : Synthèse du paracétamol}

Le paracétamol fut synthétisé pour la première fois en 1878 par Harmon Northrop Morse.
Plus tard, Friedlander modifia la synthèse ce qui permit d'améliorer sensiblement son rendement.
Le paracétamol est obtenu en faisant réagir deux espèces chimiques dont le 4-aminophénol.
Un sous-produit de la synthèse est l'acide acétique qui donne aussi sont acidité au vinaigre.
Après synthèse, le paracétamol est purifié pour éviter la présence de résidus d'autres espèces.
\end{doc}

\begin{multicols}{2}

\begin{doc}
\textbf{2 : Doliprane\textsuperscript{\textregistered} 1000 mg}

Ce médicament est un antalgique et un antipyrétique qui contient du paracétamol.
Il est utilisé pour faire baisser la fièvre et dans le traitement des affections douloureuses.

Comme tout médicament, son utilisation peut entrainer des effets indésirables.

\flushright
\textit{Source : \href{https://eurekasante.vidal.fr/medicaments/vidal-famille/medicament-ddolip01-DOLIPRANE.html}{eurekasante.vidal.fr}}
\end{doc}

\begin{doc}
\textbf{3 : Bicarbonate de soude}

Il facilite la digestion ou procure un soulagement aux maux d'estomac dus aux acidités gastriques en tant qu'antiacide.
Il est efficace contre les reflux gastriques dus au stress ou à un excès alimentaire.

Avec un acide faible, il forme un mélange effervescent.
\end{doc}

\end{multicols}

\begin{doc}
\textbf{4 : Propriétés physico-chimiques de quelques espèces}
\center
\begin{tabular}{l|c|c|c|c}
Nom & Formule brute & $T_\mathrm{fus}$ ($\degree \mathrm{C}$) & $\rho$ ($\mathrm{g/cm^3}$) & Aspect à \unit{20}{\celsius}  \\
\hline \hline
Paracétamol & $\mathrm{C_8H_9NO_2}$ & 170 & 1{,}293 & solide blanc \\
4-aminophénol & $\mathrm{C_6H_7NO}$ & ??? & 1{,}293 & solide blanc \\
Bicarbonate de soude & $\mathrm{NaHCO_3}$ & ??? & ??? & solide blanc \\
\end{tabular}
\end{doc}

\begin{doc}
\textbf{5 : Mesurer une température de fusion avec le banc Kofler}

Le banc Kofler est une plaque chauffante sur laquelle s'établit une élévation régulière de température.
La mesure de la  température de fusion d'un solide avec le banc Kofler est rapide et simple à mettre en œuvre.
Cette mesure est un critère de pureté très répandu au laboratoire.
En effet, un produit pur présente un point de fusion bien net : la transition solide-liquide a lieu sur un intervalle de moins de un degré.
En revanche, un produit impur présente une transition moins nette, à une température toujours différente de la température de fusion du produit pur.

\flushright
\textit{Source : \href{http://culturesciences.chimie.ens.fr/content/utilisation-du-banc-kofler-pour-mesurer-une-temperature-fusion-918}{culturesciences.chimie.ens.fr}}
\end{doc}

\begin{multicols}{2}

\begin{doc}
\textbf{6 : 4-aminophénol}

\begin{center}
\includegraphics[scale=0.05]{images/pict_aquatique.jpg}
\includegraphics[scale=0.05]{images/pict_nocif.jpg}
\includegraphics[scale=0.05]{images/pict_cancer.jpg}
\end{center}

\footnotesize
\noindent
H341 : Susceptible d'induire des anomalies génétiques.

\noindent
H302+H332 : Nocif en cas d’ingestion ou d’inhalation.

\noindent
H410 : Très toxique pour les organismes aquatiques.
\end{doc}

\begin{doc}
\textbf{7 : Paracétamol}

\begin{center}
\includegraphics[scale=0.05]{images/pict_nocif.jpg}
\end{center}

\footnotesize
\noindent
H302 : Nocif en cas d'ingestion.

\noindent
H315 : Provoque une irritation cutanée.

\noindent
H317 : Peut provoquer une allergie cutanée.

\noindent
H319 : Provoque une sévère irritation des yeux.
\end{doc}

\end{multicols}

\end{document}