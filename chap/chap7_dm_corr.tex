\documentclass[12pt,a4paper,fleqn]{article}
\usepackage{rmpackages}																% usual packages
\usepackage{rmtemplate}																% graphic charter
\usepackage{rmexocptce}																% for DS with cptce eval

%\cfoot{} 													% if no page number is needed
%\renewcommand\arraystretch{1.5}		% stretch table line height

\begin{document}

\begin{header}
Devoir à la maison 3 -- Correction
\end{header}

\section*{Qui suis-je ?}

\begin{enumerate}
\item Dans le tableau périodique, les éléments sont organisé par ordre de numéro atomique croissant.
Le troisième élément et le lithium ($\rm{Li}$).

\item Le nombre d'électrons de valence donne la colonne (la famille) de l'élément : avec 5 électrons de valence, cet élément appartient à la 15 colonne.
Il s'agit donc de l'azote ($\rm{N}$).

\item Aluminium ($\rm{Al}$)

\item Les gaz nobles sont les éléments de la dernière famille : le premier d'entre eux est l'hélium ($\rm{He}$).

\item La dernière couche occupée est la troisième : cet élément appartient à la troisième période.
Il possède 1 électron de valence : il appartient à la première famille.
Il s'agit donc du sodium ($\rm{Na}$).

\item \textbf{Méthode 1 :} En perdant 2 électrons, cet élément en a autant que le néon.
L'atome a donc 2 électrons de plus que le néon : il s'agit du magnésium ($\rm{Mg}$).

\textbf{Méthode 2 :} La configuration électronique du néon est : 1s\textsuperscript{2} 2s\textsuperscript{2} 2p\textsuperscript{6}.
En rajoutant 2 électrons, on obtient la configuration de l'élément recherché : 1s\textsuperscript{2} 2s\textsuperscript{2} 2p\textsuperscript{6} 3s\textsuperscript{2}.
Cet élément appartient à la troisième période et à la deuxième famille, il s'agit bien du magnésium ($\rm{Mg}$).

\item L'ion a une charge négative : il a gagné un électron pour obtenir la configuration électronique du néon.
Il s'agit donc du fluor ($\rm{F}$).
\end{enumerate}

\section*{Le carbone}

\begin{enumerate}[resume]
\item Le numéro atomique du carbone est $Z=6$.

\item Le carbone est situé dans la 14\textsuperscript{ème} famille : il possède 4 électrons de valence.

\item La configuration électronique du carbone est 1s\textsuperscript{2} 2s\textsuperscript{2} 2p\textsuperscript{2}.

\item La dernière couche occupée est la deuxième : il est donc dans la deuxième ligne.
Il possède bien 4 électrons de valence : il est donc bien dans la 14\textsuperscript{ème} colonne.

\item Chaque tiret représente un doublet (une paire d'électrons).
On vérifie que chaque atome de cette molécule est entouré de 8 électrons :
\begin{itemize}
\item[•] le carbone est entouré de 4 doublets liants (deux doubles liaisons), soit 8 électrons ;
\item[•] chaque atome d'oxygène est entouré de 2 doublets liant (une double liaison) et de 2 doublets non-liants, soit 8 électrons.
\end{itemize}
En effet, le carbone et l'oxygène sont proches du néon : pour lui ressembler, il doivent s'entourer de 8 électrons de valence.

\item Il manque les 2 doublets non-liants autour de chaque atome d'oxygène :
\begin{center}
\chemfig[angle increment=90, atom sep=20pt]{\textcolor{bleu_f}{C}([-1]-H)([2]-H)([1]-H)-C([-1]-H)([1]-H)-C([1]=\lewis{13,O})-\textcolor{orange_f}{\lewis{26,O}}-C([-1]-H)([1]-H)-C([-1]-H)([1]-H)-\textcolor{green_f}{H}}
\end{center}

\item Pour avoir autant d'électrons de valence que l'hélium qui en a 2, les atomes d'hydrogène cherchent à s'entourer d'un doublet : dans cette molécule, chaque atome d'hydrogène est bien entouré d'un doublet liant : ils forment tous une simple liaison avec un autre atome.

De la même façon que pour le dioxyde de carbone, les atomes de carbone et d'oxygène s'entourent de 8 électrons : dans cette molécule, ils sont entourés de 4 doublets.
\end{enumerate}

\section*{L'ion lithium}

\begin{enumerate}[resume]
\item Li (Z=3) : 1s\textsuperscript{2} 2s\textsuperscript{1}

En perdant 1 électron, il obtient la configuration 1s\textsuperscript{2} qui est celle de l'hélium, gaz noble le plus proche du lithium.

Après avoir perdu un électron, le lithium forme l'ion lithium chargé positivement : en effet il possède 3 protons (chargés +) et seulement 2 électrons (chargés -).

La formule de l'ion lithium est donc $\rm{Li^+}$.

\end{enumerate}

\end{document}