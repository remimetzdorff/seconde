\documentclass[12pt,a4paper]{article}
\usepackage{rmpackages}																% usual packages
\usepackage{rmtemplate}																% graphic charter
\usepackage{rmexocptce}																% for DS with cptce eval

%\cfoot{} 													% if no page number is needed
%\renewcommand\arraystretch{1.5}		% stretch table line height

\begin{document}

\begin{header}
Physique - Chimie 2\textsuperscript{nde}1

Table des matières
\end{header}

N'imprime pas cette feuille, elle évoluera tout au long de l'année.
Si tu le souhaites, tu peux t'en servir pour commencer à écrire la table des matières de ton classeur.

\section*{Chapitre 1 -- Corps purs et mélanges}

\begin{enumerate}
\item CQFR -- Chapitre 1 Corps purs et mélanges
\item Rappels de collège
\item Activité 1 -- Corps purs et mélanges
\item Activité 2 -- Identification d'espèces
\item Activité 3 -- Composition des mélanges
\item Cours
\item TP -- Qui de l'eau ou de l'huile est la plus dense ?
\item Interrogation du chapitre 1
\item Devoir surveillé 1
\item Exercices du chapitre 1
\end{enumerate}

\section*{Chapitre 2 -- Solutions aqueuses}

\begin{enumerate}
\item CQFR -- Chapitre 2 Solutions aqueuses
\item Activité 4 -- Solutions aqueuses
\item Cours
\item Fiche technique Dissolution - Dilution
\item TP Dissolution -- Sauvez Maurice !
\item TP -- Un simple verre de sirop...
\item TP -- Python et les solutions aqueuses
\item Interrogation du chapitre 2
\item Devoir à la maison 1
\item Exercices du chapitre 2
\end{enumerate}

\section*{Chapitre 3 -- Du macroscopique au microscopique}

\begin{enumerate}
\item CQFR -- Chapitre 3 Du macroscopique au microscopique
\item Cours
\item TP -- Mesurer une molécule d'huile
\item Exercices du chapitre 3 (ou quiz quizinière suivant les groupes)
\end{enumerate}

\section*{Chapitre 4 -- Le noyau de l'atome}

\begin{enumerate}
\item CQFR -- Chapitre 4 Le noyau de l'atome
\item Cours
\item Interrogation sur le chapitre 4
\item Exercices du chapitre 4
\end{enumerate}

\section*{Chapitre 5 -- Le son}

\begin{enumerate}
\item CQFR -- Chapitre 5 Le son
\item Cours
\item TP -- Son et musique
\item TP -- Un diapason électronique
\item Interrogation du chapitre 5
\item Devoir à la maison 2
\item Exercices du chapitre 5
\end{enumerate}

\section*{Chapitre 6 -- Décrire un mouvement}

\begin{enumerate}
\item CQFR -- Chapitre 6 Décrire un mouvement
\item Cours
\item TP -- La comète / C'était une petite planète
{\color{gray_c}\item TP -- Plume et marteau}
{\color{gray_c}\item Interrogation du chapitre 6}
\item Exercices du chapitre 6
\end{enumerate}

\section*{Chapitre 7 -- Vers des entités plus stables}

\begin{enumerate}
{\color{gray_c}\item CQFR -- Chapitre 7 Vers des entités plus stables}
{\color{gray_c}\item Cours}
\item Exercices du chapitre 7
\end{enumerate}

\section*{Outils mathématiques}

\begin{enumerate}
\item Le produit en croix
\item Conversions (1)
\item Conversions (2)
\item Les puissances de 10
\end{enumerate}

\section*{Fiche info -- Les compétences à acquérir en seconde}

\end{document}