\documentclass[12pt,a5paper]{article}
\usepackage{rmpackages}																% usual packages
\usepackage{rmtemplate}																% graphic charter
%\usepackage{rmexocptce}																% for DS with cptce eval

\cfoot{} 													% if no page number is needed
%\renewcommand\arraystretch{1.5}		% stretch table line height

\begin{document}

\begin{header}
CQFR -- Chapitre 4

L'atome
\end{header}

À la fin du chapitre, je suis capable de :
\begin{itemize}
\item[•] établir l'écriture conventionnelle d'un noyau à partir de sa composition et inversement (7 p 60) ;
\item[•] donner l'ordre de grandeur de la taille d'un atome (cours) ;
\item[•] calculer la masse d'un atome à partir de ses constituants (9 p 60) ;
\item[•] déterminer la composition d'un atome à partir de sa masse (21 p 60) ;
\item[•] exploiter la neutralité de l'atome (2 p 60) ;
\item[•] dire comment se forment les ions monoatomiques et déterminer leur composition à partir de celle d'un atome (11 p 60) ;
\item comparer deux valeurs ayant la même unité (8 p 60).
\end{itemize}

\end{document}