\documentclass[12pt,a4paper]{article}
\usepackage{rmpackages}																% usual packages
\usepackage{rmtemplate}																% graphic charter
\usepackage{rmexocptce}																% for DS with cptce eval

%\cfoot{} 													% if no page number is needed
%\renewcommand\arraystretch{1.5}		% stretch table line height

\begin{document}

\begin{header}
Chapitre 4 -- Exercices
\end{header}

\section*{Corrections}

\subsection*{Exercice 2 page 60}

Un atome est toujours électriquement neutre, il a donc autant de charges positives que de charges négatives, c'est-à-dire autant de protons que d'électrons.
L'atome de carbone possède 6 protons, donc il a également 6 électrons.

\subsection*{Exercice 3 page 60}

Un atome est toujours électriquement neutre, il a donc autant d'électrons que de protons : le seul modèle qui représente un atome est donc le a.

Le modèle b est un anion et le modèle c est un cation.

\subsection*{Exercice 6 page 60}

\begin{enumerate}
\item Le nombre 14 correspond au nombre de protons, aussi appelé numéro atomique.

Le nombre 28 correspond au nombre de nucléons, aussi appelé nombre de masse. C'est la somme du nombre de protons et du nombre de neutrons.

\item Ce noyau contient $14$ protons et $28-14=14$ neutrons.
\end{enumerate}

\subsection*{Exercice 7 page 60}

\begin{center}
\begin{tabular}{|l|c|c|c|c|}
\hline
\textbf{Symbole de l'élément}	& C & N & Cl & Fe \\
\hline
\textbf{Nombre de protons}		& 6 & 7 & \textcolor{bleu_f}{17} & 26 \\
\hline
\textbf{Nombre de neutrons}	& \textcolor{bleu_f}{8} & 8 & 18 & \textcolor{bleu_f}{30} \\
\hline
\textbf{Écriture conventionnelle du noyau} & $^{14}_{\color{bleu_f} {6}}\text{C}$ & $\color{bleu_f} ^{15}_{7}\text{N}$ & $^{\textcolor{bleu_f}{35}}_{17}\text{Cl}$ & $^{56}_{\color{bleu_f} {26}}\text{Fe}$ \\
\hline
\end{tabular}
\end{center}

\subsection*{Exercice 8 page 60}

\begin{enumerate}
\item $ r_\mathrm{atome} = \unit{53}{pm} = \unit{53\times 10^{-12}}{m} = \unit{5{,}3\times 10^{-11}}{m}$.
\item
\[
\frac{r_\mathrm{atome}}{r_\mathrm{noyau}} = \frac{5{,}3\times 10^{-11}}{1{,}5\times 10^{-15}} \approx 35\,000.
\]
L'atome d'hydrogène est environ 35\,000 fois plus grand que son noyau.
\end{enumerate}

\subsection*{Exercice 9 page 60}

\paragraph{Masse précise} En utilisant les valeurs du cours.
\begin{align*}
m_\mathrm{précise}	&= \mathrm{nb\ de\ protons} \times m_\mathrm{proton} + \mathrm{nb\ de\ neutron} \times m_\mathrm{neutrons} + \mathrm{nb\ d'électrons} \times m_\mathrm{électron} \\
											&= 79 \times 1{,}673\times 10^{-27} + 121 \times 1{,}675\times 10^{-27} + 79 \times 9{,}1\times 10^{-31} \\
											&\approx \unit{3{,}349\times10^{-25}}{kg}.
\end{align*}

\paragraph{Masse approchée} On considère que les protons et les neutrons ont la même masse (elle sont très proches) et que la masse des électrons est négligeable, puisqu'elle est environ 2\,000 fois plus faible que la masse des nucléons. 
\begin{align*}
m_\mathrm{approchée}	&= \mathrm{nombre\ de\ nucléons} \times m_\mathrm{nucléon} \\
											&= (\mathrm{nombre\ de\ protons} + \mathrm{nombre\ de\ neutrons}) \times m_\mathrm{nucléon} \\
											&= (79+121)\times 1{,}67\times 10^{-27} \\
											&\approx \unit{3{,}34\times10^{-25}}{kg}.
\end{align*}

\paragraph{Remarque} La différence entre ces deux valeurs est effectivement très faible.

\subsection*{Exercice 10 page 61}

\begin{enumerate}
\item 
\[
A = \frac{m}{m_{nucléon}}.
\]
\item 
\[
A = \frac{2{,}00\times 10^{-26}}{1{,}67\times 10^{-27}} \approx 12{,}0.
\]
Cet atome de carbone a 12 nucléons.
\end{enumerate}

\subsection*{Exercice 10 page 61}

\begin{enumerate}
\item On connait la masse de l'atome de sélénium, on peut calculer le nombre de nucléons $A$ du noyau de cet atome (cf. exercice 10 page 61) :
\[
A = \frac{m}{m_{nucléon}} =  \frac{1{,}32\times 10^{-25}}{1{,}67\times 10^{-27}} \approx 79{,}0.
\]
Puisque cet atome possède 45 neutrons, il possède $79-45 = 34$ protons.
\item L'écriture conventionnelle du noyau de cet atome est $ ^{79}_{35}\text{Se} $.
\item L'ordre de grandeur du noyau de cet atome est \unit{10^{-10}}{m}.
\end{enumerate}

\subsection*{Exercice 11 page 61}

\begin{enumerate}
\item En perdant deux électrons de charge négative, l'atome de magnésium devient un ion positif, un cation, chargé deux fois positivement : $\ionmagnesiumII$.

\item L'atome de magnésium possède 12 protons, donc 12 électrons car un atome est toujours électriquement neutre.
L'ion magnésium se forme quand un atome de magnésium perd deux électrons : il possède donc 10 électrons et toujours 12 protons.
\end{enumerate}

\subsection*{Exercice 12 page 61}

\begin{enumerate}
\item Cet ion possède plus d'électrons que de protons, c'est-à-dire plus de charges négatives que de charges positives.
Il est donc négatif : c'est un anion.

\item La formule de cet ion est $\text{X}^\text{2-}$.
\end{enumerate}

\subsection*{Exercice 27 page 63}

Il existe de multiples méthodes pour résoudre ce problème en suivant les étapes suggérées page 296.
En voici une.

\paragraph{Étape 1}
\begin{enumerate}
\item \label{quest:atomediamant} Qu'est-ce qu'un \og atome de diamant \fg{} ?
\item \label{quest:carat} Combien pèse \unit{1{,}1}{carat} ?
\item \label{quest:nbatomediamant} Combien y a-t-il d'\og atomes de diamant \fg{} dans \unit{1{,}1}{carat} ?
\end{enumerate}

\paragraph{Étape 2}
\begin{enumerate}
\item Le diamant n'est pas un élément chimique.
L'énoncé nous indique que le diamant est constitué uniquement d'atomes de carbone 12.
Un \og atome de diamant \fg{} est donc en réalité un atome de carbone 12.
\item L'énoncé nous indique que \unit{5}{carats} de diamant a une masse de \unit{1{,}0}{g}.
\begin{center}
\begin{tabular}{c|c}
\unit{5{,}0}{carats} & \unit{1{,}0}{g} \\
\hline
\unit{1{,}1}{carat} & ?
\end{tabular}
\end{center}
En faisant un produit en croix, on trouve que \unit{1{,}1}{carat} ont une masse de $\frac{1{,}1\times 1{,}0}{5{,}0} = \unit{0{,}22}{g}$ soit \unit{0{,}000\,22}{kg}.
\item Pour répondre à cette question il est nécessaire de calculer la masse d'un seul atome de carbone 12.
La masse approchée $m_\mathrm{C}$ d'un atome de carbone 12 est :
\[
m_\mathrm{C} \approx A \times m_\mathrm{nucléon} = 12 \times 1{,}67\times 10^{-27} \approx \unit{2{,}00\times10^{-26}}{kg}
\]
En utilisant le résultat de la question précédente, on peut calculer le nombre d'atomes de carbone 12 dans \unit{1{,}1}{carat} (comme dans les défis confinés 2) :
\[
\frac{0{,}000\,22}{2{,}00\times10^{-26}} \approx 1{,}1 \times 10^{22}
\]
Il y a $1{,}1 \times 10^{22}$ atomes de carbone 12 dans \unit{1{,}1}{carat} de diamant.
\end{enumerate}

\paragraph{Étape 3}
Quel est le prix d'un atome de carbone 12 contenu dans un diamant ?

\paragraph{Étape 4}
D'après les réponses précédentes et l'énoncé, $1{,}1 \times 10^{22}$ atomes de carbone 12 (contenus dans du diamant) coûtent 15\,000 €. 
\begin{center}
\begin{tabular}{c|c}
$1{,}1 \times 10^{22}$ atomes & 15\,000 €\\
\hline
1 atome & ?
\end{tabular}
\end{center}
En faisant à nouveau un produit en croix, on trouve qu'un atome coûte $\frac{1\times 15\,000}{1{,}1 \times 10^{22}} \approx 1{,}4 \times 10^{-18}$ €.

\paragraph{Étape 5}
La question posée dans l'énoncé revient à se demander le prix d'un atome de carbone 12 contenu dans un diamant.
Un \og atome de diamant \fg{} coûte donc $1{,}4 \times 10^{-18}$ €.
À ce prix là tout le monde peut s'acheter un atome de diamant !
Seul problème, on ne risque pas d'en faire un gros bijou...

\paragraph{Remarque} Le graphite est lui aussi composé d'atomes de carbone 12, mais arrangés différemment par rapport à ceux du diamant.
Cette organisation à l'échelle microscopique est responsable de grandes différences entre leurs propriétés physiques : le diamant est dur, transparent et isolant électrique alors que le graphite est mou, opaque et conducteur électrique.
Le coût des deux matériaux est aussi très différent : le prix du diamant est environ dix million de fois plus élevé que celui du graphite.
\end{document}