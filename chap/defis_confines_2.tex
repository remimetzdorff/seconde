\documentclass[12pt,a4paper]{article}
\usepackage{rmpackages}																% usual packages
\usepackage{rmtemplate}																% graphic charter
%\usepackage{rmexocptce}																% for DS with cptce eval
\usepackage{pythontex}

%\cfoot{}	 													% if no page number is needed
\renewcommand\arraystretch{1.5}		% stretch table line height

\begin{document}

\begin{header}
Les défis confinés -- Épisode 2
\end{header}

\section*{Combien d'atomes de fer dans un clou ?}

\begin{figure}[h]
\center
\includegraphics[scale=0.05]{images/qr_lqc_semoule.png}
\end{figure}

Regarde la vidéo en cliquant sur ce lien : \href{https://www.youtube.com/watch?v=xbgqRGvDJ2I}{https://www.youtube.com/watch?v=xbgqRGvDJ2I} ou en scannant le QR code ci-dessus.

Dans les commentaires de la vidéo, on peut lire approximativement cette phrase :
\og Suffit de diviser 500 grammes par la masse d'un grain de semoule en gramme \fg{}.

\begin{enumerate}
\item Quel est le nombre de grains de semoule dans un paquet de \unit{500}{g} ?

\emph{Donnée : la masse d'un grain de semoule est $m_\mathrm{grain}=\unit{1{,}25}{mg}$.}

\item Ce résultat est-il en accord avec celui obtenu dans la vidéo ?

\item Quel est le nombre d'atomes de fer dans un clou de masse \unit{4{,}2}{g} ?

\emph{Donnée : la masse d'un atome de fer est $m_\mathrm{Fe} = \unit{9{,}3\times 10^{-26}}{kg}$.}

\item Si l'on pouvait agrandir ce clou suffisamment, serait-il possible de compter tous les atomes de ce clou ?
Justifier.
\end{enumerate}

\section*{Construire un atome}

Cet activité est à faire en utilisant l'application construire un atome, accessible en scannant le code ci-dessous ou en cliquant sur le lien suivant : \href{https://phet.colorado.edu/sims/html/build-an-atom/latest/build-an-atom_fr.html}{https://phet.colorado.edu/sims/html/build-an-atom/latest/build-an-atom\_fr.html}.

\begin{figure}[h]
\center
\includegraphics[scale=0.05]{images/qr_phetcolorado_atom.png}
\end{figure}

\begin{enumerate}
\item Un atome d'hélium est formé de deux protons, deux neutrons et deux électrons.
Construire un atome d'hélium.
Quel est le symbole de cet élément ?

\item Construire l'atome de carbone formé de deux neutrons.
Décrire la composition de cet atome.

\item Après avoir coché la case \og Afficher la stabilité / l'instabilité \fg{}, construire un atome de béryllium stable.
Décrire la composition de cet atome.

\item À quoi correspond le nombre de masse affiché après avoir cliqué sur le signe \og + \fg{} du panneau du même nom ?

\item Un atome de fluor forme facilement un anion fluorure $\fluorure$.
Comment fabriquer cet ion ?
\end{enumerate}

\section*{Tracer une courbe avec python\texttrademark{}}

\begin{figure}[h]
\center
\includegraphics[scale=0.05]{images/qr_console_python.png}
\end{figure}
Il est possible d'utiliser python en ligne.
Pour cela, scanner le QR code ci-dessus ou cliquer sur le lien suivant : \href{https://www.lelivrescolaire.fr/outils/console-python}{https://www.lelivrescolaire.fr/outils/console-python}.
Vous y trouverez un environnement de travail similaire à celui d'edupython utilisé en classe.

On souhaite retracer la courbe qui figure à la fin du DM des vacances de la Toussaint à partir des données de concentration et de masse volumique de la saumure représentées dans le tableau ci-dessous.
\begin{table}[h]
\center
\begin{tabular}{l|c|c|c|c|c|c|c|c|c|c}
$C_\mathrm{m}$ (g/L) & 0 & 31 & 62 & 96 & 130 & 166 & 204 & 243 & 283 & 311 \\
\hline
$\rho$ (kg/L) & 1{,}000&1{,}020&1{,}041&1{,}063&1{,}086&1{,}109&1{,}132&1{,}156&1{,}180&1{,}197 \\
\end{tabular}
\end{table}

Le programme utilisé pour tracer la courbe est similaire à celui du fichier \texttt{densite\_saumure.py} :
\begin{pyverbatim}[stderr][numbers=left]
import matplotlib.pyplot as plt

Cm  = [0, 31, 62, 96, 130, 166, 204, 243, 283, 311]
rho = [1.000,1.020,1.041,1.063,1.086,1.109,1.132,1.156,1.180,1.197]

plt.plot(Cm, rho, color="yellow")

plt.xlabel("Concentration massique en sel")
plt.ylabel("Masse volumique (kg/L)")
plt.title("Le titre")

plt.grid()
plt.show()
\end{pyverbatim}

\begin{enumerate}
\item Copie-colle le contenu du fichier \texttt{densite\_saumure.py} dans la fenêtre gauche de l'environnement python (ouvre le fichier avec un éditeur de texte comme Notepad par exemple).
Exécute le programme en cliquant sur \og Voir le résultat \fg{} ou en appuyant simultanément sur les touches \texttt{CTRL} et \texttt{ENTRÉE}.
Un graphique devrait apparaitre dans la fenêtre de droite (si ce n'est pas le cas, cliquer sur \og GRAPH \fg{}).

\item Que reconnais-tu aux lignes 3 et 4 du programme ?

\item À ton avis, à quoi sert la ligne 6 ?

\item En jaune, la courbe n'est pas bien visible.
Modifie le programme pour que la courbe soit rouge (attention python ne peut interpréter que l'anglais !).

\item Le nom de l'axe des abscisses ne comporte pas d'unité.
C'est mal !
Rectifie cet oubli.

\item Modifie le programme pour que le titre du graphique soit \og Évolution de la masse volumique de la saumure \fg{} suivi de vos initiales.

\emph{Exemple : pour ton professeur de physique-chimie, cela donnerait \og Évolution de la masse volumique de la saumure - RM. \fg{}}

\item Copie-colle la courbe obtenue dans la conversation dédiée aux défis confinés 2 sur Teams.
\end{enumerate}


\end{document}