\documentclass[12pt,a4paper]{article}
\usepackage{rmpackages}																% usual packages
\usepackage{rmtemplate}																% graphic charter
\usepackage{rmexocptce}																% for DS with cptce eval

%\cfoot{} 													% if no page number is needed
%\renewcommand\arraystretch{1.5}		% stretch table line height

\begin{document}

\begin{header}
TP

Son et musique
\end{header}

\section*{Formation des groupes}

\begin{enumerate}
\item[•] S'assurer qu'il y a au minimum un Phyphox dans chaque groupe ;
\item[•] Niveaux proches
\item[•] Binômes
\end{enumerate}

\section*{Séance}

\begin{enumerate}
\item[•] Évaluation diagnostique : Plickers 006 -- Son
\item[•] Distribuer les sujets.
\item[•] Consignes :
\begin{itemize}
\item un CR chacun (suite de votre préparation), j'en tire un au sort à la fin ;
\item diapasons, première étape : écrire un protocole permettant de vérifier la fréquence du son émis par le diapason, dix minutes max
\item guitares, première étape : reformuler le problème en utilisant un vocabulaire scientifique, dix minutes max.
\end{itemize}
\end{enumerate}

\section*{Sujet Diapason}

\begin{enumerate}
\setcounter{enumi}{5}
\item \anarai{}

Acquérir le son émis par le diapason, mesurer la période du signal (bonus si mesure sur plusieurs périodes), calculer sa fréquence.

\item \rea{}

Schéma d'une période, avec nom de l'instrument, valeur de la période.

\item[Bonus.] Générateur de son Phyphox.
\end{enumerate}

\begin{center}
\begin{tabular}{l|l|c}
\textbf{Compétence} & \textbf{Aptitude} / Observable & \textbf{Niveau} \\
\hline \hline
\anarai 	& \textbf{Élaborer un protocole qui répond à la question} 	& \\
				& Protocole OK																				& A \\
				& 	Idée ok mais protocole confus												& B \\
				& Quelle grandeur permet de calculer la fréquence ?			& C \\
				& Mesure la période du signal sonore										& D \\
\hline
\rea		 	& \textbf{Faire des observations utiles à l'activité} 	& \\
				& Schémas OK																				& A \\
				& 	Manque un élément																	& B \\
				& Manque plusieurs éléments													& C \\
				& Schémas illisibles																		& D \\
\end{tabular}
\end{center}

\section*{Sujet Guitare}

\begin{enumerate}
\setcounter{enumi}{5}
\item \app{}

La n-ième corde de la guitare produit-elle un son à la fréquence voulue ?

\anarai{}

Acquérir le son émis par la guitare, mesurer la période du signal (bonus si mesure sur plusieurs périodes), calculer sa fréquence.

\item[Bonus.] Comparaison La\textsubscript{3} diapason et guitare.

\item[Bonus.] Générateur de son Phyphox.
\end{enumerate}

\begin{center}
\begin{tabular}{l|l|c}
\textbf{Compétence} & \textbf{Aptitude} / Observable & \textbf{Niveau} \\
\hline \hline
\app		 	& \textbf{Faire des observations utiles à l'activité} 	& \\
				& Formulation du problème OK													& A \\
				& 	Qu'est-ce que ça veut dire "accordée" en terme scientifiques ?	& B \\
				& Mots-clefs : émettre, son, fréquence										& C \\
				& La corde choisie emet-elle un son à la bonne fréquence	& D \\
\hline
\anarai 	& \textbf{Élaborer un protocole qui répond à la question} 	& \\
				& Protocole OK																				& A \\
				& 	Idée ok mais protocole confus												& B \\
				& Quelle grandeur permet de calculer la fréquence ?			& C \\
				& Mesure la période du signal sonore										& D \\
\end{tabular}
\end{center}

\end{document}
