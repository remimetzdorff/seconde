\documentclass[12pt,a4paper]{article}
\usepackage{rmpackages}																% usual packages
\usepackage{rmtemplate}																% graphic charter
\usepackage{rmexocptce}																% for DS with cptce eval
\usepackage{pythontex}

\cfoot{} 													% if no page number is needed
%\renewcommand\arraystretch{1.5}		% stretch table line height

\begin{document}

\begin{header}
Les défis confinés -- Épisode 4
\end{header}

\section*{Un programme pour calculer la masse d'un atome}

\begin{pyverbatim}
masse_nucleon = 1.67e-27    # masse d'un nucléon en kg
masse_electron = 9.1e-31    # masse d'un électron en kg

Z = 6                       # nombre de protons ou numéro atomique
A = 14                      # nombre de nucléons ou nombre de masse

masse = A * masse_nucleon

print("La masse de l'atome est ", masse, " kg")
\end{pyverbatim}

Le programme \texttt{masse\_atomique.py} ci-dessus peut être exécuté en utilisant l'environnement Python \href{https://www.lelivrescolaire.fr/outils/console-python}{https://www.lelivrescolaire.fr/outils/console-python}.
Pour cela, copie-colle le contenu du fichier \texttt{masse\_atomique.py} dans la fenêtre gauche de l'environnement python (ouvre le fichier avec un éditeur de texte comme Notepad par exemple).
Exécute le programme en cliquant sur \og Voir le résultat \fg{} ou en appuyant simultanément sur les touches \texttt{CTRL} et \texttt{ENTRÉE}.
L'affichage des résultats du programme doit apparaitre dans la fenêtre de droite (si ce n'est pas le cas, cliquer sur \og TEXTE \fg{}).

\emph{Si l'utilisation de Python n'est pas possible chez toi, tu peux tout de même faire le défi à l'exception de la question \ref{quest:python} en effectuant les calculs à la calculatrice et en utilisant les consignes en italiques pour les questions \ref{quest:alt1}, \ref{quest:alt2} et \ref{quest:alt3}.}

\begin{enumerate}
\item À ton avis, à quoi sert le programme \texttt{masse\_atome.py} ci-dessus ?

\item Représenter le noyau de l'atome utilisé dans le programme en utilisant l'écriture conventionnelle.

\item
\label{quest:alt1}
Modifier le programme pour qu'il calcule et affiche aussi la masse du nuage électronique de cet atome, c'est-à-dire à la masse des électrons qui entourent le noyau.
Quelle est la masse du nuage électronique de l'atome utilisé dans le programme ?

\emph{Calculer la masse du nuage électronique de l'atome utilisé dans le programme.}

\item
\label{quest:alt2}
En utilisant les valeurs données par le programme, comparer la masse de l'atome à celle de son nuage électronique.

\emph{En utilisant les valeurs calculées à l'aide d'une calculatrice, comparer la masse de l'atome à celle de son nuage électronique.}

\item
\label{quest:python}
Modifier le programme pour qu'il compare la masse de l'atome à celle de son nuage électronique.
Le programme devra afficher la phrase \og L'atome est ... fois plus lourd que son nuage électronique. \fg{}

\item
\label{quest:alt3}
Utilise ton programme pour comparer la masse de l'atome de fer $_\text{26}^\text{56}\text{Fe}$ et celle de son nuage électronique.
Recopie la phrase affichée par le programme sur ta feuille.

\emph{Même question mais en utilisant les valeurs trouvées avec la calculatrice.}
\end{enumerate}
\textbf{Remarque :} Pour l'écriture des puissances de 10, Python, tout comme ta calculatrice utilise \texttt{e} pour remplacer $\times 10$.
Ainsi, $9{,}1\times10^{-31}$ devient \texttt{9.1e-31} dans le programme.

\end{document}