\include{template}

\begin{document}

%\maketitle

\begin{header}
\begin{center}
\LARGE
\textbf{Premier cours}
\end{center}
\end{header}

\subsubsection*{Entrée des élèves}
Je reste devant la porte côté couloir et accueille chaque élève. Un œil sur le couloir, un sur la salle.
\begin{itemize}
\item \og Allez-y, vous pouvez rentrer et vous placer dans la salle. \fg{} 
\item \og Bonjour. \fg{} $\times 36$ accueil individuel.
\end{itemize}
Laisser la porte ouverte et se placer devant la classe, attendre l'attention générale.
Si les élèves sont assis, les faire se lever et attendre le calme aussi longtemps qu'il le faudra.
Expliquer que l'on se dira bonjour comme cela toute l'année.
\begin{itemize}
\item \og Bonjour à toutes et tous, vous pouvez vous asseoir. \fg{} Accueil du groupe.
\end{itemize}

\subsubsection*{Présentation personnelle}
Écrire mon nom au tableau, puis physique-chimie.
\begin{itemize}
\item \og Je suis Monsieur Metzdorff (\textit{R. Metzdorff}) et je serai votre professeur de physique chimie cette année. \fg{}
\end{itemize}

\subsubsection*{Appel}
Faire l'appel, vérifier l'orthographe et la prononciation, regarder chaque élève. Il faut avoir lu la liste des élèves à l'avance pour éviter de buter sur chaque nom.

\subsubsection*{Emploi du temps}
Vérifier qu'on a tous le même.
Présenter l'organisation des sonneries, cas particulier des TP partagés avec la SVT avec 10 min de pause entre PC et SVT.
\begin{itemize}
\item \og Le lundi nous avons une heure de cours de telle heure à telle heure et 1h25 de TP en demi groupe.
Quand un demi groupe a SVT l'autre a PC .\fg{}
\end{itemize}

\subsubsection*{Liste des fournitures}
\begin{itemize}
\item \og Je vais maintenant vous indiquer les fournitures scolaires nécessaires pour la physique-chimie : notez les dans votre agenda, vous devrez les avoir pour la semaine prochaine. \fg{}
Écrire au tableau la date du prochain cours, PC : liste
\item Classeur souple grand format ;
\item Cahier de brouillon ;
\item Calculatrice. Celle de collège va très bien
\item Blouse en coton avec des boutons pression : \og Je vous dirai quand l'amener. \fg{}
\item Livre : \og Je vous dirai quand il le faudra. \fg{}
\end{itemize}

\subsubsection*{Fiche de présentation}
Penser à bien donner la consigne.
\begin{itemize}
\item \og Je vous donne une petite fiche de renseignements, vous avez 5 min pour la remplir. \textit{A la fin :} Posez les sur le coin droit de votre table, je passe les ramasser. \fg{} Circuler dans la salle et surveiller le temps.
\end{itemize}

\subsubsection*{Plan de classe}
Faire circuler le plan de classe adapté à l'agencement de la classe.
\begin{itemize}
\item \og Je vous fait circuler le plan de classe, complétez le. Gardez le orienté correctement pour éviter les erreurs. \fg{}
\item \og Ce plan de classe est fixé à partir d'aujourd'hui sauf si je décide de le modifier.
S'il n'y a pas de problème avec cette répartition, je n'ai aucune raison de le changer.
A vous de voir... \fg{} 
\end{itemize}

\subsubsection*{Règlement}
\begin{itemize}
\item \og Pendant que le plan de classe circule, je vais vous rappeler quelques points du règlement qui seront importants dans la classe.\fg{}
\item[•] respect : des uns et des autres, du matériel.
\item porte fermée : vous allez en étude, pas la peine de frapper.
\item les retards doivent être justifiés, les cours doivent être rattrapés, les devoirs le seront aussi au cours suivant (pas la peine d'être absent juste pour les ds)
\item[•] s'engager dans les activités
\item le travail doit être fait, je le vérifierai régulièrement sinon punition.
\item les téléphones sont éteints pendant les cours. nous les utiliserons parfois, dans ce cas je vous dirai quelle appli télécharger.
\end{itemize}

\subsubsection*{Présentation du programme}
\begin{itemize}
\item \og Passons à des choses plus intéressantes... \fg{} 
\end{itemize}

\end{document}