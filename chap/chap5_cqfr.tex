\documentclass[12pt,a5paper]{article}
\usepackage{rmpackages}																% usual packages
\usepackage{rmtemplate}																% graphic charter
%\usepackage{rmexocptce}																% for DS with cptce eval

\cfoot{} 													% if no page number is needed
%\renewcommand\arraystretch{1.5}		% stretch table line height

\begin{document}

\begin{header}
CQFR -- Chapitre 5

Le son
\end{header}

À la fin du chapitre, je suis capable de :
\begin{itemize}
\item[•] expliquer comment un son est émis et comment il se propage ;
\item[•] donner la valeur de la vitesse du son dans l'air ;
\item[•] donner et utiliser la formule permettant de calculer la vitesse d'un son (5 p 216) ;
\item[•] identifier un signal périodique ;
\item[•] \textbf{mesurer une période, calculer une fréquence} (20 p 219) ;
\item[•] donner le domaine de fréquences des sons audibles, des ultrasons et des infrasons (8 p 217) ;
\item[•] donner l'unité du niveau d'intensité sonore.
\end{itemize}

\end{document}