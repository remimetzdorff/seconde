\documentclass[12pt,a4paper]{article}
\usepackage{rmpackages}																% usual packages
\usepackage{rmtemplate}																% graphic charter
\usepackage{rmexocptce}																% for DS with cptce eval
\usetikzlibrary{shapes}

\cfoot{} 													% if no page number is needed
%\renewcommand\arraystretch{1.5}		% stretch table line height

\begin{document}

\begin{header}
TP

Son et musique
\end{header}

\hrule{}
\vspace{5pt}

L'expérience \emph{Générateur de son} de Phyphox permet comme son nom l'indique de produire un son à la fréquence désirée en utilisant le haut parleur du smartphone.
\begin{itemize}
\item[•] Un membre du groupe est l'émetteur : il choisi une note et la produit en utilisant Phyphox.
\item[•] Les autres membres du groupe doivent déterminer de quelle note il s'agit.
\end{itemize}

\begin{center}
\begin{tabular}{l|c|c|c|c|c|c|c}
\textbf{Note}						& \textbf{Do\textsubscript{1}} & \textbf{Ré\textsubscript{1}} & \textbf{Mi\textsubscript{1}} & \textbf{Fa\textsubscript{1}} & \textbf{Sol\textsubscript{1}} & \textbf{La\textsubscript{1}} & \textbf{Si\textsubscript{1}} \\
\hline
\textbf{Fréquence (Hz)} 	& 65{,}4 & 73{,}4 & 82{,}4 & 87{,}3 & 98{,}0 & 110 & 123 \\
\hline \hline
\textbf{Note}						& \textbf{Do\textsubscript{2}} & \textbf{Ré\textsubscript{2}} & \textbf{Mi\textsubscript{2}} & \textbf{Fa\textsubscript{2}} & \textbf{Sol\textsubscript{2}} & \textbf{La\textsubscript{2}} & \textbf{Si\textsubscript{2}} \\
\hline
\textbf{Fréquence (Hz)} 	& 131 & 147 & 165 & 175 & 196 & 220 & 247 \\
\hline \hline
\textbf{Note}						& \textbf{Do\textsubscript{3}} & \textbf{Ré\textsubscript{3}} & \textbf{Mi\textsubscript{3}} & \textbf{Fa\textsubscript{3}} & \textbf{Sol\textsubscript{3}} & \textbf{La\textsubscript{3}} & \textbf{Si\textsubscript{3}} \\
\hline
\textbf{Fréquence (Hz)} 	& 262 & 294 & 330 & 350 & 392 & 440 & 494
\end{tabular}
\end{center}

\vspace{5pt}
\hrule{}
\vspace{5pt}

L'expérience \emph{Générateur de son} de Phyphox permet comme son nom l'indique de produire un son à la fréquence désirée en utilisant le haut parleur du smartphone.
\begin{itemize}
\item[•] Un membre du groupe est l'émetteur : il choisi une note et la produit en utilisant Phyphox.
\item[•] Les autres membres du groupe doivent déterminer de quelle note il s'agit.
\end{itemize}

\begin{center}
\begin{tabular}{l|c|c|c|c|c|c|c}
\textbf{Note}						& \textbf{Do\textsubscript{1}} & \textbf{Ré\textsubscript{1}} & \textbf{Mi\textsubscript{1}} & \textbf{Fa\textsubscript{1}} & \textbf{Sol\textsubscript{1}} & \textbf{La\textsubscript{1}} & \textbf{Si\textsubscript{1}} \\
\hline
\textbf{Fréquence (Hz)} 	& 65{,}4 & 73{,}4 & 82{,}4 & 87{,}3 & 98{,}0 & 110 & 123 \\
\hline \hline
\textbf{Note}						& \textbf{Do\textsubscript{2}} & \textbf{Ré\textsubscript{2}} & \textbf{Mi\textsubscript{2}} & \textbf{Fa\textsubscript{2}} & \textbf{Sol\textsubscript{2}} & \textbf{La\textsubscript{2}} & \textbf{Si\textsubscript{2}} \\
\hline
\textbf{Fréquence (Hz)} 	& 131 & 147 & 165 & 175 & 196 & 220 & 247 \\
\hline \hline
\textbf{Note}						& \textbf{Do\textsubscript{3}} & \textbf{Ré\textsubscript{3}} & \textbf{Mi\textsubscript{3}} & \textbf{Fa\textsubscript{3}} & \textbf{Sol\textsubscript{3}} & \textbf{La\textsubscript{3}} & \textbf{Si\textsubscript{3}} \\
\hline
\textbf{Fréquence (Hz)} 	& 262 & 294 & 330 & 350 & 392 & 440 & 494
\end{tabular}
\end{center}

\vspace{5pt}
\hrule{}
\vspace{5pt}

L'expérience \emph{Générateur de son} de Phyphox permet comme son nom l'indique de produire un son à la fréquence désirée en utilisant le haut parleur du smartphone.
\begin{itemize}
\item[•] Un membre du groupe est l'émetteur : il choisi une note et la produit en utilisant Phyphox.
\item[•] Les autres membres du groupe doivent déterminer de quelle note il s'agit.
\end{itemize}

\begin{center}
\begin{tabular}{l|c|c|c|c|c|c|c}
\textbf{Note}						& \textbf{Do\textsubscript{1}} & \textbf{Ré\textsubscript{1}} & \textbf{Mi\textsubscript{1}} & \textbf{Fa\textsubscript{1}} & \textbf{Sol\textsubscript{1}} & \textbf{La\textsubscript{1}} & \textbf{Si\textsubscript{1}} \\
\hline
\textbf{Fréquence (Hz)} 	& 65{,}4 & 73{,}4 & 82{,}4 & 87{,}3 & 98{,}0 & 110 & 123 \\
\hline \hline
\textbf{Note}						& \textbf{Do\textsubscript{2}} & \textbf{Ré\textsubscript{2}} & \textbf{Mi\textsubscript{2}} & \textbf{Fa\textsubscript{2}} & \textbf{Sol\textsubscript{2}} & \textbf{La\textsubscript{2}} & \textbf{Si\textsubscript{2}} \\
\hline
\textbf{Fréquence (Hz)} 	& 131 & 147 & 165 & 175 & 196 & 220 & 247 \\
\hline \hline
\textbf{Note}						& \textbf{Do\textsubscript{3}} & \textbf{Ré\textsubscript{3}} & \textbf{Mi\textsubscript{3}} & \textbf{Fa\textsubscript{3}} & \textbf{Sol\textsubscript{3}} & \textbf{La\textsubscript{3}} & \textbf{Si\textsubscript{3}} \\
\hline
\textbf{Fréquence (Hz)} 	& 262 & 294 & 330 & 350 & 392 & 440 & 494
\end{tabular}
\end{center}

\vspace{5pt}
\hrule{}
\vspace{5pt}



\end{document}