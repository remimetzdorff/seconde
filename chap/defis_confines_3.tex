\documentclass[12pt,a4paper]{article}
\usepackage{rmpackages}																% usual packages
\usepackage{rmtemplate}																% graphic charter
%\usepackage{rmexocptce}																% for DS with cptce eval
\usepackage{pythontex}

\cfoot{}	 													% if no page number is needed
\renewcommand\arraystretch{1.5}		% stretch table line height

\begin{document}

\begin{header}
Les défis confinés -- Épisode 3
\end{header}

\section*{A propos de Benjamin Franklin et lord Rayleigh}

Dans le TP -- Mesurer une molécule d'huile, deux scientifiques sont mentionnés : Benjamin Franklin et John William Strutt Rayleigh plus souvent nommé lord Rayleigh.
Lors de la prochaine séance de TP nous reviendrons rapidement sur l'expérience réalisée par Franklin et nous essaierons de comprendre le rôle de chacun dans son interprétation.
\footnote{Plusieurs autres scientifiques ont participé à la compréhension fine de l'expérience : Agnès Pockels, Henri Devaux, Irving Langmuir...
Cette expérience est plus subtile qu'il n'y parait peut-être.}

Ton travail : réaliser une courte biographie pour chacun de ces deux personnages.
Elle doit au minimum répondre à ces deux questions :
\begin{itemize}
\item[•] Où et quand ont-ils vécu ?
\item[•] Qui étaient-ils ?
\item[•] Quels étaient leurs domaines de prédilection ?
\end{itemize}
Tu peux bien sûr ajouter les détails que tu jugeras pertinents.

En classe, deux d'entre vous nous présenterons à l'oral les deux personnages : pas plus de cinq minutes chacun.

\section*{Réviser avec \emph{C'est pas sorcier}}

Pour rafraichir tes connaissances sur l'atome en compagnie de Fred et Jamy, le tout agrémenté de boutades, calembours et autres jeux de mots, tu peux regarder l'épisode \og Voyage au cœur de la matière \fg{} de l'émission C'est pas sorcier : \href{https://youtu.be/hxMNJ6-8n5c}{https://youtu.be/hxMNJ6-8n5c}.
La première partie (jusqu'à environ 9 min nous concerne cette année à l'exception de quelques détails).
La suite t'emmènera plus loin dans le monde des particules subatomique (plus petites encore que l'atome).

\end{document}