\documentclass[12pt,a4paper]{article}
\usepackage{rmpackages}																% usual packages
\usepackage{rmtemplate}																% graphic charter
\usepackage{rmexocptce}																% for DS with cptce eval
\usepackage{pythontex}

%\cfoot{} 													% if no page number is needed
%\renewcommand\arraystretch{1.5}		% stretch table line height

\begin{document}

\begin{header}
Les défis confinés -- Épisode 4

Correction
\end{header}

\begin{enumerate}
\item Le programme \texttt{masse\_atomique.py} permet de calculer la masse approchée d'un atome.

\item En regardant dans la classification périodique des éléments, on voit que l'élément ayant 6 comme numéro atomique est le carbone de symbole C.
Il s'agit donc ici de l'atome de carbone puisqu'il a 6 protons.
\[^\text{14}_\text{6}\text{C}\]

\item L'atome possède autant d'électrons que de protons puisqu'il est neutre : il a donc 6 électrons.
\[m_\mathrm{nuage\ électronique} = Z \times m_\mathrm{e} = 6 \times 9{,}1\times 10^{-31} = \unit{5{,}46\times 10^{-30}}{kg}.\]
\emph{Cf. fin de correction pour le programme complété.}

\item
\[\frac{m_\mathrm{atome}}{m_\mathrm{nuage\ électronique}} = \frac{2{,}34\times 10^{-26}}{5{,}46\times 10^{-30}} \approx 4\,282.\]
L'atome est environ 4300  fois plus lourds que son nuage électronique.

\item \emph{Cf. fin de correction pour le programme complété.}

\item
En utilisant le programme après avoir modifié les valeurs de $Z$ et $A$ on trouve directement que l'atome de fer est environ 4\,000 fois plus lourd que son nuage électronique.

Sinon, il faut reprendre les calculs des questions précédentes :
\[m_\mathrm{atome} \approx A \times m_\mathrm{nucléon} = 56 \times 1{,}67\times 10^{-27} \approx \unit{9{,}35\times 10^{-26}}{kg},\]
\[m_\mathrm{nuage\ électronique} = Z \times m_\mathrm{e} = 26 \times 9{,}1\times 10^{-31} \approx \unit{2{,}37\times 10^{-29}}{kg},\]
\[\frac{m_\mathrm{atome}}{m_\mathrm{nuage\ électronique}} = \frac{9{,}35\times 10^{-26}}{2{,}37\times 10^{-29}} \approx 3\,952.\]
L'atome de fer est environ 4\,000 fois plus lourd que son nuage électronique.
\end{enumerate}

\newpage

\section*{Programme modifié}

\begin{pyverbatim}
masse_nucleon = 1.67e-27    # masse d'un nucléon en kg
masse_electron = 9.1e-31    # masse d'un électron en kg

#Z = 6                       # nombre de protons ou numéro atomique
#A = 14                      # nombre de nucléons ou nombre de masse

# Valeurs de Z et A pour le fer
Z = 26
A = 56

masse = A * masse_nucleon
print("La masse de l'atome est ", masse, " kg")

masse_nuage_electronique = Z * masse_electron
print("La masse du nuage electronique est ", masse_nuage_electronique, " kg")

comparaison = masse/masse_nuage_electronique
print("L'atome est ", comparaison, " fois plus lourds que son nuage electronique")
\end{pyverbatim}

\section*{Résultats de l'exécution du programme}

\begin{ttfamily}
\begin{pycode}
masse_nucleon = 1.67e-27    # masse d'un nucléon en kg
masse_electron = 9.1e-31    # masse d'un électron en kg

#Z = 6                       # nombre de protons ou numéro atomique
#A = 14                      # nombre de nucléons ou nombre de masse

# Valeurs de Z et A pour le fer
Z = 26
A = 56

masse = A * masse_nucleon
print("\\noindent{}La masse de l'atome est "+str(masse)+ " kg\n")

masse_nuage_electronique = Z * masse_electron
print("\\noindent{}La masse du nuage electronique est "+str(masse_nuage_electronique)+" kg\n")

comparaison = masse/masse_nuage_electronique
print("\\noindent{}L'atome est "+str(comparaison)+" fois plus lourds que son nuage electronique")
\end{pycode}
\end{ttfamily}

\end{document}