\documentclass[12pt,a4paper]{article}
\usepackage{rmpackages}																% usual packages
\usepackage{rmtemplate}																% graphic charter
\usepackage{rmexocptce}																% for DS with cptce eval

%\cfoot{} 													% if no page number is needed
\renewcommand\arraystretch{1.5}		% stretch table line height

\begin{document}

\begin{header}
Chapitre 2 -- Exercices
\end{header}

\emph{Rappel : la notation du livre pour la concentration massique est $t$ alors que nous avons utilisé $C_\mathrm{m}$ dans le cours mais c'est la même chose !}

\section*{Exercices d'application faits en classe}

\begin{multicols}{4}
\begin{itemize}
\item[•] 3 page 42
\item[•] 4 page 42
\item[•] 6 page 42
\item[•] 9 page 43
\item[•] 10 page 43
\item[•] 13 page 43
\item[•] 14 page 43
\item[•] 20 page 44 modif
\item[•] 24 page 44 modif
\item[•] 27 page 45
\end{itemize}
\end{multicols}

\section*{Exercices à la maison}

\begin{multicols}{4}
\begin{itemize}
\item[•] 5 page 42
\item[•] 7 page 42
\item[•] 11 page 43
\item[•] 19 page 44
\item[•] 30 page 46
\item[•] 31 page 46
\end{itemize}
\end{multicols}

\section*{Corrections}

\subsubsection*{Exercice 5 page 42}

\begin{enumerate}
\item
\begin{equation}
t = \frac{m_\mathrm{soluté}}{V_\mathrm{solution}}.
\nonumber
\end{equation}
$t$ est en g/L, $m_\mathrm{soluté}$ en g et $V_\mathrm{solution}$ en L.
\item La masse de soluté est $7{,}00-5{,}00=\unit{2{,}00}{g}$ et le volume de la solution est celui de la fiole jaugée, c'est-à-dire $\unit{50}{mL}=\unit{0{,}050}{L}$.
En faisant l'application numérique, on trouve :
\begin{equation}
t = \frac{2{,}00}{0{,}050} = \unit{40}{g/L} 
\nonumber
\end{equation}
La concentration massique $t$ en soluté de la solution est $\unit{40}{g/L}$.
\end{enumerate}

\subsubsection*{Exercice 7 page 42}

Cette exercice permet de vérifier que l'on sait manipuler la formule de la concentration massique.
En faisant un \og produit en croix \fg{} (avec la méthode du triangle par exemple), on peut calculer la masse de soluté dissous ou le volume de solution.
Toutes les grandeurs sont exprimées avec la bonne unité, on peut faire les applications numériques directement, sans faire de conversion.

\begin{table}[h]
\center
\begin{tabular}{|l|c|c|c|}
\hline
\textbf{Masse de soluté dissous (g)}		& {\color{bleu_f}$10$}	& 8{,}0								& 0{,}15							\\
\hline
\textbf{Volume de la solution (L)}   		& 0{,}50							& {\color{bleu_f}$2{,}0$}	& 0{,}020						\\
\hline
\textbf{Concentration massique (g/L)}	& 20								& 4{,}0								& {\color{bleu_f}$7{,}5$}	\\
\hline
\end{tabular}
\end{table}

\subsubsection*{Exercice 11 page 43}

Corrigé dans le livre, page 307.

\subsubsection*{Exercice 19 page 44}

Cf. fiche protocole dilution.

\subsubsection*{Exercice 30 page 46}

\begin{enumerate}
\item Dans l'énoncé, on lit que la masse de soluté (l'acide lactique) est $m=\unit{0{,}23}{g}$ dans un volume $V_\mathrm{solution} = \unit{150}{mL}$.

On convertit le volume en litre :
\[ V_\mathrm{solution} = \unit{150}{mL} = \unit{0{,}150}{L}, \]
et on utilise la formule de la concentration massique :
\[ C_\mathrm{m} = \frac{m}{V_\mathrm{solution}} = \frac{0{,}23}{0{,}150} \approx \unit{1{,}5}{g/L}. \]
La concentration massique d'acide lactique dans le lait analysé est \unit{1{,}5}{g/L}.

\item Dans l'énoncé, on nous indique qu'un lait est frais si sa concentration en acide lactique est inférieure à \unit{1{,}8}{g/L} ce qui est le cas ici puisque $\unit{1{,}5}{g/L} < \unit{1{,}5}{g/L}$.
Le lait analysé est frais.
\end{enumerate}

\subsubsection*{Exercice 31 page 46}

Corrigé en classe.



\end{document}