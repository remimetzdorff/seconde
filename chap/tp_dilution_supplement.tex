\include{template_a5}
\cfoot{} %% if no page number is needed

\begin{document}

\begin{header}
TP

Un simple verre de sirop...

(coup de pouce)
\end{header}

\begin{enumerate}
\item Quel est le soluté qui nous intéresse dans la boisson de l'athlète ?
\item Identifier la solution mère et la solution fille.
\item Dans la relation rappelée dans l'énoncé, quelles sont les grandeurs connues ?
Indiquer leur valeur numérique.
Laquelle est inconnue ?
\item Reformuler l'objectif du TP en utilisant les termes suivant : solution, concentration massique, volume.
\item Expliquer en quelques lignes ce que vous souhaitez faire pour réaliser l'objectif.
\item À partir de la formule rappelée dans l'énoncé, exprimer la grandeur inconnue en fonction des autres puis calculer sa valeur numérique.
\item Parmi les choix suivants, quelles verreries permettent de prélever des volumes avec précision : bécher, éprouvette graduée, pipette jaugée, fiole jaugée ?
\item Faire la liste du matériel nécessaire.
\item Faire un schéma de la manipulation.
\item Lister les étapes de manipulation.
\item Réaliser la solution demandée.
\item Décrire la solution obtenue et la comparer au sirop de menthe.
\item Comment pourrait-on, visuellement, contrôler la concentration de la solution réalisée ? 
\end{enumerate}

\end{document}