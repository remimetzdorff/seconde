\documentclass[12pt,a4paper]{article}
\usepackage{rmpackages}																% usual packages
\usepackage{rmtemplate}																% graphic charter
\usepackage{rmexocptce}																% for DS with cptce eval
\usepackage{pythontex}

\cfoot{} 													% if no page number is needed
%\renewcommand\arraystretch{1.5}		% stretch table line height

\begin{document}

\begin{header}
Chapitre 3 -- Exercices
\end{header}

\section*{Applications sur les puissances de 10}

\begin{enumerate}
\item Pour les trois applications suivantes, écrire la phrase \og Je décale la virgule ... \fg{} puis le nombre sans utiliser les puissances de 10.

\textit{Exemple  : $5{,}72\times 10^{-2}$ : Je décale la virgule vers la gauche de deux rang en partant de là où elle est dans $5{,}72$ donc $5{,}72\times 10^{-2}=0{,}0572$}

\begin{multicols}{3}
\begin{enumerate}
\item[•]
\begin{pycode}
import random as rd
x = int(rd.random()*1000)/100.
n = int(rd.random()*4+1)
print("$" + str(int(x)) + "{,}" + str(int((x-int(x))*100)) + "\\times 10^{" + str(int(n)) + "}$" )
\end{pycode}

\item[•]
\begin{pycode}
import random as rd
x = int(rd.random()*1000)/10.
n = -int(rd.random()*4+1)
print("$" + str(int(x)) + "{,}" + str(int((x-int(x))*100)) + "\\times 10^{" + str(int(n)) + "}$" )
\end{pycode}

\item[•]
\begin{pycode}
import random as rd
x = int(rd.random()*1000)/1000.
n = int(rd.random()*4+1)
if rd.random() < 0.5:
    n *= -1
print("$" + str(int(x)) + "{,}" + str(int((x-int(x))*100)) + "\\times 10^{" + str(int(n)) + "}$" )
\end{pycode}
\end{enumerate}
\end{multicols}

\item Pour les applications suivantes, écrire le nombre sans utiliser les puissances de 10.
\begin{itemize}
\begin{multicols}{4}
\item[•]
\begin{pycode}
import random as rd
x = int(rd.random()*1000)/1000.
n = int(rd.random()*4+1)
if rd.random() < 0.5:
    n *= -1
print("$" + str(int(x)) + "{,}" + str(int((x-int(x))*100)) + "\\times 10^{" + str(int(n)) + "}$" )
\end{pycode}

\item[•]
\begin{pycode}
import random as rd
x = int(rd.random()*1000)/10.+1
n = int(rd.random()*4+1)
if rd.random() < 0.5:
    n *= -1
print("$" + str(int(x)) + "\\times 10^{" + str(int(n)) + "}$" )
\end{pycode}

\item[•]
\begin{pycode}
import random as rd
x = int(rd.random()*1000)/100.
n = int(rd.random()*4+1)
print("$" + str(int(x)) + "{,}" + str(int((x-int(x))*100)) + "\\times 10^{" + str(int(n)) + "}$" )
\end{pycode}

\item[•]
\begin{pycode}
import random as rd
n = int(rd.random()*4+1)
if rd.random() < 0.5:
    n *= -1
print("$ 10^{" + str(int(n)) + "}$" )
\end{pycode}
\end{multicols}
\end{itemize}
\end{enumerate}

\section*{Le plus petit film du monde}

\begin{center}
\includegraphics[scale=0.05]{images/atom_boy_ibm.png}
\end{center}

En scannant le QR code ci-dessus ou en cliquant sur le \href{https://youtu.be/oSCX78-8-q0?t=28}{lien de la vidéo}, regarder le plus petit film du monde \og A boy and his atom \fg{}.
Comme le suggère le titre, chaque petite \og bille \fg{} dans le film est un atome (en réalité, ce sont des molécules de monoxyde de carbone mais on fera comme s'il s'agissait d'atomes uniques).

Quelle est la taille du (très) petit garçon de la vidéo ?

\section*{Atomes, molécules, cations où anions ?}

Pour chaque espèce chimique ci-dessous, dire si elle est constituée d'atomes, de molécules, de
cations ou d'anions :
\begin{multicols}{4}
\begin{itemize}
\item[•] Cu
\item[•] $\dioxydedecarbone$
\item[•] $\fluorure$
\item[•] $\acetatedethyle$
\item[•] $\ionoxonium$
\item[•] Si
\item[•] $\ionmagnesiumII$
\item[•] Au
\end{itemize}
\end{multicols}

\section*{Solides ioniques}

À l'aide de la section 3 du chapitre 3, donner la formule chimique des solides ioniques suivants :
\begin{multicols}{3}
\begin{itemize}
\item[•] Hydroxyde de sodium ;
\item[•] Sulfate de magnésium ;
\item[•] Iodure de fer II.
\end{itemize}
\end{multicols}


  



\end{document}