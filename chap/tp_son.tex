\documentclass[12pt,a4paper]{article}
\usepackage{rmpackages}																% usual packages
\usepackage{rmtemplate}																% graphic charter
\usepackage{rmexocptce}																% for DS with cptce eval

%\cfoot{} 													% if no page number is needed
%\renewcommand\arraystretch{1.5}		% stretch table line height

\begin{document}

\begin{header}
Mesurer la vitesse du son
\end{header}

\section*{Production d'un son}

\href{https://www.youtube.com/watch?v=2jSnnGC-HpI}{https://www.youtube.com/watch?v=2jSnnGC-HpI}

\section*{Méthode classique}

Générateur d'impulsion ultrason et deux micros.

\section*{Phyphox}

La \href{https://youtu.be/RhWikn-uRjk?t=1431}{conférence confinée de Julien Bobroff}.

Le faire collectivement :
\begin{itemize}
\item[•] topo phyphox, montrer phyphox à l'écran : \href{https://www.businessinsider.fr/us/how-to-mirror-iphone-to-mac}{https://www.businessinsider.fr/us/how-to-mirror-iphone-to-mac}
\item[•] faire venir plusieurs trinômes pour faire la manip.
\item[•] penser à stopper le groupe après les mesures.
\end{itemize}

\end{document}