\documentclass[12pt,a4paper,fleqn]{article}
\usepackage{rmpackages}																% usual packages
\usepackage{rmtemplate}																% graphic charter
\usepackage{rmexocptce}																% for DS with cptce eval

\cfoot{} 													% if no page number is needed
%\renewcommand\arraystretch{1.5}		% stretch table line height

\newcommand{\width}{16}
\newcommand{\height}{5}
\newcommand{\step}{.5}

\begin{document}

\begin{header}
Les défis confinés -- Épisode 8
\end{header}

\section*{Un peu de vocabulaire}

Regarder la vidéo \href{https://youtu.be/QFLGIwaZe2s}{https://youtu.be/QFLGIwaZe2s}.
\begin{enumerate}
\item Donner quelques exemples d'objets ou d'instruments qui comportent une ou plusieurs lentilles (au moins deux exemples en plus de ceux indiqués dans la vidéo).
\item Comment qualifie-t-on une image quand elle est visible sur un écran ?
\item Comment s'appellent les points $O$, $F$ et $F'$ du schéma ci-dessous ?
\item Comment appelle-t-on la flèche $AB$ ?
Et la flèche $A'B'$ ?
\item Comment appelle-t-on la grandeur qui caractérise une lentille ?
Quelle est son unité ?
\end{enumerate}

\begin{center}
\begin{tikzpicture}
\pgfmathsetmacro\x{(\width)/2}
\pgfmathsetmacro\y{(\height)/2}
\coordinate (O) at (\x,\y);
\draw [->, >=stealth, thick] (0, \y) --++ (\width, 0);

\draw [<->, >=angle 60, ultra thick] (\x, \y-3) -- (\x, \y+3);
\draw (O) node [below left] {$O$};
\draw (O) ++ (-3, 0) node [below left] {$F$};
\draw (O) ++ (-3, 0) node {$|$};
\draw (O) ++ (3, 0) node [below left] {$F'$};
\draw (O) ++ (3, 0) node {$|$};

\draw [->, >=stealth, ultra thick, red] (O) ++ (-6, 0) node [below left] {$A$} --++ (0, 2) node [below left] {$B$};
\draw [->, >=stealth, ultra thick, red] (O) ++ (6, 0) node [below right] {$A'$} --++ (0, -2) node [below right] {$B'$};
\end{tikzpicture}
\end{center}


\section*{Construction d'une image}

Regarder la vidéo \href{https://youtu.be/GAX\_ZorF1hE}{https://youtu.be/GAX\_ZorF1hE}.
Vous pouvez aussi consulter cette animation : \href{http://shorturl.at/klJM6}{http://shorturl.at/klJM6}.
\begin{enumerate}[resume]
\item Reproduire le schéma ci-dessous et construire l'image $A'B'$ de l'objet $AB$.
\end{enumerate}

\begin{center}
\begin{tikzpicture}
\foreach \y in {0,\step,...,\height} {
    \draw [gray_c] (0,\y) --++ (\width,0);
}
\foreach \x in {0,\step,...,\width} {
    \draw [gray_c] (\x,0) --++ (0, \height);
}

\pgfmathsetmacro\x{(\width)/2-2}
\pgfmathsetmacro\y{(\height)/2}
\coordinate (O) at (\x,\y);
\draw [->, >=stealth, thick] (0, \y) --++ (\width, 0);

\draw [<->, >=angle 60, ultra thick] (\x, \y-2) -- (\x, \y+2);
\draw (O) node [below left] {$O$};
\draw (O) ++ (-3, 0) node [below left] {$F$};
\draw (O) ++ (-3, 0) node {$|$};
\draw (O) ++ (3, 0) node [below left] {$F'$};
\draw (O) ++ (3, 0) node {$|$};

\draw [->, >=stealth, ultra thick, red] (O) ++ (-4.5, 0) node [below left] {$A$} --++ (0, 1) node [below left] {$B$};
\end{tikzpicture}
\end{center}

\end{document}