\documentclass[12pt,a4paper]{article}
\usepackage{rmpackages}																% usual packages
\usepackage{rmtemplate}																% graphic charter
\usepackage{rmexocptce}																% for DS with cptce eval

\newcounter{memory}

\begin{document}

\begin{header}
Devoir surveillé -- Correction
\end{header}

\section*{Exercice 1 -- Cours}

\begin{enumerate}
\item
La formule qui permet de calculer la concentration massique $C_\mathrm{m}$ est 
\[
C_\mathrm{m} = \frac{m_\mathrm{soluté}}{V_\mathrm{solution}}.
\]
$C_\mathrm{m}$ s'exprime en g/L, $m_\mathrm{soluté}$ en g et $V_\mathrm{solution}$ est en L.

La deuxième formule permet de calculer la masse volumique $\rho$ de la solution.

\item 
Avec le matériel présenté sur le schéma, on réalise une dissolution.
La balance permet de mesurer la masse de soluté nécessaire à la préparation de la solution.

\item
Il s'agit d'une fiole jaugée.

\item 
La taille d'un atome est d'environ \unit{0{,}1}{nm} soit $\unit{0{,}1\times 10^{-9}}{m}$ ou encore $\unit{10^{-10}}{m}$.

\item
\begin{multicols}{3}
\begin{itemize}
\item[•] $\permanganate$ : anion
\item[•] $\text{O}$ : atome
\item[•] $\ionmagnesiumII$ : cation
\item[•] $\acideoleique$ : molécule
\item[•] $\dioxydedecarbone$ :molécule
\end{itemize}
\end{multicols}
\end{enumerate}

\section*{Exercice 2 -- Solution sucrée pour sportif}

\begin{enumerate}
\item 
Dans la solution isotonique, le soluté est le sucre.

\item
\[
\unit{100}{mL} = \unit{0{,}100}{L}.
\]
\[
C_\mathrm{m} = \frac{m_\mathrm{soluté}}{V_\mathrm{solution}} = \frac{6}{0{,}100} = \unit{60}{g/L}.
\]
La concentration en masse en sucre de la solution isotonique est \unit{60}{g/L}.

\item
Quand le sportif a bu les deux tiers de sa gourde, il en reste un tiers donc le volume restant dans la gourde est
\[
\frac{1}{3} \times 0{,}75 = \unit{0{,}25}{L}.
\]
Il reste donc \unit{0{,}25}{L} dans la gourde du sportif.

\item 
\[
C_\mathrm{m} = \frac{m_\mathrm{soluté}}{V_\mathrm{solution}}
\]
donc
\[
m_\mathrm{soluté} = C_\mathrm{m} \times V_\mathrm{solution} = 60 \times 0{,}25 = \unit{15}{g}.
\]
La masse de sucre restante est \unit{15}{g}.

\item 
Le volume de la solution $S_2$ est celui de la gourde soit \unit{0{,}75}{L}.

\item
Pour cette question, on peut utiliser deux méthodes :
\paragraph*{Méthode 1} En utilisant le résultat de la question précédente.

\[
C_\mathrm{m} = \frac{m_\mathrm{soluté}}{V_\mathrm{solution}} = \frac{15}{0{,}75} = \unit{20}{g/L}.
\]
La concentration massique de la solution $S_2$ est \unit{20}{g/L}.

\paragraph*{Méthode 2} En utilisant la formule de la dilution

\[
C_\mathrm{m} \times V_\mathrm{m} = C_\mathrm{f} \times V_\mathrm{f}
\]
donc
\[
C_\mathrm{f} = \frac{C_\mathrm{m} \times V_\mathrm{m}}{V_\mathrm{f}} = \frac{60\times 0{,}25}{0{,}75} = \unit{20}{g/L}. 
\]
La concentration massique de la solution $S_2$ est \unit{20}{g/L}.

\paragraph*{Remarque} Les deux méthodes donnent le même résultat (heureusement !).

\item

La manipulation réalisée pour préparer la solution $\mathrm{S_2}$ est une dilution.

\item 
On voit que la solution $S_2$ a la même couleur que la solution de concentration \unit{20}{g/L} donc la concentration de la solution $S_2$ est aussi \unit{20}{g/L}.
Le résultat de cette analyse est bien en accord avec le résultat de la question 6.
\end{enumerate}

\section*{Exercice 3 -- La planète rouge}

\begin{enumerate}
\item Le cation est $\ionferIII$.

L'anion est $\oxyde$.

\item 
Vérifions que le composé de formule $\oxydedefer$ est bien neutre :
\begin{itemize}
\item[•] le cation $\ionferIII$ a trois charges positives et il y a deux cations $\ionferIII$ dans l'oxyde de fer ce qui fait 6 charges positives ;
\item[•] l'anion $\oxyde$ a deux charges négatives et il y a trois anions $\oxyde$ dans l'oxyde de fer ce qui fait 6 charges négatives.
\end{itemize}
La formule de l'oxyde de fer $\oxydedefer$ correspond bien à un composé neutre puisqu'il y a autant de charges négatives que de charges positives.

\item
La formule du chlorure de sodium est NaCl.

\item
La formule du sulfate de calcium est CaSO\textsubscript{4}.

\item
L'ion magnésium a pour formule $\ionmagnesiumII$.
\end{enumerate}

\end{document}