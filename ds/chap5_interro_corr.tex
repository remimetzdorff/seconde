\documentclass[12pt,a4paper,fleqn]{article}
\usepackage{rmpackages}																% usual packages
\usepackage{rmtemplate}																% graphic charter
\usepackage{rmexocptce}																% for DS with cptce eval

%\cfoot{} 													% if no page number is needed
\renewcommand\arraystretch{1.5}		% stretch table line height

\begin{document}

\begin{header}
Interrogation -- Chapitre 5

Correction
\end{header}

\section*{Exercice1 -- Cours}

\begin{enumerate}
\item Un objet qui vibre émet un son, mais il est souvent très faible.
Il faut l'associer à une caisse de résonance pour qu'il émette un son plus fort.
On peut l'illustrer avec le cas du diapason : les branches de la fourche vibrent et avec sa caisse de résonance, il émet un son fort.

\item $ T = \unit{1{,}25}{ms} = \unit{1{,}25\times10^{-3}}{s}.$
\[
f = \frac{1}{T} = \frac{1}{1{,}25\times10^{-3}} = \unit{800}{Hz} = \unit{8{,}00\times10^2}{Hz}.
\]
La fréquence de ce son est \unit{8{,}00\times10^2}{Hz}.

\item
\[
T=\frac{1}{f} = \frac{1}{15\times10^3} \approx \unit{6{,}7\times10^{-5}}{s}.
\]
La période de ce son est \unit{6{,}7\times10^{-5}}{s}.

\item Le signal 2 est un signal périodique puisqu'on peut identifier un motif qui se répète à l'identique à intervalles de temps réguliers, ce qui n'est pas le cas du signal 1.	
\end{enumerate}

\section*{Exercice 2 -- Par temps d'orage}

\begin{enumerate}
\item La vitesse du son dans l'air est d'environ \unit{340}{m\per s}.

\item $ d = \unit{2{,}0}{km} = \unit{2{,}0\times10^3}{m}.$
\[
v = \frac{d}{\Delta t} = \frac{2{,}0\times10^3}{6{,}0} \approx \unit{3{,}3\times10^2}{m/s}.
\]
D'après ces observations, la vitesse du son dans l'air est d'environ \unit{3{,}3\times10^2}{m/s}.

%\textbf{Remarque :} il n'est pas surprenant de trouver une valeur différente de la question précédente car la vitesse du son dans l'air dépend de la température et de l'humidité.
\end{enumerate}

\section*{Exercice 3 -- Communication en Terre du Milieu}

\begin{enumerate}
\item L'oreille humaine est sensible aux sons dont la fréquence est comprise entre \unit{20}{Hz} et \unit{20}{kHz}.
\label{quest:freq_human}

\item Les sons trop aigus pour être perçus par l'oreille humaine ont des fréquences supérieures à \unit{20}{kHz} : ce sont les ultrasons.

\item
Le signal du document 1 a une période $ T = \unit{100}{ms} = \unit{100\times10^{-3}}{s}$.
\[
f = \frac{1}{T} = \frac{1}{100\times10^{-3}} = \unit{10{,}0}{Hz}.
\]
Le cor de guerre Nain émet donc des infrasons car la fréquence du son qu'il émet est inférieure à \unit{20}{Hz}.

Or, d'après le document 2 et la réponse à la question \ref{quest:freq_human}, on voit que seuls les Nains sont sensibles aux infrasons.

Les Nains de Dain ont donc pu avertir Thorin (Nain) de leur arrivé en utilisant leur cor de guerre qui émet des infrasons.
Thranduil (Elfe) et Bard (Homme) n'ont pas entendu ce signal car ni l'un ni l'autre n'est sensible aux infrasons.




\end{enumerate}

\end{document}