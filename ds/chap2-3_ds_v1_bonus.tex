\documentclass[12pt,a4paper]{article}
\usepackage{rmpackages}																% usual packages
\usepackage{rmtemplate}																% graphic charter
\usepackage{rmexocptce}																% for DS with cptce eval

%\cfoot{} 													% if no page number is needed
%\renewcommand\arraystretch{1.5}		% stretch table line height

\begin{document}

\section*{Exercice 2 (Bonus)}

Il est possible de réaliser une mesure quantitative à partir d'une échelle de teinte.
Pour cela on met un chiffre sur l'intensité de la couleur d'une solution : l'absorbance notée $A$.
Plus la couleur de la solution est intense, plus l'absorbance est élevée.

On réalise des mesures d'absorbance pour chacune des solutions préparées à partir de la boisson isotonique :
\begin{center}
\begin{tabular}{l|c|c|c|c}
\textbf{Concentration} $C\mathrm{m}$ (g/l) & 40 & 20 & 10 & 5{,}0 \\
\hline
\textbf{Absorbance} $A$ & 8{,}0 & 4{,}0 & 2{,}0 & 1{,}0
\end{tabular}
\end{center}

\begin{enumerate}
\setcounter{enumi}{8}
\item \rea{}

Tracer la courbe représentant l'absorbance des solutions (en ordonnée) en fonction de leur concentration (en abscisse).

\item \anarai{}

Une mesure d'absorbance de la solution $S_2$ donne $A = 3{,}8$.
En déduire la concentration massique de la solution $S_2$.

\item \val{}

Ce résultat est-il en accord avec celui des question \textcolor{red_f}{6} et \textcolor{red_f}{8} ?

\item \anarai{}

Quelle serait l'absorbance d'une solution de concentration $C_\mathrm{m}=\unit{0}{g/L}$ ?

\end{enumerate}

\end{document}