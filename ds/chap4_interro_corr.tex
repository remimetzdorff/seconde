\documentclass[12pt,a4paper]{article}
\usepackage{rmpackages}																% usual packages
\usepackage{rmtemplate}																% graphic charter
\usepackage{rmexocptce}																% for DS with cptce eval

%\cfoot{} 													% if no page number is needed
\renewcommand\arraystretch{1.25}		% stretch table line height

\begin{document}

\begin{header}
Interrogation -- Chapitre 4

Correction
\end{header}

\section*{Exercice 1 -- Identifier un atome}

\begin{enumerate}
\item
\begin{itemize}
\item[•] en vert : neutron
\item[•] en orange : proton
\item[•] en bleu : électron
\item[•] on peut aussi faire figurer : noyau, nucléons, atome
\end{itemize}
\item Il y a deux protons et deux électrons donc l'ensemble est bien neutre : il s'agit bien d'un atome.
\item Il s'agit d'un atome d'hélium puisqu'il a deux protons (cf. tableau périodique) : $^\text{3}_\text{2}\text{He}$.
\item $m_\mathrm{atome} = 2 \times m_\mathrm{p} + 1 \times m_\mathrm{n} + 2 \times m_\mathrm{e} \approx \unit{5{,}02\times 10^{-27}}{kg}.$
\end{enumerate}

\section*{Exercice 2 -- Écriture conventionnelle}

\begin{enumerate}
\item

\begin{center}
\renewcommand\arraystretch{1}		% stretch table line height
\begin{tabular}{|l|c|c|c|c|}
\hline
\textbf{Symbole de l'élément}	& \quad{} B \quad{} & \quad{} F \quad{} & \quad{} H \quad{} & \quad{} Cr \quad{} \\
\hline
\textbf{Nombre de protons}		& 4 & 9 & 1 & 24 \\
\hline
\textbf{Nombre de neutrons}	& 5 & 10 & 2 & 28 \\
\hline
\textbf{Écriture conventionnelle du noyau} & $^{9}_{4}\text{B}$ & $^{19}_{9}\text{F}$ & $^{3}_{1}\text{H}$ & $^{52}_{24}\text{Cr}$ \\
\hline
\end{tabular}
\renewcommand\arraystretch{1.5}		% stretch table line height
\end{center}

\item L'atome de fluor possède 10 neutrons et 9 protons donc 9 électrons car un atome est toujours neutre (autant de protons que d'électrons).
L'ion fluorure a un électron en plus, donc 10 électrons au total.
L'ion fluorure possède :
\begin{itemize}
\item[•] 10 neutrons ;
\item[•] 9 protons ;
\item[•] 10 électrons.
\end{itemize}

\item L'ion fluorure a plus d'électrons que de protons, il est négatif : c'est un anion.

\item Il y a un seul électron de plus par rapport au nombre de protons :
\[\text{F}^\text{-}\]
\end{enumerate}

\section*{Exercice 3 -- Comparaison}

\begin{enumerate}
\item Pour comparer deux grandeurs, il faut qu'elles aient la même unité :
\[
r_\mathrm{atome} = \unit{110}{pm} = \unit{110\times 10^{-12}}{m}.
\]
On peut ensuite diviser la grande valeur par la petite :
\[
\frac{r_\mathrm{atome}}{r_\mathrm{noyau}} = \frac{110\times 10^{-12}}{2{,}5\times 10^{-15}} \approx 44\,000.
\]
L'atome de béryllium est 44\,000 fois plus grand que son noyau.

\item On retrouve bien que le noyau est \textbf{beaucoup} plus petit que l'atome (dans le cours on avait trouvé que le noyau était environ 100\,000 fois plus petit que l'atome).

\end{enumerate}

\end{document}