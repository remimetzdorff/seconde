\documentclass[12pt,a4paper,fleqn]{article}
\usepackage{rmpackages}																% usual packages
\usepackage{rmtemplate}																% graphic charter
\usepackage{rmexocptce}																% for DS with cptce eval

%\cfoot{} 													% if no page number is needed
%\renewcommand\arraystretch{1.5}		% stretch table line height

\begin{document}

\begin{header}
Devoir à la maison 2

Correction
\end{header}

\section*{Vacarme animalier}

\begin{enumerate}
\item
Les organes qui jouent le rôle de la fourche du diapason chez la cigale sont les timbales.
L'abdomen creux de l'insecte joue le rôle de caisse de résonance.

\item
L'animal qui émet le son le plus fort est celui qui est capable de produire le son dont le niveau d'intensité sonore est le plus élevé.
D'après les documents, c'est de loin le dauphin avec un niveau d'intensité sonore de \unit{220}{dB}.

\item
\textbf{Reformulation :}
\begin{itemize}
\item[•] La question est de déterminer la fréquence des deux signaux afin de les associer au bon animal compte tenu des informations des documents.
\end{itemize}

\textbf{Hypothèse :}
\begin{itemize}
\item[•] Je pense que l'animal 1 est l'éléphant car la période du signal bleu est plus grande que celle du signal orange (on le voit en regardant l'échelle des abscisses).
La fréquence du signal bleu est donc plus faible et d'après les documents 2 et 3, on sait que l'éléphant utilise des infrasons ($f<\unit{20}{Hz}$) alors que le dauphin utilise des ultrasons ($f=\unit{220}{kHz}$).
\end{itemize}

\textbf{Protocole}
\begin{itemize}
\item[•] On va mesurer la période de chacun des deux signaux, puis calculer leur fréquence.
\end{itemize}
\begin{multicols}{2}
\begin{center}
\textbf{Animal 1}
\end{center}
\begin{itemize}
\item[•] Sur le graphique on voit que \unit{2{,}3}{cm} correspondent à \unit{100}{ms} et qu'une période \og mesure \fg{} \unit{1{,}4}{cm} :
\begin{center}
\begin{tabular}{c|c}
\unit{2{,}3}{cm} & \unit{100}{ms} \\
\hline
\unit{1{,}4}{cm} & $T$
\end{tabular}
\end{center}
donc 
\[ T = \frac{1{,}4 \times 100}{2{,}3} \approx \unit{61}{ms} = \unit{6{,}1\times10^{-2}}{s}. \]

\item[•] Calcul de la fréquence :
\[ f = \frac{1}{T} = \frac{1}{6{,}1\times10^{-2}} \approx \unit{16}{Hz}. \]

\item[•] La fréquence du son émis par l'animal 1 est \unit{16}{Hz}.
\end{itemize}

\begin{center}
\textbf{Animal 2}
\end{center}
\begin{itemize}
\item[•] Sur le graphique on voit que \unit{1{,}6}{cm} correspondent à \unit{0{,}0050}{ms} et qu'une période \og mesure \fg{} \unit{1{,}4}{cm} :
\begin{center}
\begin{tabular}{c|c}
\unit{1{,}6}{cm} & \unit{0{,}0050}{ms} \\
\hline
\unit{1{,}4}{cm} & $T$
\end{tabular}
\end{center}
donc 
\[ T = \frac{1{,}4 \times 0{,}0050}{1{,}6} \approx \unit{0{,}0044}{ms} = \unit{4{,}4\times10^{-6}}{s}. \]

\item[•] Calcul de la fréquence :
\[ f = \frac{1}{T} = \frac{1}{4{,}4\times10^{-6}} \approx \unit{2{,}3\times10^5}{Hz}. \]

\item[•] La fréquence du son émis par l'animal 2 est \unit{2{,}3\times10^5}{Hz}.
\end{itemize}
\end{multicols}

\textbf{Conclusion :}
\begin{itemize}
\item[•]
Le son émis par l'animal 1 a une fréquence de \unit{16}{Hz} ce qui correspond à des infrasons, or d'après le document 3, l'éléphant émet des infrasons.
Le son émis par l'animal 2 a une fréquence de \unit{2{,}3\times10^5}{Hz} ce qui correspond à des ultrasons, or d'après le document 2, le dauphin émet des ultrasons.

Les hypothèses étaient donc justes.

L'animal 1 est un éléphant et l'animal 2 est un dauphin.
\end{itemize}

\item
Le timbre de ces sons n'est pas le même car la forme des signaux est différente.
Leur hauteur n'est pas la même non plus car ils ont des fréquences différentes.
\end{enumerate}

\section*{L'effet Donald Duck}

\begin{enumerate}[resume]
\item
L'hélium (He) permet d'obtenir une voix plus aigüe.

L'hexafluorure (SF\textsubscript{6}) de soufre permet d'obtenir une voix plus grave.

\item
D'après la vidéo le son se propage plus vite dans l'hélium que dans l'hexafluorure de soufre.

\item
\[ v = \frac{d}{\Delta t} = \frac{350}{0{,}34} \approx \unit{1{,}0\times10^3}{m\per s}.\]
La vitesse du son dans l'hélium est \unit{1{,}0\times10^3}{m\per s}.

\item
\[ \Delta t = \unit{7{,}5}{ms} = \unit{7{,}5\times 10^{-3}}{s}. \]
\[ v = \frac{d}{\Delta t} = \frac{1{,}0}{7{,}5\times 10^{-3}} \approx \unit{1{,}3\times10^2}{m\per s}.\]
La vitesse du son dans l'hexafluorure de soufre est \unit{1{,}3\times10^2}{m\per s}.

\item
La vitesse du son dans l'air est d'environ \unit{340}{m/s}.
Le son se propage donc plus vite dans l'hélium que dans l'air (environ trois fois plus vite), mais plus lentement dans l'hexafluorure de soufre que dans l'air (environ trois fois moins vite).

\item
Ces résultats sont bien en accord avec les affirmations de la vidéo puisque d'après nos calculs, la vitesse du son dans l'hélium est plus grande (presque dix fois plus grande) que la vitesse du son dans l'hexafluorure de soufre.
\end{enumerate}

\end{document}