\documentclass[12pt,a4paper]{article}
\usepackage[utf8]{inputenc}
\usepackage[french]{babel}
\usepackage[T1]{fontenc}
\usepackage{amsmath}
\usepackage{amsfonts}
\usepackage{amssymb}
\usepackage{graphicx}
\usepackage[left=2cm,right=2cm,top=3cm,bottom=2cm]{geometry}
\usepackage{multicol}
\usepackage[thinspace,thinqspace,amssymb]{SIunits}
\usepackage{pifont}
\usepackage{tikz}
\usepackage{multicol}
\usepackage{fourier}
\usepackage{setspace}
\usepackage{enumitem}

\usepackage{url}
\usepackage[breaklinks]{hyperref}
\hypersetup{
    colorlinks=true,
    linkcolor=red_f,
    citecolor=bleu_f,
    filecolor=green_f,
    urlcolor=bleu_f
}
\usepackage[hyphenbreaks]{breakurl}

\renewcommand{\familydefault}{\sfdefault}

\usepackage{xcolor}
\definecolor{gray_f}{RGB}{68,84,106}
\definecolor{gray_c}{RGB}{214,220,229}
\definecolor{bleu_f}{RGB}{91,155,213}
\definecolor{bleu_c}{RGB}{222,235,247}
\definecolor{red_f}{RGB}{204,0,0}
\definecolor{red_c}{RGB}{245,204,204}
\definecolor{orange_f}{RGB}{237,125,49}
\definecolor{orange_c}{RGB}{251,229,214}
\definecolor{green_f}{RGB}{112,173,71}
\definecolor{green_c}{RGB}{226,240,217}
\definecolor{yellow_f}{RGB}{255,192,0}
\definecolor{yellow_c}{RGB}{255,242,204}

\usepackage{fancyhdr}
\pagestyle{fancy}
\lhead{\textcolor{gray_f}{Physique-Chimie\\R. METZDORFF}}
\chead{\textcolor{gray_f}{Lycée Suzanne Valadon}}
\rhead{\textcolor{gray_f}{2020-2021}}
\renewcommand{\headrulewidth}{0.4pt}
\let\HeadRule\headrule
\renewcommand\headrule{\color{gray_f}\HeadRule}

%%%%% New environnements
\usepackage[framemethod=tikz]{mdframed}
\usepackage{chngcntr}

%%% header
\mdfdefinestyle{s_head}{%
	linecolor=gray_f!,
	outerlinewidth=3pt,%
	frametitlerule=false,
	topline=false,
	bottomline=false,
	rightline=false,
	leftline=false,
	backgroundcolor=gray_c,
	innertopmargin=8pt,
	roundcorner=0pt,
	nobreak=true,
	fontcolor=gray_f
}
\newmdenv[style=s_head]{header_env}
\newenvironment{header}
{%\stepcounter{exa}%
	\addcontentsline{ldf}{figure}{0}%
	\begin{header_env}\centering\LARGE\bf}
	{\end{header_env}}
	
%%% definition
\mdfdefinestyle{s_def}{%
	linecolor=red_f!,
	outerlinewidth=3pt,%
	frametitlerule=false,
	topline=false,
	bottomline=false,
	rightline=false,
	leftline=true,
	backgroundcolor=red_c,
	innertopmargin=8pt,
	roundcorner=0pt,
	nobreak=true,
	fontcolor=red_f
}
\newmdenv[style=s_def]{def_env}
\newenvironment{definition}
{%\stepcounter{exa}%
	\addcontentsline{ldf}{figure}{0}%
	\begin{def_env}\textbf{Définition :}}
	{\end{def_env}}

%%% exemple
\mdfdefinestyle{s_ex}{%
	linecolor=gray_f!,
	outerlinewidth=3pt,%
	frametitlerule=false,
	topline=false,
	bottomline=false,
	rightline=false,
	leftline=true,
	backgroundcolor=gray_c,
	innertopmargin=8pt,
	roundcorner=0pt,
	nobreak=true,
	fontcolor=gray_f
}
\newmdenv[style=s_ex]{ex_env}
\newenvironment{exemple}
{%\stepcounter{exa}%
	\addcontentsline{ldf}{figure}{0}%
	\begin{ex_env}\textbf{Exemple :}}
	{\end{ex_env}}

%%% analyse a priori
\mdfdefinestyle{s_prior}{%
	linecolor=green_f!,
	outerlinewidth=3pt,%
	frametitlerule=false,
	topline=false,
	bottomline=false,
	rightline=false,
	leftline=true,
	backgroundcolor=green_c,
	innertopmargin=8pt,
	roundcorner=0pt,
	nobreak=true,
	fontcolor=green_f
}
\newmdenv[style=s_prior]{prior_env}
\newenvironment{prior}
{%\stepcounter{exa}%
	\addcontentsline{ldf}{figure}{0}%
	\begin{prior_env}\textbf{A priori :}}
	{\end{prior_env}}

%%% analyse a posteriori
\mdfdefinestyle{s_post}{%
	linecolor=green_f!,
	outerlinewidth=3pt,%
	frametitlerule=false,
	topline=false,
	bottomline=false,
	rightline=false,
	leftline=true,
	backgroundcolor=green_c,
	innertopmargin=8pt,
	roundcorner=0pt,
	nobreak=true,
	fontcolor=green_f
}
\newmdenv[style=s_post]{post_env}
\newenvironment{post}
{%\stepcounter{exa}%
	\addcontentsline{ldf}{figure}{0}%
	\begin{post_env}\textbf{A posteriori :}}
	{\end{post_env}}
	
%%% Experience

\mdfdefinestyle{s_experience}{%
	linecolor=bleu_f!,
	outerlinewidth=3pt,%
	frametitlerule=false,
	topline=false,
	bottomline=false,
	rightline=false,
	backgroundcolor=bleu_c,
	innertopmargin=8pt,
	roundcorner=0pt,
	nobreak=true
}
\newmdenv[style=s_experience]{experience_env}
\newenvironment{experience}
{%\stepcounter{exa}%
	\addcontentsline{ldf}{figure}{0}%
	\begin{experience_env}}
%	\begin{experience_env}[]{\noindent\colorbox[rgb]{0.1 0.1 0.53}{\textbf{\color{white} Expérience : }}\\}}
	{\end{experience_env}}

%%% Slide

\mdfdefinestyle{s_slide}{%
	linecolor=green_f!,
	outerlinewidth=3pt,%
	frametitlerule=false,
	topline=false,
	bottomline=false,
	rightline=false,
	backgroundcolor=green_c,
	innertopmargin=8pt,
	roundcorner=0pt,
	nobreak=true
}
\newmdenv[style=s_slide]{slide_env}

\newenvironment{slide}
	{%\stepcounter{exa}%
%		\newenvironment{myenv}{\begin{adjustwidth}{2cm}{}}{\end{adjustwidth}}
		\addcontentsline{ldf}{figure}{0}%
		\begin{slide_env}}
		{\end{slide_env}
	}

%%% Conseils
\mdfdefinestyle{s_conseil}{%
	linecolor=orange_f!,
	outerlinewidth=3pt,%
	frametitlerule=false,
	topline=false,
	bottomline=false,
	rightline=false,
	backgroundcolor=orange_c,
	innertopmargin=8pt,
	roundcorner=0pt,
	nobreak=true,
	fontcolor=orange_f
}
\newmdenv[style=s_conseil]{conseil_env}
\newenvironment{conseil}
{%\stepcounter{exa}%
	\addcontentsline{ldf}{figure}{0}%
	\begin{conseil_env}\textbf{Conseil :}}
	{\end{conseil_env}
	}

%%% Remarque
\mdfdefinestyle{s_remarque}{%
	linecolor=green_f!,
	outerlinewidth=3pt,%
	frametitlerule=false,
	topline=false,
	bottomline=false,
	rightline=false,
	backgroundcolor=green_c,
	innertopmargin=8pt,
	roundcorner=0pt,
	nobreak=true,
	fontcolor=green_f
}
\newmdenv[style=s_remarque]{remarque_env}
\newenvironment{remarque}
{%\stepcounter{exa}%
	\addcontentsline{ldf}{figure}{0}%
	\begin{remarque_env}\textbf{Remarque :}}
	{\end{remarque_env}
	}
	
%%% Document
\mdfdefinestyle{s_doc}{%
	linecolor=bleu_f!,
	outerlinewidth=1pt,%
	frametitlerule=false,
	topline=true,
	bottomline=true,
	rightline=true,
	backgroundcolor=white,
	innertopmargin=8pt,
	roundcorner=0pt,
	nobreak=true,
	fontcolor=black
}
\newmdenv[style=s_doc]{doc_env}
\newenvironment{doc}
{%\stepcounter{exa}%
	\addcontentsline{ldf}{figure}{0}%
	\begin{doc_env}\textbf{Document}}
	{\end{doc_env}
	}
	
%%% Données
\mdfdefinestyle{s_don}{%
	linecolor=bleu_f!,
	outerlinewidth=1pt,%
	frametitlerule=false,
	topline=true,
	bottomline=true,
	rightline=true,
	backgroundcolor=white,
	innertopmargin=8pt,
	roundcorner=0pt,
	nobreak=true,
	fontcolor=black
}
\newmdenv[style=s_don]{don_env}
\newenvironment{donnee}
{%\stepcounter{exa}%
	\addcontentsline{ldf}{figure}{0}%
	\begin{don_env}\textcolor{bleu_f}{\textbf{Données}}}
	{\end{don_env}
	}

%%%%% New command

\newcommand{\diazote}{\text{N}_2}
\newcommand{\dioxygene}{\text{O}_2}
\newcommand{\dioxydedecarbone}{\text{CO}_2}
\newcommand{\eau}{\text{H}_2\text{O}}
\newcommand{\chlorure}{\text{Cl}^-}
\newcommand{\prof}[1]{\textcolor{gray_f}{\textit{#1}}}

\newcommand{\app}{\colorbox{bleu_c}{\textcolor{bleu_f}{APP}}}
\newcommand{\rea}{\colorbox{yellow_c}{\textcolor{yellow_f}{REA}}}
\newcommand{\anarai}{\colorbox{green_c}{\textcolor{green_f}{ANA-RAI}}}
\newcommand{\val}{\colorbox{orange_c}{\textcolor{orange_f}{VAL}}}
\newcommand{\com}{\colorbox{red_c}{\textcolor{red_f}{COM}}}
\newcommand{\auto}{\colorbox{white}{\textcolor{black}{AUTO}}}
\newcommand{\rco}{\colorbox{gray_c}{\textcolor{gray_f}{RCO}}}

\newcommand{\cmark}{\ding{51}}%
\newcommand{\xmark}{\ding{55}}


%\cfoot{} %% if no page number is needed
\usepackage{setspace}
\usepackage{enumitem}

\begin{document}

\begin{header}
Devoir à la maison 1 -- Correction
\normalsize
\flushleft
%\begin{doublespace}
%Classe :
%
%NOM :
%
%\end{doublespace}
%Prénom :
\end{header}
%L'énoncé est à rendre avec la copie : indiquez vos nom et prénom sur l'énoncé.

%\noindent
%La propreté de la copie (tenue, mise en valeur des résultats, orthographe) sera valorisée dans la notation.

\noindent

\section{Verrerie de précision}

\begin{enumerate}
\item La formule de la masse volumique est
\begin{equation}
\rho = \frac{m}{V}
\nonumber
\end{equation}
Si $m$ est exprimé en kg et $V$ en L, alors $\rho$ est en kg/L.

\item La masse volumique de l'eau est $\rho_\mathrm{eau} = \unit{1{,}0}{\kilo\gram\per\liter}$.
\label{quest:masse_volumique_eau}

\item Il faut commencer par faire une conversion pour exprimer le volume en L :
\begin{equation}
V = \unit{10}{mL}=\unit{0{,}010}{L}.
\nonumber
\end{equation}
On fait ensuite un \og produit en croix\fg{} pour exprimer la masse en fonction de la masse volumique et du volume, puis on remplace par les valeurs numériques :
\begin{equation}
m = \rho \times V = 1{,}0 \times 0{,}010 = \unit{0{,}010}{kg}.
\nonumber
\end{equation}
On termine par une conversion pour exprimer le résultat en grammes :
\begin{equation}
m = \unit{0{,}010}{kg} = \unit{10}{g}.
\nonumber
\end{equation}
La masse de \unit{10}{\milli\liter} d'eau est \unit{10}{g}.
\label{quest:masse_10mL_eau}

\item Ces résultats sont en accord avec le résultat de la question~\ref{quest:masse_10mL_eau} car les valeurs obtenues sont proches de \unit{10}{g}.
La moyenne des valeurs mesurées avec chaque cas est bien d'environ \unit{10}{g}.

\emph{Certains d'entre vous ont été jusqu'à faire un calcul d'écart relatif, c'est très bien !}

\item La verrerie la plus précise est la pipette jaugée car les valeurs mesurées sont les plus proches proches de \unit{10}{g} et sont peu dispersées.

À l'inverse, le bécher est le moins précis puisque les valeurs mesurées sont parfois assez éloignées de \unit{10}{g}, elles sont très dispersées.

\item Oui car on avait noté lors du TP que la verrerie avec l'incertitude relative la plus faible (la plus précise donc) était bien la pipette jaugée.

À l'inverse le bécher avait l'incertitude relative la plus grande ce qui en fait donc bien une verrerie peu précise.

\item Voir correction en classe.

\item En regardant l'axe des abscisses, on voit qu'avec la verrerie B, on obtient les valeurs les plus dispersées : c'est donc le bécher.

Avec la verrerie C, les valeurs sont les moins dispersées : c'est la pipette jaugée.

Par élimination, la verrerie A est donc l'éprouvette graduée.
\end{enumerate}

\section{Encore un peu d'eau salée}

\begin{enumerate}[resume]
\item La formule de la concentration massique est :
\begin{equation}
C_\mathrm{m} = \frac{m_\mathrm{soluté}}{V_\mathrm{solution}}.
\nonumber
\end{equation}
Pour déterminer la masse de sel à peser, on fait donc un \og produit en croix \fg{} pour trouver l'expression de $m_\mathrm{soluté}$.
En utilisant par exemple la méthode du triangle, on trouve :
\begin{equation}
m_\mathrm{soluté} = C_\mathrm{m} \times V_\mathrm{solution}.
\nonumber
\end{equation}
Avant de remplacer par les valeurs numériques, il faut convertir le volume de solution à préparer en L :
\begin{equation}
V_\mathrm{solution} = \unit{50}{mL} = \unit{0{,}050}{L}.
\nonumber
\end{equation}
Finalement, on peut faire l'application numérique :
\begin{equation}
m_\mathrm{soluté} = 150 \times 0{,}050 = \unit{7{,}5}{g}.
\nonumber
\end{equation}
Il faut peser \unit{7{,}5}{g} de sel pour préparer cette solution.

\item Voir fiche technique dissolution.

\item Voir fiche technique dissolution.

\item C'est une dissolution.

\item On exprime la masse en kg et le volume en L pour obtenir la masse volumique en kg/L :
\begin{equation}
V = \unit{50}{mL} = \unit{0{,}050}{L} \quad ; \quad m = \unit{54{,}9}{g} = \unit{0{,}0549}{kg}.
\nonumber
\end{equation}
On peut ensuite utiliser la formule de la masse volumique rappelée à la question 1 :
\begin{equation}
\rho = \frac{m}{V} = \frac{0{,}0549}{0{,}050} = \unit{1{,}098}{kg/L}.
\nonumber
\end{equation}
La masse volumique de la solution est \unit{1{,}098}{kg/L}.

\item Si la concentration massique de la solution était $C_\mathrm{m} = \unit{0}{g/L}$, la solution est en fait de l'eau pure.
Sa masse volumique est donc $\rho = \unit{1}{kg/L}$ comme on l'a dit à la question~\ref{quest:masse_volumique_eau}.

\item Par une construction graphique adaptée (voir correction en classe), on voit que les valeurs trouvées précédemment sont bien en accord avec les valeurs lues sur la courbe :
\begin{itemize}
\item[•] Pour $C_\mathrm{m} = \unit{0}{g/L}$ on lit en effet $\rho = \unit{1}{kg/L}$ ;
\item[•] Pour $C_\mathrm{m} = \unit{150}{g/L}$ on lit $\rho \approx \unit{1{,}1}{kg/L}$, ce qui est proche de la valeur trouvée précédemment.
\end{itemize} 

\item On voit sur la courbe qu'une solution d'eau salée de masse volumique \unit{1{,}175}{kg/L} a une concentration massique en sel de \unit{275}{g/L}.

La masse de sel dans 1 litre de mer Morte est donc \unit{275}{g}.

\item On pourrait par exemple faire bouillir \unit{1}{L} de mer Morte jusqu'à évaporation complète de l'eau et peser le sel qui reste dans le récipient.

\item Il n'est pas possible de dissoudre indéfiniment du sel dans une quantité d'eau fixée.
La concentration massique de \unit{358{,}5}{g/L} est la concentration maximale de sel qu'il est possible d'obtenir en solution aqueuse.
\end{enumerate}

\end{document}