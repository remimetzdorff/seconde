\documentclass[12pt,a4paper]{article}
\usepackage{rmpackages}																% usual packages
\usepackage{rmtemplate}																% graphic charter
\usepackage{rmexocptce}																% for DS with cptce eval

\cfoot{} 													% if no page number is needed
\renewcommand\arraystretch{1.}		% stretch table line height

\begin{document}

\section*{Perseverance : rapide ou lent ? (Bonus)}

Le 18 février 2021, après un voyage de 171 jours et $\unit{470\times10^6}{km}$, le rover Perseverance s'est finalement posé sur Mars.
\begin{enumerate}
\item \rea{} 

Calculer la vitesse moyenne de Perseverance, en km/h, pendant son trajet depuis la Terre vers Mars.
\end{enumerate}

Une fois posé, le rover effectue des tests pour vérifier qu'il fonctionne normalement.
Il s'est ainsi déplacé sur le sol martien de \unit{6{,}5}{m} en \unit{33}{min}.

\begin{enumerate}[resume]
\item \rea{}

Calculer la vitesse moyenne de Perseverance, en m/s, pendant ce test.

\item \val{}

D'après les valeurs du tableau ci-dessous, de quelle vitesse se rapproche le plus chacune des deux valeurs trouvées précédemment.
\end{enumerate}

\begin{center}
\begin{tabular}{|l|c|c|}
\hline
													& \textbf{Vitesse}		& \textbf{Vitesse} \\
													& (m/s)										& (km/h) \\
\hline
\hline
\textbf{Lumière dans le vide}	& $3\times10^8$		& $10^9$ \\
\textbf{Terre autour du Soleil}	& $30\times10^3$		& $107\times 10^3$\\
\textbf{Son dans l'air}				& $340$						& $1220$ \\
\textbf{Guépard}						& $30$						& $110$ \\
\textbf{Escargot}						& $0{,}005$	& $0{,}002$ \\
\hline

\end{tabular}
\end{center}

\hrule

\section*{Perseverance : rapide ou lent ? (Bonus)}

Le 18 février 2021, après un voyage de 171 jours et $\unit{470\times10^6}{km}$, le rover Perseverance s'est finalement posé sur Mars.
\begin{enumerate}
\item \rea{} 

Calculer la vitesse moyenne de Perseverance, en km/h, pendant son trajet depuis la Terre vers Mars.
\end{enumerate}

Une fois posé, le rover effectue des tests pour vérifier qu'il fonctionne normalement.
Il s'est ainsi déplacé sur le sol martien de \unit{6{,}5}{m} en \unit{33}{min}.

\begin{enumerate}[resume]
\item \rea{}

Calculer la vitesse moyenne de Perseverance, en m/s, pendant ce test.

\item \val{}

D'après les valeurs du tableau ci-dessous, de quelle vitesse se rapproche le plus chacune des deux valeurs trouvées précédemment.
\end{enumerate}

\begin{center}
\begin{tabular}{|l|c|c|}
\hline
													& \textbf{Vitesse}		& \textbf{Vitesse} \\
													& (m/s)										& (km/h) \\
\hline
\hline
\textbf{Lumière dans le vide}	& $3\times10^8$		& $10^9$ \\
\textbf{Terre autour du Soleil}	& $30\times10^3$		& $107\times 10^3$\\
\textbf{Son dans l'air}				& $340$						& $1220$ \\
\textbf{Guépard}						& $30$						& $110$ \\
\textbf{Escargot}						& $0{,}005$	& $0{,}002$ \\
\hline

\end{tabular}
\end{center}

\end{document}