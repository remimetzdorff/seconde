\documentclass[12pt,a4paper]{article}
\usepackage{rmpackages}																% usual packages
\usepackage{rmtemplate}																% graphic charter
\usepackage{rmexocptce}																% for DS with cptce eval

\cfoot{} 													% if no page number is needed
%\renewcommand\arraystretch{1.5}		% stretch table line height

\begin{document}

\begin{header}
Les puissances de 10
\end{header}

\begin{figure}[h]
\center
\includegraphics[scale=0.05]{images/qr_puissance_de_10.png}
\caption{Lien vers la \href{https://youtu.be/Rn91_g60EsI}{vidéo sur les puissances de 10.}}
\end{figure}

\section*{Les puissances de 10}

\begin{definition}
Une puissance de 10 permet de décaler la virgule.
\end{definition}

\begin{Huge}
\[
\textcolor{black}{3{,}45} \textcolor{red}{\times 10} ^ { \textcolor{blue}{-} \textcolor{green_f}{2} }
\]
\end{Huge}

Pour passer d'un nombre écrit avec une puissance de 10 à un nombre sans puissance de 10, on va commencer par remplacer l'écriture par une phrase :
\begin{enumerate}
\color{red}
\item Je décale la virgule
\color{blue}
\item vers la gauche
\color{green_f}
\item de deux rangs
\color{black}
\item à partir de là où elle est dans 3{,}45.
\end{enumerate}
\og Je décale la virgule vers la gauche de deux rangs à partir de là où elle est dans 3{,}45. \fg{}

En faisant exactement ce qu'indique la phrase, j'obtiens donc :
\[
3{,}45 \times 10^{-2} = 0{,}0345.
\]

\paragraph*{Exemples :} Pour chaque exemple, écris la phrase commençant par \og je décale la virgule... \fg{} puis réécris le nombre, mais sans utiliser de puissance de 10.
\begin{itemize}
\item[•] $2{,}8\times10^{+1}$
\item[•] $88\times10^{-3}$ \hfill \emph{aide : il faut remarquer que $88 = 88{,}0$}
\item[•] $3\times10^{8}$
\item[•] $10^{-4}$ \hfill \emph{aide : ajouter $1{,}0\times$ devant ne change pas la valeur du nombre}
\end{itemize}


\flushright
\footnotesize
\begin{turn}{180}
Réponses :
$2{,}8\times10^{+1} = 28$ \quad ; \quad
$88\times10^{-3} = 0{,}088$ \quad ; \quad
$3\times10^{8} = 300\, 000\, 000$ \quad ; \quad
$10^{-4} = 0{,}0001$
\hfill
\end{turn}


\end{document}