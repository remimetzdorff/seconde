\include{template}
\cfoot{} %% if no page number is needed

\begin{document}

\begin{header}
FICHE MÉTHODE

Rédiger un compte-rendu en physique-chimie
\end{header}

\section*{Le sujet}

Le sujet du TP ou de l'activité est souvent formulé comme une question assez ouverte :
\begin{itemize}
\item[•] Qu'y a-t-il dans l'air ?
\item[•] Pourquoi le ciel est-il bleu ?
\item[•] Comment les arc-en-ciel se forment-ils ?
\item[•] Comment mesurer une température ?
\end{itemize}

\section*{Les hypothèses}

Vous devrez toujours essayer de donner vos \textbf{hypothèses} : c'est ce que vous pensez de la question.
Les hypothèses doivent être \textbf{justifiées}.
\begin{center}
\emph{Je pense que ... car ...}
\end{center}

\section*{Mes hypothèses sont-elles justes ?}

Vous commencerez par expliquer en quelques lignes ce que vous voulez faire pour savoir si les hypothèses sont justes ou fausses.

Pour savoir si vos hypothèses son justes ou non, il  faudra établir un \textbf{protocole}. Un protocole est composé de toutes les étapes qui permettent de répondre à une question. Il peut contenir s'il est nécessaire de faire une expérience :
\begin{itemize}
\item[•] une \textbf{liste de matériel} ;
\item[•] un \textbf{schéma} de l'expérience ;
\item[•] la réalisation de l'\textbf{expérience} ;
\item[•] le relevé des \textbf{mesures} utiles ;
\item[•] les \textbf{observations} associées à l'expérience : on pourra utiliser le \textbf{schéma narratif} pour cette partie.
\end{itemize}
Un protocole peut aussi contenir simplement :
\begin{itemize}
\item[•] des \textbf{calculs} ;
\item[•] des \textbf{raisonnements} ;
\item[•] une \textbf{argumentation} ;
\item[•] un \textbf{modèle informatique} ;
\item[•] etc.
\end{itemize}

\section*{Et pour finir...}

Pour terminer un compte-rendu :
\begin{itemize}
\item[•] il faut donner des \textbf{conclusions} en reprenant ce qui a été trouvé dans le protocole ;
\item[•] il faut \textbf{répondre à la question} posée.
\end{itemize}

\end{document}