\include{template}
\cfoot{} %% if no page number is needed

\begin{document}

\begin{header}
FICHE INFO

Les compétences à acquérir en seconde
\end{header}

\begin{center}
\begin{tabular}{l|l}
\textbf{Compétences} & Exemples d'aptitudes à vérifier \quad \quad \quad \quad \quad \quad \quad \quad \quad \textbf{Suis-je capable de ... ?} \\
\hline
\hline
S'approprier & Respecter les consignes données dans l'énoncé \\
\app         & Me servir correctement des ressources disponibles (doc, énoncé, ...) \\
             & Choisir les informations qui me seront utiles \\
             & Avoir une attitude critique et réfléchie sur les documents proposés \\
\hline
Réaliser     & Reconnaitre sur un schéma les instruments et appareils \\
\rea         & Respecter les consignes de sécurité \\
             & Compléter correctement un schéma d'un dispositif expérimental \\
             & Réaliser des gestes précis à partir d'un protocole détaillé donné \\
             & Réaliser un schéma correct d'un dispositif expérimental \\
             & Réaliser le bon montage correspondant à l'expérience proposée \\
             & Réaliser le schéma correspondant à mon hypothèse \\
             & Maîtriser des gestes techniques \\
             & Faire des observations utiles pour l'activité \\
             & Effectuer des procédures classiques (calculs, représentations, mesures, etc.) \\
\hline
Analyser --  & Faire une hypothèse, la justifier \\
Raisonner    & Identifier un problème \\
\annrai      & Proposer une méthode pour vérifier mon hypothèse \\
             & Utiliser, choisir ou élaborer un modèle adapté pour répondre à une question \\
             & Faire des prévisions à partir d'un modèle \\
             & Justifier le protocole choisi\\
             & Élaborer un protocole qui répond à la question \\
             & Trouver les paramètres qui influencent un phénomène \\
             & Mener correctement les mesures \\
             & Donner des ordres de grandeur des valeurs mesurées \\
\hline
Valider      & Estimer l'ordre de grandeur de l'incertitude de la mesure \\
\val         & Estimer l'incertitude d'une série de mesures \\
             & Écrire mon résultat de mesure avec l'incertitude associée \\
             & Dire si mes résultats sont en accord avec ceux attendus \\
             & Avoir un regard critique sur mes résultats \\
             & Trouver des solutions pour améliorer la démarche ou le modèle utilisé\\
             & Extraire et exploiter des informations des données expérimentales \\
\hline
Autonomie--  & Travailler en autonomie ou en équipe \\
Initiative   & Être curieux et créatif \\
\auto        & Prendre des initiatives, des décisions, anticiper\\
             & Gérer mon temps \\
\hline
Communiquer  & Rendre compte de façon écrite ou orale \\
\com         & Échanger entre pairs \\
             & Utiliser un vocabulaire et des modes de représentation adaptés \\
\hline
Mobiliser ses& Argumenter \\
Connaissances& Démontrer \\
\rco         & Restituer ses connaissances \\
\end{tabular}
\end{center}

\end{document}