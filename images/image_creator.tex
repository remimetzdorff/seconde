\documentclass[varwidth,preview,border=5mm,convert={density=600,outext=.png}]{standalone}
\usepackage{tikz}
\usepackage{rmtemplate}
\usepackage{chemfig}

\begin{document}


\end{document}

%propane
\chemfig[angle increment=90, atom sep=20pt]{H-C([-1]-H)([1]-H)-C([-1]-H)([1]-H)-C([-1]-H)([1]-H)-H}

%dioxygène
\chemfig[angle increment=90, atom sep=20pt]{\lewis{35, O}=\lewis{17,O}}

%acide fluoridrique
\chemfig[angle increment=90, atom sep=20pt]{H-\lewis{026,F}}

%sulfure d'hydrogène
\chemfig[angle increment=90, atom sep=20pt]{H-\lewis{26,S}-H}

%molécule d'eau coudée
\chemfig[atom sep=20pt]{H-\lewis{26,O}-H}

%molécule d'eau coudée
\chemfig[atom sep=20pt]{H-[:30]\lewis{13,O}-[:-30]H}

%classification périodique
\newcommand{\y}{\linewidth/8.01}
\newcommand{\z}{\y/2}
\begin{tikzpicture}
\draw [color=white,fill=white] (0,0) rectangle ++ (8*\y, 3.25*\y);

\foreach \i in {0,1, 2} {
  \draw [fill=red_c] (0, \i*\y) rectangle ++ (\y, \y);
}
\foreach \i in {0,1} {
  \draw [fill=orange_c] (\y, \i*\y) rectangle ++ (\y, \y);
}
\foreach \i in {0,1} {
  \draw [fill=gray_c] (2*\y, \i*\y) rectangle ++ (\y, \y);
}
\foreach \i in {0,1} {
  \draw [fill=gray_c] (3*\y, \i*\y) rectangle ++ (\y, \y);
}
\foreach \i in {0,1} {
  \draw [fill=gray_c] (4*\y, \i*\y) rectangle ++ (\y, \y);
}
\foreach \i in {0,1} {
  \draw [fill=yellow_c] (5*\y, \i*\y) rectangle ++ (\y, \y);
}
\foreach \i in {0,1} {
  \draw [fill=green_c] (6*\y, \i*\y) rectangle ++ (\y, \y);
}
\foreach \i in {0,1,2} {
  \draw [fill=bleu_c] (7*\y, \i*\y) rectangle ++ (\y, \y);
}

\draw [line width=3pt] (2\y, 0) -- ++ (0, 2\y);

\draw (\z, 3*\y) node [below] {Hydrogène};
\draw (\z, 2.5*\y) node [] {\Large $\mathrm{H}$};
\draw (\z, 2*\y) node [below] {Lithium};
\draw (\z, 1.5*\y) node [] {\Large $\mathrm{Li}$};
\draw (\z, 1*\y) node [below] {Sodium};
\draw (\z, .5*\y) node [] {\Large $\mathrm{Na}$};

\draw (1.5\y, 2*\y) node [below] {Béryllium};
\draw (1.5\y, 1.5*\y) node [] {\Large $\mathrm{Be}$};
\draw (1.5\y, 1*\y) node [below] {Magnésium};
\draw (1.5\y, .5*\y) node [] {\Large $\mathrm{Mg}$};

\draw (2.5\y, 2*\y) node [below] {Bore};
\draw (2.5\y, 1.5*\y) node [] {\Large $\mathrm{B}$};
\draw (2.5\y, 1*\y) node [below] {Aluminium};
\draw (2.5\y, .5*\y) node [] {\Large $\mathrm{Al}$};

\draw (3.5\y, 2*\y) node [below] {Carbone};
\draw (3.5\y, 1.5*\y) node [] {\Large $\mathrm{C}$};
\draw (3.5\y, 1*\y) node [below] {Silicium};
\draw (3.5\y, .5*\y) node [] {\Large $\mathrm{Si}$};

\draw (4.5\y, 2*\y) node [below] {Azote};
\draw (4.5\y, 1.5*\y) node [] {\Large $\mathrm{N}$};
\draw (4.5\y, 1*\y) node [below] {Phosphore};
\draw (4.5\y, .5*\y) node [] {\Large $\mathrm{P}$};

\draw (5.5\y, 2*\y) node [below] {Oxygène};
\draw (5.5\y, 1.5*\y) node [] {\Large $\mathrm{O}$};
\draw (5.5\y, 1*\y) node [below] {Soufre};
\draw (5.5\y, .5*\y) node [] {\Large $\mathrm{S}$};

\draw (6.5\y, 2*\y) node [below] {Fluor};
\draw (6.5\y, 1.5*\y) node [] {\Large $\mathrm{F}$};
\draw (6.5\y, 1*\y) node [below] {Chlore};
\draw (6.5\y, .5*\y) node [] {\Large $\mathrm{Cl}$};

\draw (7.5\y, 3*\y) node [below] {Hélium};
\draw (7.5\y, 2.5*\y) node [] {\Large $\mathrm{He}$};
\draw (7.5\y, 2*\y) node [below] {Néon};
\draw (7.5\y, 1.5*\y) node [] {\Large $\mathrm{Ne}$};
\draw (7.5\y, 1*\y) node [below] {Argon};
\draw (7.5\y, .5*\y) node [] {\Large $\mathrm{Ar}$};

\draw (.5\y, 3\y) node [above, color=red_f] {\large\textbf{1}};
\draw (1.5\y, 2\y) node [above, color=red_f] {\large\textbf{2}};
\draw (7.5\y, 3\y) node [above, color=red_f] {\large\textbf{18}};
\foreach \i in {13,14,...,17} {
  \draw (\i*\y-10.5\y, 2\y) node [above, color=red_f] {\large\textbf{\i}};
}
\end{tikzpicture}

% trajectoire rectiligne accélérée
\begin{tikzpicture}
\draw [color=bleu_f] (0, 0) node {$\bullet$};
\draw [color=bleu_f] (0, 0) node [above left] {$M_0$};
\foreach \t in {1,2,...,6} {
  \newcommand{\x}{\t*\t/4}
  \draw [color=bleu_f] (\x, 0) node {$\bullet$};
  \draw [color=bleu_f] (\x, 0) node [above] {$M_\t$};
}
\end{tikzpicture}

% vecteur vitesse horizontal
\begin{tikzpicture}
\draw [->, >=stealth, color=bleu_f, ultra thick] (0,0) -- (2,0) node [midway, above] {$\vec{v}$};
\end{tikzpicture}