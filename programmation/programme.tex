\documentclass[12pt,a4paper]{article}
\usepackage[utf8]{inputenc}
\usepackage[french]{babel}
\usepackage[T1]{fontenc}
\usepackage{amsmath}
\usepackage{amsfonts}
\usepackage{amssymb}
\usepackage{graphicx}
\usepackage[left=2cm,right=2cm,top=2cm,bottom=2cm]{geometry}
%\author{Rémi Metzdorff}
\title{Programme de physique-chimie de seconde générale et technologique}
\renewcommand{\familydefault}{\sfdefault}
\date{}

\begin{document}

\maketitle

\section*{Constitution et transformations de la matière}
  \subsection*{Constitution de la matière de l'échelle macroscopique à l'échelle microscopique}
    \subsubsection*{Description et caractérisation de la matière à l'échelle macroscopique}
    \subsubsection*{Modélisation de la matière à l'échelle microscopique}
  \subsection*{Modélisation des transformations de la matière et transferts d'énergie}
    \subsubsection*{Transformation physique}
    \subsubsection*{Transformation chimique}
    \subsubsection*{Transformation nucléaire}

\section*{Mouvement et interactions}
\subsection*{Décrire un mouvement}
\subsection*{Modéliser une action sur un système}
\subsection*{Principe d'inertie}

\section*{Ondes et signaux}
\subsection*{Émission et perception d'un son}
\subsection*{Vision et image}
\subsection*{Signaux et capteurs}

Robot suiveur de ligne :
Les élèves réalisent les yeux et le cerveau du robot, ils le testent sur un corps réalisé par l'enseignant.

\end{document}